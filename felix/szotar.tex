\documentclass[a4paper]{memoir}
\usepackage{charter}
\usepackage[utf8]{inputenc}
\usepackage{t1enc}
\usepackage[hungarian]{babel}
\usepackage[final]{microtype}
\usepackage{multicol}
\usepackage{ifthen}
\isopage[12]
\checkandfixthelayout

\title{Apétól Manyáig}
\author{Félix--magyar szótár, 2016 február}
\date{}

\newcommand{\f}[1]{\textbf{#1}} % Félix formázás.
\newcommand{\e}{\hangpara{1.5em}{1}} % Szótár bejegyzés.
\newcommand{\felix}[1]{\e \f{#1} \hspace{1ex}} % Félix bejegyzés.
\newcommand{\feli}[1]{\hspace{1ex} \f{#1} \hspace{1ex}} % További Félix kifejezés.
\newcommand{\magyar}[1]{\ifthenelse{\equal{#1}{1}}{}{\hspace{1ex}\ }\textbf{#1.}} % Magyar jelentés száma.
\newcommand{\pelda}[1]{\textit{``#1''}} % Példa mondat.

\begin{document}

\maketitle

\begin{multicols}{2}

\felix{ácci} virág

\felix{aggódó} aggódó tekintet

\felix{alló} telefon

\felix{alma} alma (viccesen \f{olmu}), \f{almafa}, \f{almalé}

\felix{apé} jelentése ismeretlen \pelda{Apé? -- Mányá!}

\felix{apáé}
  \magyar{1}~piros kisautó
  \magyar{2}~apa tulajdona

\felix{apuka} apa általában, különösen az \textit{Apanap} című könyvben

\felix{arra} arra \feli{erre} erre 

\felix{aú} fájdalom, baleset

\felix{autó}
  \magyar{1}~autó
  \magyar{2}~ajtó
  \feli{baba autó} babakocsi

\felix{autóbu} autóbusz

\felix{bácsi}
  \magyar{1}~kukásbácsi
  \magyar{2}~általános bácsi

\felix{béka} béka

\felix{bicik} bicikli

\felix{bika} bika (bármi aminek szarva van)

\felix{boj} bojt

\felix{bordó} bordó nadrág

\felix{cica} cica

\felix{cici} cici

\felix{csibe} csibe 

\felix{detete} jelentése ismeretlen, a spaletta tokjával kapcsolatos \pelda{Nini, detete.}

\felix{duda}
  \magyar{1}~WiFi router
  \magyar{2}~autóduda
  \magyar{3}~jelentése ismeretlen \pelda{Duda, duda!} \textit{(énekelve, lelkesen)}

\felix{-é} \textit{(képző)} valamilyen autó, például \f{kukáé}

\felix{effiki} jelentése ismeretlen

\felix{eldőlt} eldőlt, felborult, fejen áll, eltér a szokásos irányától

\felix{elme}
  \magyar{1}~menni, jönni, vagy egy pillanatra elhagyni a látóteret \pelda{Apa elme.} \textit{(szomorúan)}
  \magyar{2}~ceruzaelem

\felix{Eli} Allie Brosch karakter a \textit{Hyperbole and a Half}ból

\felix{ennyi} művelet vége

\felix{fa} fa

\felix{farka} farok

\felix{fáj} szűk

\felix{fehér} fekete

\felix{felhő} felhő, füst \pelda{Ritty-rotty. --- Felhő.}

\felix{fekete} fehér

\felix{fiki} takony

\felix{foka} fog

\felix{ga} sárga 

\felix{gitár}
  \magyar{1}~gitár
  \magyar{2}~G betű
  \feli{nem gitár} hegedű

\felix{haku} harkály

\felix{hattyú} hattyú

\felix{hápu} kacsa

\felix{harka} szarka

\felix{hó}
  \magyar{1}~hold
  \magyar{2}~homlok
  \magyar{3}~hó

\felix{hürke} szürke \pelda{Hüke autó elme.}

\felix{iá} szamár \pelda{Anya mint egy málhás szamár. -- Iá!}

\felix{ide} ide

\felix{igó} rigó

\felix{ikó} csikó

\felix{jalla} teve (Mamánál a teve ezt énekli)

\felix{ka}
  \magyar{1}~általános madár (valószínűleg a kacsából)
  \magyar{2}~kanál

\felix{kaka}
  \magyar{1}~kaki, pisi léte illetve gondolata
  \magyar{2}~tele van vagy lesz a pelenka

\felix{kati}
  \magyar{1}~katica
  \magyar{2}~bármilyen bogár
  \magyar{3}~Volkswagen bogár (\f{kati autó} is)

\felix{ké}
  \magyar{1}~kék
  \magyar{2}~\textit{Időkék dal} a Gryllus Vilmos CD-ről

\felix{kettő} általános számnév \pelda{Látod mennyi hó esik? -- Kettő!}

\felix{kém} krém

\felix{kémmoku}
  \magyar{1}~jelentése ismeretlen
  \magyar{2}~cumi
  \magyar{3}~Q betű

\felix{kész} művelet vége

\felix{khr(i)-khr(o)} traktor

\felix{kí - becsu} kinyit - becsuk 

\felix{kí-pok} "kipp-kopp", kinyitni valamit

\felix{kl-kl} kilincs

\felix{köt} vasúti szerelvény összekapcsolása

\felix{kubi} krumpli

\felix{kuka}
  \magyar{1}~kuka
  \magyar{2}~kukac (később \f{kukak})

\felix{kukáé}
  \magyar{1}~kukásautó
  \magyar{2}~teherautó

\felix{kukagyí} \textit{Kukásautó dal} (KerekMese) YouTube-on

\felix{kuko} kuki

\felix{kuta} kormány

\felix{kú} kulcs

\felix{le} le, lefele, vagy más irány \pelda{Hürke autó elme le.}

\felix{lé}
  \magyar{1}~gyümölcslé
  \magyar{2}~naracsszínű
  \magyar{3}~narancs, mandarin

\felix{lila} lila

\felix{makkó} mackó

\felix{mak mak}
  \magyar{1}~nyuszi
  \magyar{2}~fázik a lábam (csak a nyuszis papucs segíthet)

\felix{mama}
  \magyar{1}~anya
  \magyar{2}~nagymama
  \magyar{3}~másik nagymama
  \magyar{4}~apa (ha segítség kell és ő van kéznél)

\felix{mamáé}
  \magyar{1}~szürke kisautó
  \magyar{2}~anya tulajdona

\felix{mányá} jelentése ismeretlen \pelda{Mányá? -- Apé!}

\felix{meki} kecske

\felix{miau} cica \feli{ké miau} kék szánkó, akiért aggódik, mert kinn van a lépcsőházban

\felix{miku} Mikulás (bárki akin piros sapka van)

\felix{me}
  \magyar{1}~meleg
  \magyar{2}~hideg

\felix{még} kérek szépen, légy szíves

\felix{móka}
  \magyar{1}~mókus
  \magyar{2}~munka \pelda{Apa elme móka.}

\felix{Móka Miki} fejrázás jobbra-balra kiöltött nyelvvel

\felix{mú} tehén

\felix{na} nagy

\felix{nem}
  \magyar{1}~kérés nyugtázása
  \magyar{2}~kérés \pelda{Tea nem.}
  \magyar{3}~nem

\felix{nini} nini, kukucs

\felix{nínó autó} mentőautó, rendőrautó, tűzoltóautó, utcaseprő autó

\felix{nínó bácsi} tűzoltó, rendőr

\felix{nulla} nulla, O-betű

\felix{nyik} nyissz \textit{(lásd olló)}

\felix{o-ó} leesett, vagy más probléma van vele

\felix{odagada} odaragadt

\felix{oda}
  \magyar{1}~oda
  \magyar{2}~szóda

\felix{olló} olló \pelda{Olló. Nyik-nyik.}

\felix{pa}
  \magyar{1}~paradicsom
  \magyar{2}~paprika
  \magyar{3}~papagáj
  \magyar{4}~lámpa, zseblámpa

\felix{palinta} hinta, mérleghinta \pelda{Na palinta! Még palinta!}

\felix{papu} papucs, cipő

\felix{pé} pénz

\felix{peő} pehely (kukoricapehely, hópehely) \pelda{Hópeő!}

\felix{pici} kicsi \pelda{Pici mak-mak}

\felix{pí}
  \magyar{1}~piros
  \magyar{2}~pillangó

\felix{pipi} tojásban lakó plüsscsibe 

\felix{po} porszívó

\felix{pok}
  \magyar{1}~gombnyomás
  \magyar{2}~szappanbuborék

\felix{pók} pók

\felix{potya} postás, posta

\felix{pő} repülő

\felix{pötty}
  \magyar{1}~pont
  \magyar{2}~betenni

\felix{pszichopata} \textit{(bizonytalan)} indexlámpa

\felix{répu} répa

\felix{róka} \textit{(gutturális \textsc{r}-rel)} róka

\felix{ő}
  \magyar{1}~ez, különösen ha két dolog megegyezik \pelda{Ő ő.}
  \magyar{2}~zöld

\felix{tátá} nagypapa

\felix{tartó} csomagtartó

\felix{tea} tea 

\felix{teki} teknősbéka

\felix{téta} tészta

\felix{tiktak} óra, iránytű

\felix{to} csont

\felix{toj} toll, ceruza

\felix{tuttap} "tudtam!" felkiáltás \pelda{Apa! Tuttap!}

\felix{ükape} jelentése ismeretlen (néha a szivecskés pólós plüssmaci) 

\felix{va} varjú

\felix{vavau} kutya

\felix{vége} könyv vége

\felix{ví}
  \magyar{1}~víz
  \magyar{2}~világít

\felix{vijja}
  \magyar{1}~Villám, a Duplo autó
  \magyar{2}~villamos \feli{faf vijja} CAF villamos

\end{multicols}

\section*{Testbeszéd és egyebek}

\felix{hal hangjának utánzása}
  \magyar{1}~hal
  \magyar{2}~bálna
  \magyar{3}~málna

\felix{keserves nyávogás és tekergés} cica

\felix{két könyék összeérintése}
  \magyar{1}~tejföl
  \magyar{2}~más savanyú dolog vagy gondolat

\felix{röfögés és a föld túrása} malac

\felix{szuszogás}
  \magyar{1}~sündisznó
  \magyar{2}~vonat

\felix{evés közben szemdörgölés} nem kérek ilyet többet

\felix{szemöldökráncolás és szájlegörbítés} morcos malacok a \textit{Tesz vesz város}ból

\section*{2016 június}

\begin{multicols}{2}

\felix{nevet hív ennek} őt hogy hívják, mi a neve

\felix{fölkap a lámpát} kapcsold fel a lámpát

\felix{nem belépettem} nem léptem bele

\felix{kinyi ta-tót} nyisd ki az ajtót

\felix{becsükte} becsukta

\felix{mehnézem Olit} megnézem Olit

\end{multicols}

\section*{2016 szeptember}

\e \f{Mi a neve őnek?}

\e \f{Megmondod!}

\e \f{Picike vagy. Az a neved, Darabos Olivér.}

\e \f{Megfogod embert. Ember megnézi hogy megy a Mercedes.}

\e Melyik a kedvenc halad? \f{A ponty!} Melyik nem a kedvenced? \f{A durbincs.}

\e \f{Olika! Ütött az óra!}

\e Miért ütöd Olit? \f{Mert nem törékeny.}

\e \f{Antantémusz, szórakatiketuka labalabimbambusz vén ela kénkampusz, NÉGY/EF/Ó/MÁSIK Ó/TÁTÁ BETŰ/BUSZ!}

\end{document}
