\documentclass[12pt]{memoir}
\usepackage{titlesec}
\usepackage{t1enc}
\def\magyarOptions{hyphenmins=34}
\usepackage[magyar]{babel}
\usepackage[final]{microtype}
\usepackage{fontspec}
\usepackage{incgraph}

\clubpenalty=10000
\widowpenalty=10000
\raggedbottom

\setcounter{secnumdepth}{0}
\setstocksize{22.86cm}{15.24cm}
\settrimmedsize{\stockheight}{\stockwidth}{*}
\settrims{0pt}{0pt}
\setlrmarginsandblock{2cm}{1.25cm}{*}
\setulmarginsandblock{2.5cm}{*}{1}
\checkandfixthelayout
\setlength{\emergencystretch}{2em}
\setmainfont{Crimson Text}
\newfontfamily\playfair{Playfair Display SC}
\def\dash{---\kern 0.2em}

\newcommand{\beforesection}{\vspace*{0.25\textheight}}
\titleformat{\section}
  {\center\huge\bfseries\playfair}
  {\thesection}
  {1em}
  {\beforesection}
  [\vspace*{1em}]
\newcommand{\image}[1]{%
\incgraph[documentpaper][keepaspectratio=true,width=\paperwidth,height=\paperheight]{#1}}

\author{Darabos Félix}

\date{2018}

\title{Gepárd}

\begin{document}

\begin{titlingpage}
  \centering
  \vspace*{0.2\textheight}
  {\fontsize{80pt}{96pt}\playfair \thetitle}\\[\baselineskip]
  {\large\theauthor}\\[\baselineskip]
  \vfill
  {\thedate}
  \vspace*{0.1\textheight}
\end{titlingpage}

\pagestyle{plain}

\vspace*{0.3\textheight}{\begin{center}\textit{``Anya, te vagy a legszebb.''}\end{center}}

\cleartoverso
\section{Karácsonyváró}
A gepárd nem veszi ki a karácsonyváróból az ajándékokat, csak mikor már mind
tele van. Mert autókat szeretne találni, hogy nekem adja. Ő csak ruhákat akar
megtartani, mert nincs rózsaszín és lila ruhája.
\image{karácsonyváró.png}


\section{Kormányzár}
A gepárdnak is van kormányzára. Tudod, milyen színű a gepárd kormányzára? Lila!

Az oroszlán a gepárd barátja. Neki nincs olyan sok szép autója. Mindig el
akarja lopni a gepárd barna Lamborghinijét. Beül és el akarja indítani. De nem
indul az autó! Kormányzár!


\section{A gepárd és a macska}
A gepárd Olaszországban él. Lány, de mély hangja van. Hörr! Három nyelven
beszél: olaszul, angolul és magyarul. A gepárd már 18 éves. Egyetemista.

A gepárd elég gazdag. Nagyon sok autója van. De csak a gyors sportkocsikat
gyűjti. Van egy felhőkarcolója. Az alján van a garázsban a sok autó. És ő fent
lakik, a legtetején.

A Datsun elég ritka autó. Csak éjszaka lehet látni. Éjfélkor merészkedik csak
elő. A gepárd ki szokott menni éjszaka az utcára, hogy megnézze a Datsunt.
Nagyon csendben tud osonni.

A gepárd szeretett volna venni egy Ferrarit. De Olaszországban nem lehet
Ferrarit venni. Csak Angliában lehet. A macska a gepárd barátja. Ő Angliában
él. Ő is nagyon szereti az autókat, de ő nem csak gyors autókat gyűjt. A macska
még Ferrarit is tud venni. A gepárdnak is ő vett.

A hölgyi gepárd két völgy között lakik. Magyarul úgy hívják, hogy völgyi
gepárd. Angolul úgy, hogy hölgyi gepárd.


\section{A gepárd hátsó ülésén}
A gepárd vett múltkor egy űrhajót. Engem is elhívott kipróbálni. Együtt
repültünk vele. A gepárd ült elöl, és én ültem a hátsó ülésen. Nagyon gyorsan
mentünk, majdnem elestem.

Gofrit is csinált nekem a gepárd. A hátsó ülésen ettem a gofrit az űrhajóban. A
Holdra szállt le az űrhajóval, és még ott is ettem gofrit.

 
\section{A gepárd születése}
A gepárd akkor született, amikor dinoszauruszok éltek. Meg mamutok. De
ősemberek még nem.

A Lamborghinije már akkor is megvolt.

A gepárd örökké fog élni, mert az űrben lakik, a Plutón. Én is örökké akarok
élni.
\image{gepard.png}


\section{A gepárd halakat tanít}
A gepárd 18 éves. Egyetemista. Már úszni is tud. A gepárd nagyon jól tud úszni.
Ő tanította a halakat is úszni. Tanított kicsi halakat, nagyobb halakat, óriási
halakat… Cápákat is tanított.

Valamelyik halnak, ami nagyon kicsi és cuki, az a neve hogy Fiki. Fiki hal.
Hullám Fiki.

Amikor kimászott a vízből, a halak szomorúan néztek utána. Ők ott kellett, hogy
maradjanak.


\section{Sárkányok}
A gepárd drága autókat is gyűjt. Csak sárkányokat nem gyűjt. A sárkányokat csak
lefényképezi. Fényképezett már alvó sárkányt. Ülő sárkányt. Álló sárkányt.
Megyő sárkányt.

És ha valaki megkérdezi hogy milyen egy sárkány, megmutatja neki ezeket a
képeket. Odahozza őket dobozostul.

Én kérdeztem tőle:

\dash Miért dobozostul hoztad őket ide?

És a gepárd azt mondta:

\dash DOBOZOSTUL!

Újra megkérdeztem tőle:

\dash Miért dobozostul hoztad őket ide?

És a gepárd azt mondta:

\dash DOBOZOSTUL!

Újra megkérdeztem tőle:

\dash Miért dobozostul hoztad őket ide?

És a gepárd azt mondta:

\dash DOBOZOSTUL!

Újra megkérdeztem tőle:

\dash Miért dobozostul hoztad őket ide?

\dash Azért, mert be van lakatolva — mondta a gepárd.


\section{A szarvasok és a nyuszi}
A gepárdnak is van egy nyuszija. De nem fekete, hanem tarka-barka. A feje
fehér, valahol sötétszürke, valahol sötétkék, valahol sötétbarna.

Egyszer a szarvasokról gondoskodtam. Terelgettem őket az erdőben. A gepárd
pedig szimatolt, hogy észrevegye, ha jönnek farkasok vagy rókák.

Egyszer csak jöttek a farkasok. Én hajtottam a szarvasokat gyorsan-gyorsan. A
gepárd meg betette a nyuszit egy kosárba és felszívta a farkába. Ilyen
varázslatos farka van.

Felültem a gepárd hátára, és olyan gyorsan szaladt, hogy majdnem leestem!


\section{Csiribú csiribá}
A királyi család reggelihez készülődött.

\dash Mit ennél szívesen, édes lányom? — kérdezte a király a királylányt.

\dash Uborkát!

A király tett-vett és elkészült a finom reggeli uborkából.

\dash De ó jaj! — sopánkodott a király. — Sajnos nincsen tányérunk! Mi lesz most
velünk?

\dash Ne félj, királyom! Én tudok varázsolni. Csiribú csiribá — mutatott a
királylány a királyra. — Legyél tányér!


\section{Jelmezek}
Az én barátom, a gepárd köt nekem egy rókajelmezt.

A kis fehér bundám fonalból lesz. Szürke nyakam lesz. Az fonalból lesz, a
szürke nyak. Mert ők nagyon ügyesen tudnak kötni. Gyorsan kötnek, nagyon
gyorsan.

És fonalból lesznek a körmeim is. Nagy körmeim lesznek. Meg tudom vele enni a
húsos szendvicset. Így fogom bekapdosni a szendvicset: (nyam-nyam-nyam)

A macska meg királynőjelmezt köt Anyának. Köt hozzá koronát is. Ha tud. Mert
nem biztos, hogy tud.

A béka köt Olinak ördögjelmezt. Apának a kutyus köt Hókuszpók-jelmezt.


\section{Galyatetőn}
Meg kell várjuk a gepárdot. Még gallyakat gyűjt Galyatetőn. És még beszélnie
kell ott egy nővel. Nem tudom, mit kell beszélnie vele. Felhívom a gepárdot.

\dash Halló, gepárd? Miért kell beszélned még vele?

Azt mondja a gepárd, hogy a szállodában nincs víz. Nincs víz a medencében és
nem tudnak fürdeni a vendégek. A csapból se folyik víz. Nyállal kell fürdeniük!

Úgy tudnak csak mosdani, hogy nyalogatják magukat. Mosógép sincsen. A zoknit és
más holmikat is nyalogatniuk kell. Így mosnak fogat: (a nyelvükkel). Még az
orrukat is nyállal mossák meg.


\section{A sárga Ferrari}
Nagyon régen az én barátom, a gepárd akart venni egy Ferrarit. Ez nagyon régen
volt, amikor a gepárd született. Akart venni egy Ferrarit, de sehol sem kapott.
Nem volt sehol Ferrari. Aztán tudod, hol talált egyet? A macskánál talált egy
Ferrarit, Angliában.

Ez a Ferrari sárga volt. Csupa sárga volt mindenhol. Kívül is sárga volt, a
kereke is sárga volt, a jele is sárga volt, még a kormány is sárga volt.

Mert mézet öntöttek rá. Szürke volt, de ráöntötték a mézet és csupa sárga lett.

Nagyon gyors volt. Nagyon gyors autó volt. Ez volt a leggyors autóbb. Mint egy
versenyautó. Még a gepárdnál is gyorsabb volt!


\section{A szúnyog és a zsiráf}
Tili-tili, csörög a telefon.

\dash Halló, rendőrség?

\dash Mi történt?

\dash Azt mondja, elrabolták a zsiráfot az állatkertből!

\dash Nahát, ki tehetett ilyet?

\dash A szúnyog volt! Menjünk, szabadítsuk ki!

\dash Állj meg szúnyog! Engedd el a zsiráfot!

\dash De nagyon éhes vagyok. Mit fogok enni, ha elengedem a zsiráfot?

\dash Tessék, itt ez a vér. Ez nagyon sok vér. Nyugodtan szívj belőle, nem fog
elfogyni. Csak nagyon sokára fog elfogyni. Amikor ez elfogy, akkor már nem is
fogsz sokat élni. Már csak két napot fogsz élni, amikor ez elfogy.

\bigskip

Másnap.

\dash Halló, rendőrség?

\dash Mi történt?

\dash Azt mondja, elrabolták a szúnyogot. Most a zsiráf rabolta el a szúnyogot!


\section{Ha nagy leszek}
Amikor felnőtt leszek és lesz nekem lányom, majd úgy fogják hívni, ahogy te
szereted, Anya. Zita. És ha te akkor már a Mennyországban leszel, akkor majd
elmesélem neked, amikor én is ott leszek.


\section{A hosszúlábú egér}
A gepárdnak vannak ilyen kis üregei. Be vannak ásva a falhoz. És ebben laknak a
kis egérkék. Minden nap tesz bele egy kis ennivalót. Megeszi az egér és aztán
kijön. Van neki ajtaja. Nem veszélyes bent lenni, mert van ajtaja az
egérlyuknak.

Amikor kijönnek, át tudnak menni a szomszéd egérlyukba beszélgetni a másik
egérkével.

A gepárd ért az egerekhez. Mondta nekem, hogy vizet isznak. Sajtot és
gyümölcsöket esznek.

És még hangyát is van, hogy megesznek. Igazából csak a hosszúlábú egér eszi meg
a hangyákat. De ő még a cicát is megeszi! Világítanak a szemei. Este vadászik.
Ráadásul az utcára megy ki vadászni. Én elkísérem a kapuig, ott elköszönünk, és
én visszajövök a házba. Ő meg elmegy vadászni.


\section{Az iskolabuszállomás}
Mi is jártunk ott a gepárddal. És amikor ott voltunk, a gepárd mutatott nekem
valami nagyon szépet! Olyan szép volt! Képzeld, mit mutatott…

Egy iskolabuszállomást mutatott. Nagyon szép öreg iskolabuszok voltak ott. Egy
kiállítás volt.

És tudod, mi volt az állomás mellett? Egy gyönyörű szép szakadék. Folyt le a
víz és úsztak benne a halak. A vízben csobogtak lefelé. Be is volt fagyva az
egész, de belül azért folyott és a halak tudtak csobogni.


\section{Tavasz}
Mama mondta, hogy visszaköltözött az első gólya. Benne volt a hírekben. A
gepárd látta ezt a gólyát. Távcsővel nézte, és le is fényképezte.

Egy kismadár is hazaköltözött. A gepárd mesélte. Most már ketten is vannak, egy
gólya és egy kismadár.

Egy béka is hazaköltözött. Ez egy költöző béka. Mindig oda költözik, ahol meleg
van, mert nem akar megfagyni. Ez a béka piros. Nem eheti meg a gólya, mert
mérgező. Ha megenné, nem tudná mozgatni a szárnyait. Megbénulnának a szárnyai.

A gepárd szokott madarakat fényképezni, és békákat.


\section{Farsang}
Ismerek egy dínót, aki nagyon szereti a gyerekeket. Annyira szereti a
gyerekeket, hogy farsangra gyereknek öltözött.

Egyszer csengettek. Bimm-bamm. Aki hallotta a csengőt, odament és kinyitotta az
ajtót. Odakint egy gyerek állt. (Igazából a dínó.)


\section{Qubi}
Nem gyártanak már Saabot. Vagyis egy darabig nem gyártottak Saabot. De a gepárd
elkezdett Saabot gyártani. Aki akar Saabot venni, a gepárdtól vehet. Én szoktam
tőle venni.

Van egy Saab verseny. Ezen csak Saabok versenyezhetnek. És olyan autók, amiknek
ugyanaz a jele, mint a Saabnak. Például a Qubi. A Qubi nagyon-nagyon lapos és
át tud ugrani egy tüzes karikán. De a hangja, az nagyon félelmetes!

A jele a Saabra hasonlít. Az alakja inkább a Smartra hasonlít. De a hangja… A
hangja nem hasonlít semmilyen autóra. A hangja olyan, mint a zimbinek.

A zimbi egy különleges állat. A szeme és a szája oldalt van a fején. A lábai
pedig a feje tetején vannak. És a szeme nagyon-nagyon sötét.


\section{A kutató}
Van egy tatu, ami Apa szerint ankiloszaurusz. Ezt majd a gepárdtól kell
megkérdezni. A gepárd jobban tudja. Mert a gepárd kutató. Szemüvege is van és
egy monitor előtt ül. Olyan számítógépe van, aminek nem lehet lecsukni a
tetejét. Két részből áll. Ott mindent meg tud nézni.

A gepárd mindenhez ért. Ő meg fogja tudni mondani, hogy tatu vagy
ankiloszaurusz.


\section{Egykerekű}
A gepárdnak is van egykerekű biciklije. De az olyan, hogy még ülése sincs. Csak
egy kerék és a pedál.

Kettő egykerekűje van. Az egyik az, aminek nincs ülése. A másiknak van egy kis
tartója. Abban tud tartani ezt-azt. Például nyalókát. És ennek még pedálja
sincs!

Ha meglökik, elgurul. Figyelni kell, hogy alatta maradjon a kerék.


\section{A katica mentőakció}
A gepárd odaslisszolt a kisvakondhoz egy Mercedes rendőrautóval, hogy megmentse
a katicát a rókától. Betette a hátizsákjába. A gepárdnak csak menő autói vannak
és kabriói. Lecsukta a tetőt, és leeresztette a redőnyt is. Mert ez az autó át
tud alakulni házzá.

Ott aludt a gepárdnál a katicabogár. Odajött a róka, és bekukucskált. Volt egy
kis rés, ahol nem volt redőny és a tető sem ért el odáig. Kiesett itt a csempe.

Leselkedett a róka, és be akart jönni. De nem tudott bejönni. Zárva volt az
ajtó és az ablak. És a róka nem találta a slusszkulcsot. Mert a slusszkulcs
bent volt, a gepárdnál.

A rókának van egy kése. Nem kell, hogy ő vágjon vele. Csak megnyom egy gombot,
elengedi, és vág magától. Össze van kötve egy távirányítóval. A róka megy utána
és a kés vág. Van kereke. Ezzel akarta elvágni az autót. De ez a kés csak füvet
tud vágni!

\bigskip

A gepárd beült a rendőrautóba. Morcos volt a rókára, mert nem szereti a rókát.
A róka sem szereti a gepárdot. A gepárd becsukta a tetőt. Nem esett az eső, de
mégis becsukta.

Betett a katicának a hátizsákba ételt. Cukorkát, édes dolgokat, de sósat is.
Halat. A cukorkát ki kellett, hogy bontsa neki a gepárd, mert ő nem tud bontani
annyira ügyesen.

Egyszer csak kopogtatnak. Te is kitalálod, ki kopog? Három halacska volt.
Gyagya, Kutyima, és Nemo. Fáztak, éhesek voltak. Amikor a három halacska
kopogott, már este volt. Hideg volt. A katica is félt. Nem akart kimenni.

Azt mondta a gepárd:

\dash Ha majd reggel lesz, egy picit még itt leszek. Mert meg akarom nézni,
milyen tortátok lesz.

Mert a három halnak egyszerre lesz a szülinapja. Nem ikrek, de egy napon van a
szülinapjuk.  Testvérek.

És vége a történetnek, úgy látom. Fuss el véle!


\section{A harcsa látogatása}
A gepárd kiesett az ágyból a zaj miatt. Mert a harcsa hangosan kopogott. Mert
megharapta a sneci. Pedig a sneci nem is eszik halat. Akkor miért harapta meg?

Mert ez a sneci egy ragadozó sneci volt! Megharapta a harcsa uszonyát.

A gepárd kinyitotta az ajtót, de a harcsa nem fért be rajta. Túl nagy volt. Ez
a gepárdnak való ajtó volt. A gepárd kinyitotta a másik ajtót, amin befért a
harcsa. Itt egy nagyon tiszta előszobába érkezett. De nem ott volt a
fürdőszoba, hanem a konyha mellett balra.

Ez 22 éve történt.


\section{A Nílus}
A gepárd megsajnálta a halakat és bedobált nekik ennivalót.

Eleredt az eső. Nem csepergett, hanem ömlött! A halakat elsodorta. Nem tudtak
jól úszni a folyóban. De a cápa tudott. A halak után ment, és tolta őket. És
amikor nem folyt olyan erősen a víz, akkor a halak tudtak úszni, és a cápa
mellettük úszott. Mert a cápa a barátjuk volt.

A gepárd szereti a halakat. A gepárd szereti a vizet. Hagyta folyni a folyót,
mert a folyókat is szereti. Felment egy magas toronyba, és onnan nézte.

A folyó folyt mindenfelé. Összevissza folyt. Felfolyt egy útra. Ráfolyt egy
bukkanóra. Onnan ráugrott egy ház tetejére. Befolyt a kéményen, és a plafonon
folyt.

Szerencsére senki nem volt a házban. Éppen nem voltak otthon. Nem is látták a
vizet. Messziről nézték a házukat, és csak azt látták, hogy valami van a tetőn.
De nem tudták, hogy az a folyó.


\section{A macska nyaralója}
A gepárd házában nem a kenyéren van a lekvár. Minden lekvárból van. A kenyér is
lekvár, a tányér is lekvár, az asztal is lekvár, a falak is lekvárból vannak.
De nem igaziból: a gepárd csak erre gondol.

A macska nyaralója, ami a Balatonon van, az tényleg lekvárból van. Le fogják
rombolni. De én megnéztem. Le is fényképeztem. Aztán egy kicsit elmentem,
visszafordultam, és már azt láttam ahogy épp lerombolták.

Én elkapkodtam a hamikat. Lekvár-hamikat!


\section{Macskababa születik}
A gepárd most nem tud velünk jönni.

A gepárd barátja, a macska most fog szülni. Már van egy babája, egy lány. Most
bement a kórházba. Meg fog születni a kisbabája.

A gepárd vigyáz addig a macska lányára. Vigyáznia kell, hogy a baba ne egye meg
a szobát. Mert a falak lekvárból vannak. A plafon mézeskalácsból. Az ablakok is
lekvárból.


\section{Nyár télen}
Télen volt egy nap, amikor nem esett az eső, nem esett a hó. Semmi sem esett,
és olyan meleg volt, mint nyáron. A gepárd egy szál bugyiban szaladgált.
Felfújt egy medencét és abban úszkált. Volt, hogy még bugyi sem volt rajta.
Pucéran szaladgált.


\section{A dinoszauruszok menekülése}
Ezer évvel ezelőtt a gepárd olvasta az újságban, hogy esni fog az eső. Nagyon
nagy eső lesz. Húsz napig fog esni. A gepárd szereti a dinoszauruszokat. Meg
akarta menteni őket.

Beugrott a vízbe, és azt kiabálta a dinoszauruszoknak:

\dash Hé, itt vagyok, kapjatok el, ha tudtok!

Aztán futott nagyon gyorsan, és a dinoszauruszok futottak utána.  Elfutott a
Plutóra és ott egy ketrecbe tette a dinoszauruszokat. Ha valamelyik nem fért
már be a ketrecbe, akkor egy másik ketrecbe tette. Olyan sok dinoszaurusz volt,
hogy húsz ketrecbe kellett tennie őket.


\section{Krémtúrótorta}
Egyszer kaptam egy krémtúrótortát. Megkérdeztem a gepárdot, hogy kér-e belőle.
Azt mondta, á, nem, ő nem szereti a krémtúrótortát.

De utána tudod, mi történt? A gepárd megette az egész tortát! Megette előlem
mindet!


\section{Hurka recept}
A gepárd finomabb hurkát tud csinálni. Összerántja, a tojásnak a levét beteszi
egy pohárba, megissza, és utána a brokkoliban kifőzi a hurkát.

A tigrisről is van mese. A gepárdnak a barátja a tigris. De már meghalt. Nagyon
öreg volt. Nagyon híres is volt. Arról volt híres, hogy fát eszik. Erős volt a
foga.


\begin{KeepFromToc}
\makeatletter
\renewcommand*{\@tocmaketitle}{}
\makeatother
\renewcommand{\beforesection}{\cleartorecto}
\section*{Tartalom}
\tableofcontents
\end{KeepFromToc}

\end{document}
