\documentclass[10pt]{memoir}
\usepackage{pifont}
\usepackage{charter}
\usepackage[utf8]{inputenc}
\usepackage{t1enc}
\usepackage[hungarian]{babel}
\usepackage[final]{microtype}
\usepackage{hyperref}
\clubpenalty=10000
\widowpenalty=10000
\raggedbottom
\setstocksize{6.87in}{4.25in}
\settrimmedsize{6.87in}{4.25in}{*}
\isopage[12]
\checkandfixthelayout
\setlength{\emergencystretch}{2em}

\renewcommand{\pfbreakdisplay}{\bigskip \ding{70} \bigskip}
\newcommand{\secbreak}{\fancybreak{\pfbreakdisplay}}

\author{Darabos Dániel}

\date{2014}

\title{A csillagok mélyén}

\begin{document}

\begin{titlingpage}
  \centering
  \vspace*{0.2\textheight}
  {\Huge \thetitle}\\[\baselineskip]
  {\large\itshape \theauthor, \thedate}
\end{titlingpage}

\vspace*{0.4\textheight}

\itshape
\noindent
A Stanton rendszer legbelső óriásbolygója sivatagos volt a terraformálás előtt.
Évmilliárdokon át csak homokviharok formálták a rozsdavörös sziklákat. Nem
vájtak medret maguknak folyók, és nem szántották fel a vidéket gleccserek. A
Birodalmi Terraformáló Művek űrhajói százhárom üstökös pályáját változtatták
meg, hogy a bolygóba csapódjanak. Mikor az üstökösök jege aláhullott, a robogó
árvizek riadt ménesként kerestek utat maguknak. A mély medencéket három rövid
hét alatt tengerek töltötték meg, és a hömpölygő áradat lassan elcsendesedett.
A bolygó nem szomjazott többé.

De a legnagyobb üstökös túl laposan érkezett és elégett a légkörben. A
tízmillió köbkilométer víz vastag felhőburokba vonta a bolygót. Azóta nem érte
napfény a felszínt. Havazni kezdett, és az ifjú tengerek befagytak.

A Stanton rendszer négy hatalmas bolygóját elárverezték a terraformálás után.
Négy csillagközi óriásvállalat birtokába kerültek. A hibás terraformálás miatt
a legbelső bolygó kedvezményes áron kelt el. A MicroTech digitális technológiai
cég vette meg. A MicroTech nem a nyersanyagok miatt vette meg a bolygót. A
hóviharok miatt nem volt gazdaságos kitermelni őket. Nem is a földterület
érdekelte. Csak egy várost hoztak létre a bolygó egyenlítőjén. A MicroTechet a
másik három óriásvállalat közelsége vonzotta a rendszerbe.

A MicroTech látta el digitális technológiai eszközökkel a másik három bolygót
elfoglaló cégeket. A Hurston Dynamics fegyvergyártó konszern egy bolygóméretű
ipartelepet hozott létre a második bolygón. A fegyvergyártókat az vonzotta a
rendszerbe, hogy itt nem kellett környezetvédőkkel és munkajoggal bajlódniuk.
Felhők és havazás helyett a Hurston időjárását szmog és savas eső jellemezte. A
harmadik bolygón az ArcCorp rendezte be gyárait. A hihetetlen méretű épületek
bevonták a bolygó teljes felszínét, és a felszín alatt is beláthatatlan
csarnokokat vájtak ki. A fúziós meghajtók építése kiemelkedően sok teret,
szaktudást és digitális technikát igényelt. A negyedik bolygón a Crusader
Industries teherhajói készültek. A szinte gázóriás méretű bolygó egyedi
adottságai lehetővé tették, hogy a hatalmas űrhajókat gazdaságosan, a légkörben
lebegő platformvárosokban szereljék össze.

A MicroTech tehát gyártósorokat programozott és ipari robotokat fejlesztett.
Számítási kapacitást biztosított lézerfegyverek és fúziós meghajtók
modellezéséhez. A Crusader teherhajók fedélzeti számítógépeit is ők
szállították. A másik három bolygón még épültek a kontinensnyi gyárak, mikor a
MicroTech befektetése már megtérült.

A MicroTech egy szép kisvárost épített alkalmazottainak a huszonhetedik század
jellegzetes stílusában. A város fűtött utcái egy színes kővel burkolt téren
futottak össze. A mozaik egy klasszikus áramkört ábrázolt, amit pontosan félbe
vágott a bolygó egyenlítője. A város déli oldalán, a déli féltekén laktak
jellemzően a fiatal alkalmazottak, és itt nyíltak egymás után a
szórakozóhelyek. Az északi félteke kertesházaiba a cég felsővezetői, idősebb
alkalmazottai és családaik költöztek. Az első építkezési hullám után is
megtartotta a város a két félteke kettősségét.

Mikor befejeződött a csipgyárak és adatközpontok építése, a beáramló új
munkaerő inkább a déli oldal felhőkarcolóiban telepedett le. A MicroTech
platformjára építő kisebb cégek is szívesen költöztek a bolygóra. Tehetős
vezetőik az északi oldal lejtőin építettek maguknak házakat. Ezek ritkán
követték a huszonhetedik századi stílust. A kis völgyekkel szabdalt táj
dombtetőin megjelentek a legmodernebb görbék, de az emberiség korai
történetének stílusai is visszaköszöntek.

Az egyik dombon egy középkori észak-európai fatemplom másolata állt. A bolygóra
telepített fenyők fájából építették az erőteljes, kúpos tornyokat. A templom
szárnyait megnagyobbították és lakosztályokat rendeztek be bennük. A központi
csarnok tágas családi étkezőként szolgált és vacsora után sokszor mesék
töltötték meg.

— Látjátok ezt a csövet? — lóbálta Howard az alkatrészt unokái feje felett. —
Rozsdamentes acél. Ez az egy alkatrész készül még mindig rozsdamentes acélból a
mai űrhajókon. Meg tudja valamelyikőtök mondani, miért? A grafén olcsóbb,
erősebb és könnyebb. Miért csinálják mégis acélból?

A ráncos öreg körül csak feszengett a tucatnyi unoka. A kicsik a nagyokra
néztek, a nagyok a vállukat vonogatták. Kint elcsendesedett a hóvihar és a
kandalló tüzét is alacsonyabb fokozatra kapcsolta a vezérlőrendszer. Csak a
székek nyikorgása hallatszott a faragott gerendák alatt.

Howard két legidősebb fia közösen alapított egy céget. Az idős pilóta nem hitt
a vállalkozásban, hiszen még mindig csak veszteséget termelt. Egy befektetési
bank viszont hitt benne, és kerek százmillió kreditért megvásárolta tőlük a cég
49\%-át. Howardot még mindig bosszantotta, hogy a fiai egy ingyenes MobiGlas
programot fabrikálnak ``rendes munka'' helyett. De örömmel költözött be a
fatemplom egyik szárnyába.

— Mi az a cső? — a legkisebb unoka tette fel a kérdést, amire mind reménykedve
várták a választ. Ha tudják, mire való az alkatrész, talán megsejtik, miért van
acélból.

— Hát ezt se tudjátok? Mindent tanítanak nektek az iskolában, csak azt nem,
amit kéne! Erről is a MicroTech tehet. Nem tudják, mire van szükség az életben.
Azt hiszik, minden csak az elektronikáról szól. Az orrukig látnak. Nem tovább.
És még ha hosszú orruk lenne! De nem, kicsi malacorruk van.

A gyerekek dőltek a nevetéstől, mikor Howard malacorrot formált. Hozzászoktak
már nagyapjuk kifakadásaihoz az oktatásról, a MicroTechről és a nagybetűs
életről. Gyakran terelték ezek mellékvágányra az öreg meséit. De most a kezében
tartott rozsdamentes acél cső súlya lehorgonyozta gondolatait, és mély
lélegzetvétellel belekezdett felejthetetlen történetébe.

\secbreak
\upshape

Annyi idős lehettem, mint ti. 2872-t írtunk. Ebben az évben költöztünk a
MicroTechre. Édesanyám, a ti dédmamátok, itt talált munkát. Akkor épültek a
hegyen túl a régi csipgyárak, és sok robotkezelőre volt szükség. Egy évre rá
apám is talált állást. Az első év szűkös volt, de mi nem vettük észre. Minden
sokkal jobb volt, mint a Hurstonön. Tudtunk saját lakást bérelni. Ha elült a
hóvihar, lehetett kint játszani. Először láttam állatokat. És először mentem
iskolába.

Emlékszem, akkor kezdték építeni a Szintit. Egy teljesen szintetikus világ.
Élet az élettelen kőből. A Birodalom büszkesége. Húsz év és elkészül, azt
mondták. Hát nem készült, és már nem is mondják. Nagyon súlyos bűnt kéne
elkövessek, hogy sztázisba zárjanak, míg elkészül. És ezt ma mindenki tudja is.
De akkoriban máshogy gondolkodtak az emberek. Mindenki bizakodott. Mindenki
álmodozott.

Az iskolatársaim mind a legújabb űrhajókatalógusokat bújták. Akkor jelentek meg
az első Constellation modellek, és mind róluk álmodoztak. Mindenki, kivéve
engem. Én hallgattam nagyapám meséire.

Ő is pilóta volt, és arra tanított, hogy csak használt hajókat vegyek. Zsugori
is volt Hubert papa, de nem ezért haragudott az új űrhajókra.

— Elment az életkedved? — kérdezte mindig. — Te akarsz lenni az első ember, aki
kirepül vele? Mi van, ha az egyik munkás csippje épp csörgött, amikor meg
kellett volna húzza a légzsilip egyik csavarját?

Persze a hajókat robotok építik, de a papa bennük sem bízott jobban. Valaki meg
kell, hogy húzza a robot csavarjait is.

Így hát én használt alkatrészeket vásároltam a menzára kapott pénzemből. Enni
az új szomszédainkhoz jártam. Egy idős raëlista pár lakott ott. Ez egyike a
legrégebbi fennmaradt hiteknek. Az űrkorszak előtt született, és a bölcs
űrlények eljövetelét várta. Hiába jöttek el aztán a Banuk, a Xi’Anok, a
Tevarinok és a Vanduulok, egyik faj sem bizonyult elég bölcsnek a raëlisták
számára. Tovább vártak.

A szomszéd pár családja egy távoli bolygón élt, így nem tudtak saját unokáiknak
sütni és főzni. Én szívesen hallgattam a meséiket egy tál főzelék felett. Egy
szép kis éttermet vezettek, mielőtt a Stanton rendszerbe költöztek. Mindenük
megvolt, és nagyon szerettem hallgatni a történeteket a kényelmes, meleg
életükről. Azt mondták, azért hagytak hátra mindent egy nap, mert meghallották
Stanton hívó szavát. Meséltek a vallásukról is, a mindenható idegen lényekről,
akik Stanton óriásbolygóit teremtették, és akik szeretettel fogadnak majd
minket egy nap. Ez nem fogott meg igazán, de annak örültem, hogy én egy ilyen
fontos csillagrendszerben születtem.

Néha a leckét is kikérdezték tőlem, és azt is tudták, hogy alkatrészekről
álmodozom. Egy nap kaptam tőlük egy szenzorprogramot. Nagyon büszkék voltak rá,
mert a velem egyidős unokájuk írta a kódot. Azt mondták, olyan érzékeny, hogy
még azt is érzékeli, ami nincs is ott.

Nem akarom bántani a kislányt, de a program katasztrófális volt. Még nem is
csatlakoztattam szenzorokhoz, de már ellenséges támadást jelzett minden
irányból. Annyiban szerencsém volt vele, hogy fedélzeti számítógépem már volt —
ezt a legkönnyebb a MicroTechen beszerezni, — de szenzorom egy szál se. A hibás
programnak köszönhetően mégis átélhettem első űrcsatáimat.

A lakásunkat egy diáknak vagy gyakornoknak tervezték. Egy háló, egy fürdő és
egy tágas gardrób. A tágas gardrób az én szűkös gyerekszobám lett. Itt
építetgettem a fedélzeti rendszert. Egy villogó riasztólámpával kezdődött, de
mikor a szenzorprogramot kaptam, már az egész falat betöltötték a szedettvedett
képernyők.

Áthívtam Jam barátomat, és késő estig játszottunk a fedélzeti számítógéppel.
Lelkesen vezéreltük a nem létező ágyúinkat, és szitává lőttük a szenzorprogram
által jelzett nem létező ellenségeket. Jam csendes fiú volt, de vele jöttem ki
legjobban az iskolában. Ő sosem hencegett azzal, hogy az apukája milyen űrhajót
fog neki venni, ha megnő. Pedig az ő apja valódi pilóta volt.

Mikor beiratkoztam az iskolába, bemutatkoztak az osztálytársaim. Név, életkor.
Jam Joys, 114 éves. Kuncogás. Csak sokkal később, mikor az első működőképes
szenzort csatlakoztattuk a fedélzeti számítógéphez, mertem megkérdezni, hogy
lehet 114 éves. Elmondta, hogy ő hogyan került a MicroTechre.

Jam két éves volt, mikor szülei a Stanton rendszerbe érkeztek vele. A házaspár
a Terraformálási Művek legsikeresebb csapatát vezette. Az édesapja, Owen volt a
pilóta, az édesanyja, Meria a biológus. Volt velük még egy meteorológus és egy
navigátor, két régi barátjuk. Stanton óriási bolygóinak a terraformálását a
Művek négy legjobb brigádjára bízták. Jam szülei és barátaik kapták a legbelső
bolygót. Mikor épp a legnagyobb üstökös pályáját módosították, hogy vízzel
lássák el a sivatagos bolygót, balesetet szenvedtek. A hajó felrobbant még
mielőtt bárki be tudott volna szállni a mentőkapszulába.

Jam apja két hétig egy szál űrruhában sodródott a bolygóközi űrben. Alig élt
már, mikor megtalálták. Hónapokig kezelték. Utána kezdődött a per. Mint pilóta,
ő felelt a balesetért, felesége és két barátjuk életéért, és a bolygó hibás
terraformálásáért. Száz év sztázisra ítélték. Jam négy éves volt ekkor és nem
volt se rokonuk, se barátjuk Stantonban. A bíróság úgy határozott, hogy Jamnek
legjobb, ha édesapjával együtt sztázisban várja meg a büntetés leteltét.

Száz évvel később kiszabadultak és a MicroTechen maradtak. Owen a tíz eltelt
évben nem talált állást. Ki alkalmazná azt a pilótát, aki fél az űrtől? Ki
alkalmazná azt az embert, akinek egy bolygó lakossága átkozza a nevét ahányszor
hóvihar fagyasztja be az űrkikötőt? Úgy tett hát, ahogy bármelyikünk tett volna
a helyében. Inni kezdett.

\secbreak

Jam igyekezett minél kevesebb időt tölteni otthon. Mindig nálunk játszottunk,
sosem náluk. Ha talált egy űrhajóalkatrészt, az én hajómhoz hozta el beépíteni.
A hajó az övé is volt annyira, mint az enyém. Mikor már apám is dolgozott,
kaptam zsebpénzt. Még mindig a raëlista szomszédokhoz jártam enni, mert jól
főztek és szüleim későn értek haza munkából. De már segítettem nekik a
bevásárlással. Jam is sokszor velünk evett. De maradt annyi pénz, hogy nagyobb
alkatrészekre is félre tudjak tenni.

Megvettem egy halott műholdat. A terraformálás előtti felmérésekhez használták,
és azóta kihasználatlanul keringett. A megvett vezérlőkóddal tudtam neki
parancsokat küldeni, amikor a város felett repült. Készítettem pár képet a
műhold nagyfelbontású kamerájával, de a sűrű felhőrétegen át nem látszott semmi
a városból. Más hasznos műszer nem volt rajta. Érthető, hogy nem kértek sokat a
vezérlőkódért.

Csináltunk egy olyan vezérlőprogramot, ami lehozta a műholdat a felszínre, a
város közelében. Ezek a régi műholdak nem voltak igazán manőverezhetők, de
annyit el tudtunk érni, hogy ne égjen el a légkörbe lépés során. Az északi
félteke egyik puccos birtokán csapódott be. Éjszaka, hóviharban mentünk el
érte, hogy a biztonsági rendszerek ne csípjenek el. Jam és én két nagy
hólapáttal lapátoltuk a havat a fejünkre rögzített lámpák gyenge fényénél.
Végre kiástuk a roncsot és leszánkáztunk rajta a birtok határáig. Mire
eljutottunk a fűtött útig, mindketten majd megfagytunk. Egy hétig nem éreztem a
lábujjaim. Őrültek voltunk.

A műhold szerkezete lett kis átalakítással az én űrhajóm alapja. Találtunk egy
nagyon olcsó kis dokkot a kikötőben, és ott kezdtük el összeszerelni. A fő
különbség egy műhold és egy űrhajó között, hogy a műhold nem lakható. Három
réteg légzáró fólia beragasztása után már az lett. A mikrometeorok elleni
energiapajzs elég erős volt, hogy az utasteret is megvédje a kozmikus
sugárzástól.

A dokkőr egy idős úriember volt, mint én most. Túl a nyolcvanon. Szigorúnak
éreztük, mert minden előírást betartatott velünk, és kikapcsolta a dokk
fűtését, ha nem fogadtunk szót. De visszagondolva látom, hogy nagyon sokat
köszönhetek neki. Egyrészt a szabályok a mi biztonságunkat szolgálták. Másrészt
hivatalosan ki kellett volna dobnia minket a kikötőből az első mulasztás után.

Az egyik előírás, ami gondot okozott, az a szabványnak megfelelő légzsilip
követelménye volt. Volt persze légzsilipünk, de a műhold egy felnyitható
paneljéből alakítottuk ki. Mi be tudtunk mászni a kis nyíláson, de messze volt
a szabvány két méterszer két méteres méretezésétől. A dokkőr hiába magyarázta,
hogy erre a méretre szükség van, ha egy mentőhajónak kell dokkolnia velünk.
Csak akkor vettük komolyan, mikor kikapcsolta a fűtést, és az életfenntartó
rendszer feladta a harcot a MicroTech időjárásával.

Néhány heti munkával és sok kölcsönkapott szerszámmal ki tudtuk tágítani a
nyílást. Az űrhajó egész baloldalát betöltötte a hatalmas kapu. De még mindig
nem felelt meg az előírásnak. A biztonságos dokkoláshoz a két zsilip össze
kell, hogy kapcsolódjon. Ezt mindkét zsilipen hat mágnes és egy rozsdamentes
acél cső teszi lehetővé. Hát erre való a cső, amit a fejetek felett lóbálok!
Mágneses illesztőkarnak hívják.

Jam különösen viselkedett, mikor előjött az illesztőkar kérdése. A dokkőr és én
hosszasan latolgattuk, hogy hol szerezhetnénk egyet. Én tele voltam ötletekkel,
az öreg pedig sorban lelőtte őket. Az acél lágy az űrhajós grafénhez képest, és
a kar a zsilip külső szegélyén van. Az olcsón megszerezhető roncsokban biztos,
hogy az illesztőkar pereccé görbült vagy elolvadt. Túltermelés sem volt belőle,
mert másra nemigen használhatóak a csövek. Egy régebbi kor maradványa ez a
zsilipcsatlakoztató rendszer, de nem lehet lecserélni. Az új rendszer nem tudna
a régihez kapcsolódni.

Jam egy szót se szólt. Hazafelé menet faggatni kezdtem.

— Mi van veled, Jam? Befagyott a szád?

— Nem fogja odaadni — mondta. — Nem fogja odaadni az illesztőkart. Az az
egyetlen dolog, amit apám jobban szeret a viszkinél.

Mint kiderült, Owennek volt egy mágneses illesztőkarja és mindig a keze ügyében
tartotta. Jam biztos volt benne, hogy egy üveg viszkivel nem tudjuk
megvásárolni. De bíztam benne, hogy két üveg elég lesz.

Két üveg viszkit sem sokkal könnyebb tizenhat évesen beszerezni, mint egy
illesztőkart. Nem is találtunk végül legális, morális vagy biztonságos
megoldást. De ott álltunk Jam lakásának ajtajában, hátizsákomban két palack.

— Ez a barátom, Howie — mutatott be Jam. — Két üveg viszkit adna a botodért.

Owen magas, kövér férfi volt. Óriásnak tűnt, ahogy előbukkant a kis lakás
ajtajából. Se cipő, se ing nem volt rajta, csak egy mosásra szoruló nadrág és
egy elnyűtt vászonsapka. A sapka alól zsíros, őszülő tincsek lógtak ki. Az
előszoba megsárgult műanyag falának támaszkodott és hunyorogva végigmért.

— Ezért a botért? — kérdezte vontatottan, és háta mögül előhúzta a másfél
méteres acélcsövet.

Az elején úgy éreztem, sikerülhet az üzlet. Megnézte, milyen viszkit kínáltam.
Kezével méregette a cső súlyát. De minden elromlott, mikor elmondtam, hogy az
űrhajómhoz kell az illesztőkar.

— Mit képzelsz, mit tudsz az űrről, te gyerek? Ez nem egy nyavalyás
játékprogram! Mit akarsz te odakint? Csak ránk hozod a bajt! A husáng kéne?
Gyere ide, adok én neked husángot!

Mire az utolsó mondatokhoz ért, én már futásnak eredtem. Owen az ajtóban rázta
a csövet, mikor a folyosó végén a lépcsőházhoz értem. Visszapillantva láttam,
hogy Jam is utánam indult volna, de az apja elkapta a grabancát. Úgy éreztem,
vissza kell menjek segíteni, de a lábaim nem mozdultak. A beázott lépcsőház
első lépcsőfokáról hallgattam, ahogy Owen ordított a fiával.

— Hát neked anyád emléke semmit nem jelent? Belőlem is csúfot akarsz űzni? Be
ne merj szállni egy űrhajóba, megértetted? Ha megtalálnak, téged is apró
darabokra szednek! De látom a szép szóból nem értesz!

Egy darabig hallgattam az ordítozást és az üres üvegek csörömpölését, de aztán
haza indultam. Sírva. Nem voltam egy ijedős srác, de Owen erőszakossága
megrázott. A MicroTech egy különösen biztonságos bolygó. Jobb, ha tudjátok,
hogy ez nem mindenhol van így. Itt a cég megteheti, hogy kitoloncolja az
erőszakos alakokat. A Hurston mindig szívesen fogadja a munkaerőt. De Owennek
MicroTech volt a büntetés. A bolygó, amit elrontott. A bolygó, ami elrontotta
őt.

Újra és újra lejátszódott a fejemben a jelenet az úton hazafelé, majd az ágyban
forgolódva.

— Ha megtalálnak, téged is apró darabokra szednek!

Jól értettem? Kikről beszélhetett?

\secbreak

Jam pár napig nem jött iskolába és az üzeneteimre sem válaszolt. Aztán
előkerült, de nem jött át többet játszani és nem járt ki velem a dokkba. Nem
beszéltük meg soha, de azt hiszem, tényleg megbánta hogy megszegte Owen
tilalmát. Azt éreztem rajta, hogy engem is nagyon féltett az űrhajózástól. Az
édesanyja halála után érthető is volt, persze, de én mégis cserben hagyva
éreztem magam.

Tovább dolgoztam a hajón, de minden nehezebb volt egyedül. Illesztőkart sehol
nem kaptam. A zsilipkapuhoz előírt hat mágnest drága pénzért megvettem egy
állítólagos pilótától. Az egyik már nem fogott egyáltalán és a többi is
lehetett volna erősebb. A döglött mágnest becsempésztem a maglev vasút alá,
hogy a mozdony újramágnesezze. De ügyetlenül rögzítettem, és a szerelvény
elvitte magával.

Szüleim megtalálták a két üveg viszkit, amit Owennek ajánlottam. Egy hétig nem
engedtek sehova, és csak az alkohol veszélyeiről beszéltek. Hogy hány ezer éve
nem tud megszabadulni tőle az emberiség. Van, ami sosem változik.

— Gyere át hozzánk. Siess!

Ennyit írt Jam. Ezt az üzenetet kaptam tőle egy este. A legkevésbé sem akartam
Owennel találkozni, és Jamre is haragudtam. Tudtam, hogy bajban van. De ha ő
nem segített nekem, én miért segítsek neki? Kicsinyes az ember, amikor fél. Tíz
perc duzzogás után mégis útnak eredtem.

Mikor Jam felvitt a lakásukhoz, a lépcsőn másztunk fel. Büdös volt a lépcsőház
és hosszú az út a huszonharmadik emeletre. A világítással is baj volt egy
hosszú szakaszon. Mai napig nem értem, hogy lehetnek ilyen épületek a bolygón.
Állítólag nem is megrendelésre készültek, csak az építőmunkások húzták fel őket
a többi házból megmaradt anyagokból. Ha így van, megmaradhatott volna több
takarítórobot is.

Inkább a liftet választottam. Élt benne egy őrült Banu. — Aloha! — köszöntött.
Egy virágfüzért akasztott a nyakamba, és felajánlotta, hogy ``űrhajóján'' felvisz
a huszonharmadik emeletre. Félúton ``fotonviharba'' kerültünk, és elszedte minden
pénzem. A huszonharmadik emeleten kirúgott a liftből és én elterültem a műanyag
padlón.

Utólag visszagondolva vicces, de akkor nagyon rosszul éreztem magam. Ahogy
feltápászkodtam, láttam, hogy Jamék ajtaja nyitva van. Az ajtón át a folyosóra
is kiszóródtak az üvegszilánkok. Csend volt. Lélegzetvisszafojtva figyeltem,
míg az automata le nem kapcsolta a folyosó világítását. Ahogy összerezzentem,
visszakapcsolta.

Az ajtón belesve csak az előszobát láttam. A padlót összetört üvegek
borították. A falak műanyag fedőlapjai le voltak hántolva. A nyers fémvázat is
karmolások csúfították el.

— Jam? — próbáltam szólni, de lehet, hogy nem jött ki hang a torkomon.
Kénytelen voltam belépni a lakásba. Emlékszem, hogy ropogott a törmelék a
talpam alatt. Minden lépés után füleltem, de nem hallottam semmit.

A fürdőszoba padlóján találtam meg őket. Jam a földön ült. Owen a padlón
feküdt, fejét fia ölében nyugtatta. A fürdő fehér felületein piros vérfoltok
ragyogtak.

— Howie, — szólított meg Jam. Ettől is összerezzentem. Minden pillanatban
készen álltam hanyatt-homlok elmenekülni.

— Mi történt? — kérdeztem. — Túl sokat ivott?

— Nem. Túl keveset. Az alkohol segített neki elfelejteni dolgokat. Elfelejteni,
ami anyuval történt.

Owen összetörte a lakást és összetörte magát is. Hívtam a mentőket és
segítettem feltakarítani. Nem kérdeztem semmit. De azon az éjszakán Jam
elmesélte, amit a terraformálási balesetről tudott.

A Kék Madár, szülei űrhajója, egy átalakított Orion volt. Egy sima Orion
vonósugarai is elbírnak kisebb aszteroidákkal, de a Kék Madár sugárgenerátorait
egy mechanikus rögzítőrendszerrel egészítették ki a Terraformálási Művek
mérnökei. A csapat feladata az volt, hogy a rendszer legnagyobb üstökösét
eltérítse pályája leggyorsabb szakaszán, méghozzá olyan precízen, hogy
kétszázmillió kilométerrel később telibetalálja a MicroTechet.

A hatalmas jéghegyet pályája közel vitte a rendszer napjához. A Kék Madár a
naphoz legközelebbi ponton akarta elvégezni a korrekciót. De ahogy közeledtek,
különös dolgok történtek. Kimaradozott a rádiókapcsolat. Elvesztek üzenetek.
Törlődtek pályaszámítási eredmények. Túl közel repültek a naphoz, és nem minden
rendszer volt felkészülve a sugárzásra. De az űrhajósok babonás népség, és
mindenben baljós jeleket láttak. Nem tudtak mindent megmagyarázni. Az elveszett
üzeneteket a hajó mégis visszaigazolta. A pályaszámítást megismételve
ugyanazokkal a paraméterekkel más eredményeket kaptak.

A legénység vissza akart fordulni. A manőver szoftverhibák nélkül is elég
veszélyes volt. Túl kockázatosnak látták a kiszámíthatatlan jelenségek után
folytatni az utat. Owen nem hallgatott rájuk. Ő volt a pilóta, tehát a hajó
arra ment amerre ő akarta, de a hajón megromlott a hangulat. Elérték az
üstököst, a hajóhoz láncolták és hozzákezdtek a pályamódosításhoz. A háromnapos
manőver második napján eltűnt Evine, a meteorológus.

Egy űrhajón sokáig lehet bújócskázni, és egy Orion elég tágas négy embernek. De
olyan nincs, hogy valakit ne találjanak meg idővel. Evine mégsem lett meg. A
szenzorok nem jeleztek semmit, de már napok óta nem lehetett megbízni bennük. A
többiek átfésülték a hajót, szétbontottak minden panelt, benéztek minden
tartályba, de Evine nem volt sehol. A légzsilipen át nem távozhatott. A nap
betöltötte az egész baloldali látóteret. Ha ilyen közelségben kinyitották volna
a légzsilipet, az utastér megtelt volna sugárzással.

Egy űrhajón az is szokatlan, ha egy csavarhúzó eltűnik. A legénység egy
tagjának eltűnése mély bizalmatlanságot keltett. Már Owen is vissza akart
fordulni, de össze voltak láncolva az üstökössel. A lehetséges megoldásokról
vitatkoztak, amikor Meria meghalt. Egy mondat közepén mintha kis kockákra
darabolták volna. Szemei tágra nyíltak és szétesett. A kockák Owen és Tellow, a
navigátor lábaira omlottak. Meria vére cipőik körül folyt szét.

A két férfi egymásra nézett és elöntötte őket a pánik. Tellow tudta, hogy nem ő
ölte meg Evine-t és Meriát, tehát csak Owen lehetett. És Owen tudta, hogy csak
Tellow lehetett. Még ha elképzelhetetlen, különös módszer és felfoghatatlan
indíték kellett hozzá, akkor sem lehetett más magyarázat. Összeverekedtek. Épp
mikor Owen felülkerekedett, kinyílt a légzsilip. A hajót ragyogó fény töltötte
be. Versenyt futottak a mentőkapszulához. Tellow érte el előbb, és bezárta az
ajtót Owen előtt. Kilőtte a kapszulát, és Owen látta, ahogy elkanyarodott a
naptól távolodó pályára, a biztonság felé.

Owen egyedül maradt a hajón. Felvette az űrruháját, de mást nem tudott tenni. A
légzsilipet nem lehetett bezárni. A nyitott kapuban állt, az illesztőkarba
kapaszkodott és farkasszemet nézett a mindent betöltő nappal. A fedélzeti
berendezés olvadni kezdett a forróságban. Meria vére felforrott. Az űrhajó
szétesett.

Két hét múlva találták meg Owent. Az űrben sodródott, még mindig az
illesztőkarba kapaszkodva. A hónapokig tartó vizsgálat során egy szavát sem
hitték el. Tellow és a mentőkapszula nem kerültek meg. Gyilkosság és szabotázs
miatt száz év sztázisra ítélték.

Owen akkor is az acélcsövet szorongatta a fürdőszobában, amikor megérkeztek
érte a mentősök. Ők nem bajlódtak sem a lépcsővel sem a Banu-lakta lifttel. Egy
lebegővel dokkoltak a lakás ablakához és ott hozták be a hordágyat. Mielőtt
felemelték volna a nagydarab férfit, Owen egy pillanatra magához tért és rám
nézett.

— Vidd vissza — mondta, és nekem adta a kart. — Vidd vissza a válaszom.

Húsz évet kellett várnom, hogy megértsem, miről beszélt. Nektek csak a
történetem végét kell megvárnotok.

\secbreak

Így szereztem be a mágneses illesztőkart. Egy évre rá már minden
nélkülözhetetlen alkatrészt sikerült begyűjtenem. Még egy év lehetett, mire
repülésre készen össze is szereltem őket. A hajónak a Szánkó nevet adtam. A
dokkőr büszke volt rám. Én is büszke voltam magamra.

Aszteroidákat térképeztem, űrszemetet gyűjtöttem, apróságokat csempésztem. A
bevételből mindig jobb alkatrészeket vettem, vagy megjavítottam, ami
meghibásodott. Segítettem sötét dokkokban elvégzett javításokkal egy
kalózbandának. Segítettem egy sikertelen forradalmárnak a Hurston körül
kémműholdakat elrejteni. Segítettem a MicroTech Biztonsági Osztálynak egy
elrabolt alelnököt megtalálni. Mikor nagyobb meghajtóm lett, kereskedtem is.
Mikor egy üzlet rosszra fordult, a rendszeren kívül éltem öt éven át. Mikor
visszatértem, magánnyomozónak adtam a fejem és átépítettem a hajóm egy új
vázra. Új nevet is kapott. A Száncsengő.

Veszélyes, értelmetlen és rosszul fizető karrierem volt. Mégse cserélném másra.
Sok haragosom volt, de sok barátom is, és megismertem nagymamátokat. Mikor
2892-ben nagybátyátok születését vártuk, tudtam, változnia kell valaminek. Egy
legális, jól fizető munkára volt szükségem.

Ez a munka szerencsére meg is talált. Egy birodalmi ügyész keresett meg. A
méltóságos Bryan Wingate.

Erről nem írtak akkor az újságok, de most már nem titkosak a részletek. Ötven
évvel ezelőtt a Birodalmi Ügyészség nyomozást folytatott a MicroTech ellen. A
gyanú az volt, hogy maga a MicroTech szabotálta a bolygó terraformálását, hogy
olcsón megvásárolhassa. Hivatalosan csak évtizedekkel a terraformálás után
született a döntés a bolygók elárverezéséről. De Bryan bebizonyította, hogy a
csillagközi vállalatok kapcsolataik folytán előre tudtak az árverésről. Ezért
tudtak olyan rettenetes összegeket kifizetni.

A nyomozás következő lépése a baleset részletes kivizsgálása volt. Bryan a
Stanton rendszerbe repült, és szüksége volt egy helyi nyomozó segítségére. A
bolygón viszont szinte minden nyomozó a MicroTech alkalmazásában állt. A
szabadúszók szűkös kínálatából rám esett a választása. Nyomozóként és
pilótaként is szerződtem az Ügyészséggel, és a hajómat is bérbe adtam. Hogy ne
legyen érdekem lassítani a munkát, a nyomozás lezárásához kötötten is jelentős
jutalmat ígértek. Nem láttam veszélyt egy több, mint száz éves ügy
vizsgálatában. Nem is lehettem volna boldogabb.

Bryan először is kikérte a véleményemet a balesetről. A nyilvános jelentések
alapján mit gondoltam, mi okozhatta a balesetet? Megtisztelt, hogy egy
birodalmi ügyésznek az én szakvéleményemre van szüksége.

— A hajó nem volt hitelesítve akkora sugárzási szintre — mondtam. — Ha
meghibásodott a fedélzeti vezérlőrendszer, összeütközhettek az üstökössel.

Ez volt az eredeti nyomozás egyik lehetséges magyarázata. Nyolc különböző
magyarázatot találtak az esetre, és nem volt megbízható módja, hogy ezek
bármelyikét kizárják. Az űrben elkövetett bűncselekményeknél sokszor elpusztul
minden bizonyíték. Elfogadott eljárás, hogy a pontos igazság helyett a
lehetséges magyarázatok szűkebb vagy tágabb körét állapítja meg a bíróság. A
magyarázatokhoz tartozó legenyhébb büntetést szabják ki ilyenkor. Mivel a
bolygó hógolyóvá változott, a hajó és teljes legénysége odaveszett, és egyedül
a pilóta élte túl, Owen esetében a legenyhébb büntetés is száz év sztázis volt.

— Hogyan igazolhatnánk, hogy ez történt? — kérdezte Bryan.

— Megtalálhatnánk a Kék Madár fedélzeti naplóját. Vagy megtalálhatnánk az
ütközés nyomait az üstökösről készült csillagászati megfigyeléseken.

— A napló nincs meg. A megfigyeléseket ellenőriztem. Az incidens után készült
felszíni térképeken nincs nyoma friss karcolásoknak. Elemeztem a légkörbe lépő
üstökös színképét is. Az izzó anyagok fénye megfelel egy vízjég-üstökös
színképének. Nincs nyoma az űrhajó építőanyagaira jellemző fémeknek és
lantanoidáknak.

— Az incidens pozíciójáról készült közvetlen csillagászati megfigyeléseken nem
észlelhető sem a Kék Madár, sem az üstökös — folytatta. — Túl közel voltak a
naphoz. De egy erőteljes robbanás észlelhető lenne, tehát egy nagysebességű
ütközés kizárható. És lenne még egy módja a feltételezés igazolásának. Egy
tanuvallomás.

— Owen, a pilóta, meghalt egy évvel ezelőtt.

— Tudom. És a temetésről készült felvételeken láttam, hogy te is megjelentél.
Ismertétek egymást?

Bryan nagyon kedves volt, de ijesztően sokat tudott. Tele volt kérdéssel, de
minden kérdésre tudta a választ is. Gyorsan beláttam, hogy nem a
szakvéleményemre volt szüksége. Csak udvariasan akarta megosztani az új
nyomozásban addig begyűjtött eredményeket, és leellenőrizni a hozzáértésemet és
őszinteségemet. Elmeséltem neki a gyerekkori találkozásunk történetét. Nem
tudom, hogy ebben volt-e számára új információ.

Megkerestük Jamet is. Iskola után is tartottam vele a kapcsolatot. Ő volt az
egyetlen ismerősöm a MicroTechen aki nem elektronikával vagy űrhajókkal
dolgozott. Egy kis klinikát üzemeltetett a déli féltekén, főleg
szenvedélybetegek számára. Ott találkoztunk. Bryan jól szót értett vele.
Elnyerte a bizalmát, majd előállt egy meglepő kérdéssel.

— Hiszel a csillaglényekben?

Jam csak nézett rá. Nem meglepetten. Haragosan. Megfeszült rajta a fehér ápolói
köpeny. Rám nézett sértetten, és vissza Bryanre.

— Nem.

Nem értettem a kérdést, de azt láttam, hogy Jamnek nem esett jól. Valahogy
kimentettem magunkat, és az utcán kérdőre vontam Bryant. Azt mondta, jobb, ha
Owentől hallom a magyarázatot.

Az utunk a MicroTech Mauzóleumba vezetett. Ez a cég és a bolygó hálózatról
leválasztott archívuma. A bizalmas adatokhoz csak személyesen lehet hozzáférni,
és csak a MicroTech legfelsőbb vezetőinek engedélyével. Vagy a Birodalom
legfelsőbb vezetőinek felhatalmazásával, mint ahogy a méltóságos Bryan Wingate
és Howard papátok tette.

Itt tárolták a Kék Madár balesetét vizsgáló nyomozás és az Owen Joys elleni per
jegyzőkönyvét. A Mauzóleum egy recepcióból és egy kis ablaktalan szobából állt.
A szoba egyetlen keskeny ajtaja két méter vastag volt. A falai három méter
vastagok is lehettek. Minden fajta sugárzás ellen szigetelték. A hangok olyan
gyorsan haltak el odabenn, hogy nem is voltam biztos benne, mondtam-e valamit.
A fény is furcsa volt, de nem tudom megmondani, miért. Mintha semmi nem vetett
volna árnyékot.

Bryan egy videókkal teli mappát nyomott a kezembe. A teljes időtartam több,
mint három nap volt. Bryan is először látta a felvételeket, de valahogyan mégis
tisztában volt a tartalmukkal.

— Mi volt a küldetésük célja? — kérdezte a Terraformálási Művek nyomozója az
első vallomástevés felvételén.

— A célunk a 191P/2766 D187 üstökös pályájának módosítása volt. Sugárirányú
parabolikus pályára kellett állítanunk, hogy merőlegesen lépjen be a Stanton I
légkörébe. A küldetés nem sikerült.

— A biztonsági előírások ellenére beléptek a nap körüli forró zónába. Miért?

— A négy óriásbolygó terraformálásával a Birodalmi Terraformálási Művek négy
külön brigádot bízott meg. A 191P volt a rendszer legnagyobb üstököse. Mind a
négy brigád fel akarta használni a munkájához. A Stanton I volt a legszárazabb.
Nekünk kellett legjobban az óriási üstökös. Ha megvárjuk, míg kilép a forró
zónából, a Stanton II vagy Stanton IV brigádja előbb érte volna el.

Owen annyi idős lehetett a felvételen, mint a\-mennyi én voltam. Emlékszem, hogy
alig ismertem meg. Gyerekkoromban mint ijesztő, kövér óriás állt előttem, de a
felvételeken nagyon vékony volt.

Az űrhajósok diétája szigorú. Egy kéthónapos úton nem tudsz megállni egy gyors
hamburgerre. Sokan elhíznak aztán, amikor felhagynak az űrutazással. Az Owen,
akit gyerekkoromban láttam, biztos nem tudott ellenállni a minden pillanatban
elérhető változatos ételek vonzerejének. A felvételen viszont azután láttam,
hogy két hétig egy űrruhában sodródott.

Owen összeszedett volt az első néhány kikérdezés során. De egyre zaklatottabb
és zavarodottabb lett ahogy újra és újra feltették a kérdéseket.

— A biztonsági előírások ellenére beléptek a nap körüli tiltott zónába. Miért?

— Csapda volt. Idecsaltak minket. A jéghegyet is ők csinálták. Felhasználják az
embereket. Nekik a bolygók csak porszemek. Kihúnyt csillagokból kihullott hamu.
Nincs jelentőségünk számukra. De észrevettek. És felhasználnak.

— Mi történt a Kék Madár legénységével?

— Meghaltak. Meria is halott. Láttam, ahogy felbontották. Felbontották őket a
csillaglények. Engem is felbontottak.

— Úgy érti, idegenek megölték őket? És ön hogy élte túl, ha ``felbontották''?

— Engem áttettek egy csillaglény testébe. A rabjuk voltam egy évig. Vagy csak
egy másodpercig? Nekünk egy év hosszú idő és egy másodperc rövid. De a
csillaglényeknek nem. Nekik egy millió év a hosszú idő és egy milliomod
másodperc a rövid.

— Tehát bizonytalan ideig a rabjuk volt. Le tudja írni a börtönét? Le tudja
írni a fogvatartóit?

— A nap mélyén tartottak fogva. Ők nem anyagi természetűek, nem úgy, mint mi. A
testük mágneses örvények állandóan változó önfenntartó rendszere. Az örvény
maga gyorsabban tud mozogni, mint az örvényt alkotó anyag. Kitágulhat,
összehúzódhat. Nekünk egy tíz méteres tárgy óriási, egy egy milliméteres tárgy
alig észrevehető. Nekik ezer kilométer az óriási és egy ångström az alig
észrevehető. Markukban hidrogén atomokat tudnak héliummá gyúrni.

— Igen. És hogyan szabadult ki a börtönéből?

— Megszöktettek. Ők sem egységesek. Folyamatosan háborúban állnak egymással. A
háborúik évmilliárdokig tartanak. A csatáik másodpercekig.

— Miért szöktették meg?

— Végtelenül ravaszak az intrikáik. Az emberi agy túl kicsi. Nem tudom
elmondani. Nem lehet. Megszöktettek és kaptam tőlük egy új testet. Újra ember
lettem. Újra gyenge emberi agyam van. Semmire nem jó.

— Rendben. És ezután az űrben sodródott, míg a Domino 22 mentőhajó meg nem
találta. Milyen bizonyítékkal tudja alátámasztani a történetét?

— A kar… A karom. A bal karom hosszabb lett. Ez nem az igazi testem! Hol az
igazi testem? Kockákra vágták!

Owen itt veszélyessé vált és le kellett fogni. A tizenhetes sorszámú interjú
véget ért.

A vallomás nagy vonalakban egybevágott azzal, amit Jamtől hallottam húsz évvel
korábban. A műszaki hibák. A meteorológus eltűnése. Meria véres halála. A
verekedés a navigátorral. A harc a mentőkapszuláért. A nyitott légzsilip.

De Owen mindezért a Stanton napjában lakó lényeket tette felelőssé. Isteni erőt
és tudást tulajdonított nekik. Azt mondta, ők építették az ugrópontokat is,
amik lehetővé tették, hogy az emberiség bejárja a galaxist. Az ugrópontok
rendszere az ő kivetett hálójuk volt szerinte. És hogy Owen is csillaglényként
élt. Sokat tanult közöttük. A csillaglények a világegyetem első intelligens
lényei voltak, már az első bolygók megformálódása előtt kifejlődtek.
Évmilliárdokkal előttünk járt a tudományuk Owen elmondása szerint. De sajnos
mikor visszatették emberi testbe, az apró emberi agy nem tudta megőrizni
mindazt, amit megtanult.

Az eredeti nyomozás és a per elsősorban Owen elmeállapotáról szólt. Őrületét
tényként kezelték. Csak az volt a kérdés, mikor és miért őrült meg. És
őrületében mit tett a legénységgel és a hajóval.

Egy héten át néztük a Owen vallomásáról készült felvételeket. Próbáltuk
kihámozni a vad lázálmok közül a racionális magyarázathoz szükséges jeleket.
Építettünk egy részletes modellt arról, hogy a legénység tagjai melyik
pillanatban hol lehettek a hajón. Szimulációkat készítettünk a fedélzeti
számítógép különböző meghibásodásairól. Próbáltuk megfejteni a rejtélyt.

— Mik a lehetséges magyarázatok? — kérdezte Bryan.

— Lássuk csak. Egyes számú lehetőség — kezdtem bele. Bryan precíz, strukturált
gondolkodásmódja kezdett rám is átragadni. Sokat tanultam tőle. — Baleset. A
Kék Madár fedélzeti számítógépe meghibásodott az erős sugárzástól, és
ismeretlen módon a hajó szerkezeti összeomlását okozta. Owen hibája, hogy a
hajó nem volt felkészülve, és hogy az első meghibásodások után is folytatták a
küldetést. A bűntudattól félőrülten próbálta magát felmenteni a felelősség
alól.

— Kettes számú lehetőség — folytattam. — A navigátor, Tallow, MicroTech ügynök
volt. Megölte Evine-t és Meriát, szabotálta a légzsilipet, és sorsára hagyta a
hajót. Egy másik MicroTech ügynök felvette Tallow mentőkapszuláját. Owen
hallucinált az erős sugárzástól és a szomjúságtól.

Ez volt a legerősebb elméletünk, és ennek a bebizonyítása lett volna a
legnagyobb horderejű az Ügyészség számára. De nem volt könnyű bizonyítékokat
találni.

— Tallow nem élte túl az incidenst. Egy éven át kerestem a nyomait.
Megvizsgáltam minden Birodalmi rendszerben minden esetet amikor valaki Tallow
kedvenc pizzáját és az általa használt sampont egy napon vásárolta meg. Minden
esetet, amikor az ő navigációjára jellemzően sorolt be dokkoláshoz egy űrhajó.
Ha barlangi remeteként élt volna egy Banu bolygón, arról is tudnék. Ha tényleg
ő volt a szabotőr, vele is végeztek. Tiszta munka… Mi lehetett még?

— Hármas számú lehetőség. Az üstökösön rejtőzött egy másik hajó. Innen
szabotálták a Kék Madarat és követték el a gyilkosságokat.

— Igen. Ennek előnye az elkövetők számára, hogy bármelyik brigád érte volna el
elsőként az üstököst, ők be tudtak volna avatkozni.

— Számunkra pedig az az előnye, — mondtam, — hogy nagyon kevés űrhajó és nagyon
kevés pilóta tudna tökéletesen elrejtőzni egy üstökös mögött. Ha ott volt, de a
Kék Madár nem vette észre, akkor egy apró egyszemélyes gép kellett, hogy
legyen. Kisebb, mint egy Hornet. Vagy egy módosított, szárny nélküli Hornet,
vagy egy F7C-S. Leszűkül a gyanusítottak köre.

Bryan jól értett a hajókhoz. A katonai specifikációkat nálam is jobban ismerte.
De nem bütykölt hajókat a saját két kezével. Nem tudta, mi lehetséges és mi
nem. Én sem láttam még szárny nélküli Hornetet, de tudtam, hogy meg lehet
csinálni. A szárny csak üzemanyagot tárol, a légköri manővereknél segít és egy
hosszú antenna van belehúzva. Leszerelném őket, ha egy üstökösön kéne
megbújnom.

— Négyes számú lehetőség. Owen volt a MicroTech ügynöke — folytatta Bryan. — A
szabotázs után elfogták és emlékezetmódosító műtétet végeztek rajta. Ennek a
beavatkozásnak a következménye a csillaglények közti rabság emléke. Nincsenek
hivatalos feljegyzések az emlékezetmódosítás mellékhatásairól, de az Ügyészség
archívumában találtam titkosított leírásokat. A három feljegyzett áldozat közül
kettőnél részletesen vizsgálták a kialakult tévemlékeket. A tér és idő zavara
mindkettőnél jellemző volt. Egyiküknél a kiszolgáltatottság erős érzése is
áthatotta a kialakult álomképeket.

— Ötös számú lehetőség. Tényleg léteznek a csillaglények!

Vicceltem, de Bryan nem nevetett.

— Csak mert valami valószínűtlen, még nem zárhatjuk ki.

\secbreak

Az ügyészség fizetett a nyomozásért. Nem bántam, hogy alapos munkát végeztünk.

Mindenkit megvizsgáltunk, aki érintett volt az ügyben. Ellenőriztük a vallatást
végző nyomozók bankszámláját. Valaki az egyik beszélgetés során behozott egy
csésze kávét. Lenyomoztuk az ő nagybátyjának a politikai kapcsolatait.
Végignéztük a perről készült felvételeket is. A bíró kivel járt együtt
golfozni. Hol vacsoráztak az ügyvédek. Kik vettek részt a Kék Madár építésében.
Melyik alkatrészt honnan szerezték be. A legénység milyen hangulatban volt
indulás előtt, ki hogyan érzett a többiek iránt. Hiába telt el több mint száz
év, a Birodalmi Ügyészség adatbankja megőrzött minden kommunikációt és
pénzmozgást.

Szép nyár volt. Mindenki odakint sétált és játszott a sekély hóban. De mi
hónapokon át alig mozdultunk ki, csak az adatbank ódon állományait bújtuk.
Örültem, hogy nagymamátok közelében lehettem. Szépen gömbölyödött a hasa. Mégis
vágytam rá, hogy újra a pilótafülkémben üljek és kinyíljanak fejem felett a
dokk súlyos kapui. Eljött ennek is az ideje.

A nyomozás minden szála az űrbe vezetett. Ha megtalálnánk a hajónaplót,
bizonyíthatnánk a balesetet. Ha megtalálnánk a roncs egy darabját,
bizonyíthatnánk a külső támadást. Ha megtalálnánk a mentőkapszulát, megtudnánk
Tallow sorsát. Ha megtalálnánk a csillaglényeket… minket is bolondnak néznének.

Kibéreltük egy órára a MicroTech elosztott teleszkóphálózatát és
végigpásztáztuk a belső keringési pályákat. Az incidensben létrejött bármilyen
kémiai nyom, vagy elveszített alkatrész olyan pályára kellett, hogy álljon, ami
metszi az üstökös eredeti pályáját. Ilyen távolságból nem találhattunk semmi
konkrétumot, de be tudtunk azonosítani a legjobb vadászhelyeket. Ezeket
kellett, hogy végiglátogassuk és átfésüljük.

Az expedíció előtt jelentős átalakításokat végeztem Száncsengőn. A hajómra a
civil piacon kapható legnagyobb DiSys radart szereltem. A mai WillsOp
radarokhoz képest egy ormótlan szerkezet volt, és nem is volt forgatható, csak
előre látott. De tízezer kilométeren belül egy elkallódott anyacsavart is
megtalált volna.

Kíváncsiak voltunk fennmaradt kémiai lenyomatokra is. Bár több mint százhúsz év
eltelt, bíztunk benne, hogy ha felrobbant a Kék Madár, a szétszórt gázok
nyomokban még azonosíthatóak lesznek. Semmi nem tűnik el, csak felhígul. Nem
árulnak persze minden kikötőben olyan érzékeny műszert, amivel néhány kallódó
molekula észlelhető a bolygóközi űrben. Szerencsére szomszédunk, a Stanton
rendszer negyedik bolygója, a Crusader. Ipari teherhajókat és munkahajókat
építenek. Az ő egyik lebegő városukban találtam meg és telepítettem a megfelelő
műszert, egy Chimera nanohálót.

Felpakoltunk egy hónapra elegendő vizet és ételt, és vadászni indultunk.

A DiSys radar tömegével alig bírtak a Száncsengő hajtóművei. Lassan cammogtunk
a rendszer belseje felé. Hasznosan töltöttük az időt addig is. Bryan tovább
dolgozott a MicroTechen megkezdett modelleken, én pedig az új műszereket
kalibráltam. A gyártók persze elvileg tökéletesen beállítják őket, de sosem
tudni, hogy a hajó egyéb berendezései milyen hatással vannak rájuk.
Bekapcsoltam a radart és kibocsátottam a nanohálót. Napokig tartott, míg a
napszél kifeszítette a háló végtelenül finom szövetét. Öt négyzetkilométeres
területén minden háromatomnál nagyobb molekulát érzékelt. A Chimera hálónak a
hirdetés szerint jobb a szaglása, mint a Föld összes kutyájának együttvéve.

A nap egyre csak nőtt. Mikor a MicroTech körüli pályáról megszokott méret
százszorosára nőtt, be kellett csukjuk a pilótafülke spalettáit. A hűtőrendszer
maximális fokozaton működött, mégis szauna lett a Száncsengőből. A méltóságos
Bryan Wingate próbálta megőrizni méltóságát, így én is magamon tartottam egy
törölközőt. Szenvedtünk, de megérkeztük a megcélzott pályára. Ezen a távolságon
pusztult el a Kék Madár, és itt reméltünk a rejtély megfejtésére akadni.

A nanohálót nem lehet behúzni, túl finom hozzá. Egyszerhasználatos. És a
gyorsulást is rosszul tűri. A hajtóműveket kikapcsolva sodródtunk a retrográd
pályán. A Stanton rendszerben a legtöbb bolygóközi részecske az óramutató
járásával megegyezően kering, mi pedig ezzel ellenkező irányban soroltunk be.
Ennél a távolságnál tizenegy standard földi nap volt a körpálya keringési
ideje, tehát egy szűk hét alatt szembetalálkozhattunk minden, a szokásos
irányban keringő objektummal.

Három napig semmi nem történt. Az izzasztó melegben le is adtam azt a pár
kilót, amit az otthon töltött hónapok alatt felszedtem. A mirelit ételeket már
kiolvasztás nélkül ettük. Sem a radar, sem a nanoháló nem érzékelt semmit. Hogy
elüssük az időt, mindenről beszélgettünk. Fogadásokat kötöttünk a nyomozás
kimenetelére. Bryan a karrierjét tette rá, hogy a MicroTech bűnös, én pedig a
bolygómat, hogy ártatlan.

A negyedik napon végre dübörögni kezdett a Vanduul heavy metal. Az Up In
Armitage együttes slágerét állítottam be ébresztőnek, ha a radar észlel
valamit. Egy napok óta épülő kártyavárat felrúgva vetettük magunkat a
pilótafülkébe. Egymillió kilométerre egyenesen előttünk érzékeltünk valamit.
Épp a radar felbontásának határán volt, tehát a pontos méretét nem tudtuk
megmondani. Lehetett akár mentőkapszula, vagy egy roncs darabja is. Nyolc
percünk volt a találkozásig. Megnyitottam a vonósugár célzóprogramját és
vártam, hogy fel tudjuk szippantani a tárgyat.

Egy perccel később a tárgy eltűnt. Tárgyak nem szoktak eltűnni, tehát a DiSys
radar kellett, hogy meghibásodjon. Radar nélkül nem tudjuk befogni a tárgyat.
Kilyukasztja a nanohálót, és sose tudjuk meg, hogy mi volt az. Bár ütközésnek
gyakorlatilag semmi esélye nem volt, legrosszabb lehetőségként még ez is
eszembe jutott.

— Kimegyek és helyrekalapálom — mondtam Bryannek két átkozódás között. A
légzsilip felé futva ledobtam magamról a törölközőt és felmarkoltam egy űrruhát
a szekrényből. Beindítottam a zsilipet, és míg kiszivattyúzta a levegőt
körülöttem, bebújtam az űrruhába.

— Hat perc — szólt Bryan hangja a szkafanderemben.

Elmúlt a mesterséges gravitáció, kinyílt a zsilipkapu, és már másztam is a
radar felé. Dőlt rám a rettentő sugárzás, de az űrruha fel volt erre készülve.
Még jobban is hűtött, mint a fedélzeti rendszer. Mászás közben a hajó haladási
irányába dobtam az Arclight replika pisztolyomat. Sosem tanultam meg jól lőni,
és szerencsére sosem bántam meg.

— Bryan, látsz valamit a radaron?

— Semmit.

Tehát tényleg a radar volt a hibás. Az öt méter átmérőjű rádioantenna ott volt,
ahova felhegesztettem indulás előtt. Megkarcolódhatott, vagy ki is
lyukaszthatta egy mikrometeor, de akkor is működnie kellett volna. Biztosan a
kábelezés hibásodott meg. Letéptem a panelt, és elkezdtem szétszedni és
összerakni a kábelek illesztéseit.

— Öt perc. Még mindig nem látok semmit.

— Négy perc. Még mindig semmi.

— Három perc.

— Howard, gyere be! Hagyd a radart! — hívott Bryan izgatottan. — Növekszik a
sugárzás, mindjárt elkap minket egy napkitörés!

Nem akartam elhinni a balszerencsémet. Nem volt időm már visszamászni a
zsiliphez. A nap felé fordítottam a haszontalan radarernyőt és az árnyékában
bújtam el. Épp rádiózni akartam Bryannek, amikor elért minket a kitörés. A
Száncsengő energiapajzsa felizzott és viharos színkavalkádba borított mindent.
Mintha forgószél szippantott volna fel egy festékgyárat. A veszélyes
sugárzástól megvédett minket, de az életre ártalmatlan hullámhosszokat
átengedte. A szkafander rádiója a pokol minden hangjával zörgött a fülemben.

Amilyen gyorsan kezdődött, olyan lassan maradt abba az érzékszerveim ostroma. A
zörgés lassan elhalkult, és lehet, hogy abba is maradt. Annyira csengett a
fülem, hogy nem tudom biztosan megmondani. A pajzs is fokozatosan átlátszóvá
vált.

— Kint minden rendben — rádióztam.

— Bent minden rendben — rádiózta Bryan.

Kibújtam a radar árnyékából és visszafordítottam az útirányba. 

— Bryan, látsz valamit a radaron?

— Mintha egy pisztoly távolodna tőlünk lassan.

Kicsit meglepett volt, hogy hogy került oda egy pisztoly. Én is kicsit
meglepett voltam, hogy megjavult a radar. Az ismeretlen tárgy sajnos már rég
elhagyott minket. Nem ütköztünk össze. A nanohálón kellett, hogy lyukat üssön.

A fejemet forgattam, hogy megtaláljam a sérülést a finom szöveten, mikor
megláttam a tüzet. A nanoháló nap felé eső oldala lángolt. Nem tudom, milyen
vegyületekből csinálják a Chimera hálót, de mint kiderült, oxigén nélkül is jól
ég. Persze egy napkitörés kellett a begyújtásához.

A tűz a hajó felé terjedt. Le kellett vágnom a háló baloldali pányváit. Míg
nyiszáltam a rugalmas fonatokat, megtaláltam a tárgy által ütött lyukat is.
Közel volt a hajóhoz. Szabályos kerek lyuk, talán egyméteres átmérővel. A
lángok körültáncolták. Az izzó nanoháló kavargó mintákat rajzolt köré.

Alighogy levágtam, a háló baloldalát elemésztették a lángok. De a tűz nem
terjedt át a jobboldalra, és a hajó is sértetlen maradt. Kicsit megnyugodva
indultam vissza a zsiliphez.

Ahogy a zsilip ajtaját nyitottam volna, megálltam egy pillanatra. Bal kezem a
Kék Madárról örökölt mágneses illesztőkarba kapaszkodott. A kesztyűmtől balra a
gyártó cég különös logója csillogott. Rengeteg kör volt finoman egymásba
gravírozva. A minta roppant rendezett volt, de mégsem lehetett megfogalmazni a
rendezőelvet. Elvettem a kezem. Alatta a logó ügyetlen mása volt. Mindig azt
hittem, Jam véste bele, mikor kicsi volt. Itt nyoma sem volt finomságnak, és a
szerkezetet sem másolta le jól. De ez a rajz is a szimmetria megfoghatatlan
illúzióját keltette. Úgy, mint a nanohálón körtáncot lejtő lángok.

Hirtelen megértettem, miről beszélt Owen húsz évvel korábban.

— Vidd vissza a válaszom — mondta a hordágyon, amikor elvitték a mentősök.

A különös jel nem logó volt. A durva másolatot nem Jam, hanem ő maga véste oda.
Ez volt Owen válasza, és én voltam a postás.

A Száncsengő hajtóművei beindultak és elrepültek. Az egyik jobbra ment, a másik
balra. Mintha összevesztek volna egymással és a hajó többi részével.

— Howard, valami nem stimmel a fedélzeti rendszerrel — hallottam Bryan hangját.

— Ötös számú lehetőség — motyogtam. — Szétesik a hajó.

— Gyere a mentőkapszulához! Siess, nem tudom, mi történik!

A légzsilip külső kapuja leoldott és egy lassú piruettel eltáncolt mellettem. A
belső kapu sem volt már a helyén. A kiszökő levegőben egy másodpercig hallottam
a széteső hajó hangjait. Jó nevet adtam neki. Az elszabadult fémalkatrészek
csilingeltek az utastérben.

A légzsilip kapujával szemben volt a mentőkapszula bejárata. Bryan épp akkor
vetette be magát. Rajta nem volt űrruha, szerencséje volt, hogy elérte a
kapszulát. Nem hallottam, de a kapszula üvegajtaján át láttam, ahogy kiált
nekem. Integetett, hogy menjek vele. A maga melletti ülésre mutogatott.

Körülöttünk végképp darabokra hullott minden. A radar nagy ernyője a nanohálóba
akadva keringett a hajó körül. A pilótafülke tizenkét üveglapja a semmiben
forgolódott, és szétszórta a nap ragyogó fényét. Már a váz panelei is elváltak
a grafén merevítőcsövekről. Minden csavar, amit a hajómon valaha meghúztam,
elengedett, minden hegesztés kiolvadt, minden alkatrész elszabadult és saját
útjára indult.

Én a mágneses illesztőkarba kapaszkodtam és néztem a képtelen jelenetet.
Intettem Bryannek, hogy ne várjon rám. Tovább kiabált, de mikor elkezdtek
kigyulladni körülöttünk a bútorok, rácsapott a kapszula indítógombjára.
Elrepült, és tudtam, hogy nem látom többet.

\secbreak
\itshape

— A csillaglények! A csillaglények! — kiabálták a gyerekek Howard papa körül.
Lélegzet visszafolytva hallgatták eddig a mesét, de most érezték, nem várhat
tovább az igazság pillanata.

— Két hétig sodródtam az űrben. Csak én és ez az illesztőkar — folytatta
Howard. Tekintete időnként az ölében pihenő acélcsőre tévedt. — Nehezen talált
meg a MicroTech mentőcsapata, mert már egy nappal korábban nyomát vesztették a
Száncsengő regisztrációs jelének. Milyen volt a csillaglények között? Milyen
érzés egy csillag mélyén élni a világegyetem legősibb, leghatalmasabb lényei
között? Nem tudom. Engem nem ejtettek fogságba. Ép elmével túléltem a
hánykolódást.

— Bryan mentőkapszuláját nem találták meg. A nyomozók azt mondták, a
napkitöréstől megsérülhettek a kapszula rendszerei is. Nyilván ez a kitörés
pusztította el a Száncsengőt. Bár nem így éreztem akkor, szerencsém volt, hogy
rajtam volt a nagy sugárzású környezetekre tervezett űrruha.

— Hogy a jövőbeli baleseteket elkerüljék, kijelöltek egy tiltott zónát a
csillag ötvenmillió kilométeres környezetében. Civil és katonai űrhajóknak is
engedélyköteles a belépés, és nem adnak ki engedélyeket. Új Birodalmi Ügyész
kapta meg az ügyet, aki Stanton meglátogatása nélkül lezárta a nyomozást. Engem
kifizettek, és egy pénzjutalommal nem járó kitüntetést is kaptam tőlük.

— De nem a kitüntetés jut eszembe, ha erre az ügyre gondolok, hanem ez az acél
cső — mondta Howard a levegőbe emelve az illesztőkart. — Látjátok, itt volt a
finoman gravírozott minta. És itt volt Owen odavésett válasza. És mindkettő
nyom nélkül eltűnt, míg az űrben sodródtam. Ezt magyarázzátok meg!

A hallgatóságnak csak ennyi kellett. Mind meg akarták nézni az acélcsövet,
kezükbe fogni a súlyos bizonyítékot. Gyerekricsaj töltötte be a nagy faház
étkezőjét. Késő estig űrhajósat és csillaglényeset játszottak, és azt
tervezték, hogyan fognak mindent kinyomozni, ha nagyok lesznek.

\upshape

\begin{center}
Vége
\end{center}

\chapter*{Jegyzetek}

A novellát a \url{starcitizen.hu} 2014-es novellaíró pályázatára írtam.

\begin{description}

\item[Szereplők:] ~ \par
Howard, narrátor \par
Jam, iskolatárs \par
Owen, pilóta \par
Meria, biológus \par
Evine, meteorológus \par
Tallow, navigátor \par
Bryan, ügyész \par

\item[Dátumok:] ~ \par
Owen született: 2723 \par
Jam született: 2758 (Owen 35 éves) \par
Terraformálás: 2769 \par
Owen sztázisban: 2770–2870 \par
Howard született: 2858 \par
Synthworld: 2872 (Howard és Jam 14, Owen 49 éves) \par
Nyomozás: 2892 (Howard 34 éves, Owen 69 lenne) \par
Ma: 2942 (Howard 84 éves) \par

\item[MobiGlas:] Smartphone. \textit{Writer's Guide: Part Eight}

\item[``Aloha'':] A ``wewl-whoa'' Banu üdvözlés félrehallása. \textit{Timothy
Brown, A Human Perspective, Episode 3}

\item[Stanton:] Egy csillagrendszer. \textit{Jump Point, Issue 6, Galactic
Guide: Stanton System}

\item[RSI Constellation:] 4-5 fős népszerű, sokoldalúan használható űrhajó. \textit{Jump
Point, Issue 6}

\item[RSI Orion:] 8 fős bányászhajó. \textit{Star Citizen Wiki}

\item[191P/2766 D187:] 191-edikként felfedezett periodikus (P) üstökös, és a 187-edik
üstökös amit 2766 február második felében (D) felfedeztek. Ekkor kezdték
feltérképezni az üstökösöket a terraformáláshoz.

\item[DiSys:] Domináns szenzorgyártó 2922 előtt. \textit{Galactic Guide: WillsOp Systems}

\item[Chimera:] Szkennergyártó. \textit{Writer's Guide: Part Four}

\item[``Wingate'':] \textit{The Shadow Out of Time} hőse.

\item[8 perc a találkozásig:] A MicroTech keringési ideje 350 nap. Ha százszor
erősebben süt a nap a Száncsengőre, akkor tízszer közelebb kell legyen. Ezen a
pályán a keringési idő 11 nap ($350 \mbox{ nap} \times 0.1^{1.5}$). Ha feltesszük, hogy a Stanton
csillaga hasonló a miénkhez, legyen 1 AU (150M km) sugarú a MicroTech pályája.
A belső pálya hossza akkor $30 \mbox{M km} \times \pi = 94 \mbox{M km}$, a keringési sebesség tehát kb 1000
km/s. Az űrhajó és az objektum szemben mennek, tehát a relatív sebességük 2000
km/s. Egymillió km-es távolságot tehát 500 másodperc (8 perc és 20 másodperc)
alatt zárnak.

\item[``Up In Armitage'':] Az Armitage bolygót letarolta egy Vanduul flotta. \textit{Jump Point:
Orion System}

\item[Arclight:] Lézerpisztoly. \textit{Galactic Guide: Klaus \& Werner}
\end{description}

\cleartoverso
\cleartorecto
\end{document}
