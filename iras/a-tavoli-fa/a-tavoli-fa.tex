\documentclass[10pt]{memoir}
\usepackage{pifont}
\usepackage{charter}
\usepackage[utf8]{inputenc}
\usepackage{t1enc}
\usepackage[hungarian]{babel}
\usepackage[final]{microtype}
\clubpenalty=10000
\widowpenalty=10000
\raggedbottom
\setstocksize{6.87in}{4.25in}
\settrimmedsize{6.87in}{4.25in}{*}
\isopage[12]
\checkandfixthelayout
\setlength{\emergencystretch}{2em}

\renewcommand{\pfbreakdisplay}{\bigskip \ding{166} \bigskip}
\newcommand{\secbreak}{\fancybreak{\pfbreakdisplay}\indent}
\newcommand{\napdal}{
  Fényesség, sárgán ragyogó érem, \\*
  Segítséged felhők alatt alázattal kérem. \\
  Fényesség, drága sugarak télen, \\*
  Cseppbe zárva mosolyogtat kedves édességem. \\
  Fényesség, égben ragyogó élet, \\*
  Éjszakára megpihen, de visszavár a lélek. \\*
  De ha elfáradtál, szívesen cserélek.
}

\author{Darabos Dániel}

\date{2012}

\title{A távoli fa}

\begin{document}

\begin{titlingpage}
  \centering
  \vspace*{0.2\textheight}
  {\Huge \thetitle}\\[\baselineskip]
  {\large\itshape \theauthor, \thedate}\\[\baselineskip]
  \vfill
  {\itshape 
  Nyomtatva 2015 februárjában.}
  \vspace*{0.1\textheight}
\end{titlingpage}

\vspace*{0.4\textheight}
Filep és szülei egy kis házban laktak, amit szülei együtt építettek. Házuk egy
fiatal falevél legvégében volt. A levél hegye mozog legerősebben, ha a szél
fúj, így Filepet édesanyja minden nap kérte, hogy óvatos legyen. Csendes
estéken viszont a levél peremén ülhettek együtt és lógathatták lábukat a
végtelen mélységbe.

Levelek ezrei terültek el alattuk.

-- Minden falevél olyan, mint a miénk? -- kérdezte édesapját Filep.

-- Ezen a gallyon igen. De az alattunk növő ágon más nyelven beszélnek a manók.
Felettünk pedig a levelek aljára fonják a lakásaikat, mert a levelek felszíne
sokkal csúszósabb ott és hidegebb van. Minden faág más. Van, ahol még sosem
jártak manók.

\secbreak

Filep felfedezőnek állt már másnap. Batyujába tett egy tucat pollent, két
összehajtogatott muslincaszárnyat és egy üres útinaplót, és útnak indult a
levél szára felé.

A levelet, ahol Filep lakott Cseppcsurrannak hívták. Napkeltekor mindig
harmatgyöngyök nőttek a házak tetején. Amikor akkorák lettek, mint egy pollen,
lebukfenceztek az ereszre, ott összegyűltek és manófejnyi gömbökben lehuppantak
a levél erezetére. A levél erezete volt Cseppcsurran úthálózata és a harmat
gömbjei minden reggel százával gurultak végig rajta. A gyerekek labdáztak
velük, a felnőttek pedig bosszankodtak, ha felborították a ház előtti padot.
Mire a nap egészében a láthatár fölé emelkedett, a vízcseppek mind a levél
közepére sereglettek és egy óriási gömbbé álltak össze. A gömb szikrázott az új
nap fényében és a levél minden lakosa útjába ejtette Cseppcsurran főterét, hogy
még idejében merítsen a friss harmatvízből.

Filep is a főtér felé ment. Amikor a főutcára ért, a sarkon épp nyitott a
kitinbolt. Vágyakozva szemlélte a kínálatot. A fényes pengék, vékony tűk és
lemezek között ma egy ragyogó pikkelyekből készült páncélt is látott a
kirakatban. Ha vadászni menne, épp ilyenre lenne szüksége -- gondolta Filep,
-- de felfedezőként csak akadályozná a mozgásban a kemény felszerelés.

-- Állítsd meg! -- kiáltotta valaki jobbról. Ahogy Filep a kétségbeesett hang
felé fordult, nyomban orrba találta egy cipónyi harmatgömb. A tovább pattanó
csepp után kapott, de elvesztette az egyensúlyát és hanyatt esett a falevélen.

Mária segítette fel futtában és már együtt kergették a vízcseppet.

-- Mamusz a cseppben van! -- kiáltotta Mária és a cseppre vetette magát. El
is érte, de nem tudta két oldalról megragadni és a gömb kirúgta magát a
kezéből.  Most már Filep is látta, hogy miért üldözik. Máriáék papucsállatkája
csapdába esett a vízgömbben. A papucsállatkák jó úszók, de ha a kis csepp
egyesül a nagy cseppel a főtéren, Mamusz az egész napot ott kell, hogy töltse.

A csepp sebesen siklott a főutca sima erezetén. Filep jó futó volt és lépést
tudott tartani vele, de tudta, hogy nem lesz egyszerű megállítani a gömböt.
Nincs csúszósabb a vízcseppek felületénél, és ha nem tudják beszorítani
valahova, Mamuszt sem tudják kihúzni belőle.

-- A Foltra kell terelnünk! -- kiáltotta Filep és belökte a cseppet egy
mellékutcába. Ennek a hangulata semmiben nem hasonlított a főutcára. Nem voltak
hívogató kirakatok, nem szűrődött friss illat éttermek teraszáról és nem
töltötte be a főtér házak fölé magasodó cseppjének ragyogása. Az életteli
hangzavar egyre halkult mögöttük, és ahogy a szűk utcácskán kergették tovább a
harmatcseppet, megcsapta orrukat a Folt illata.

Egyszer csak elhagyták az utolsó házakat és egy végtelen barna sivatagban
találták magukat. Lábuk alatt hangosan recsegett és ropogott az elszáradt
levélfelület, de nem akarták most elveszíteni Mamuszt. A csepp lassan
zsugorodni kezdett. A száraz, halott felület folyamatosan szívta el belőle a
nedvességet.

Mária lába alatt egy lépésnél beszakadt egy elszáradt lemezke. Elesett és ahol
földet ért még több lemezke tört be, körberajzolva teste alakját. Filep a nagy
reccsenésre megfordult és aggódva lépett Mária felé.

-- Fuss tovább, Mamusz begurul a Folt közepére! -- mondta Mária.

-- Nem futhatunk tovább. Itt már teljesen halott a levél, nagyon óvatosnak
kell lennünk. -- mondta Filep és miután felsegítette Máriát, figyelmesebben
mentek tovább. Még látták Mamuszt és azt is látták, hogy a csepp szinte
teljesen elfogyott. Amint a harmat nem fedte el, Mamusz bukfencezett párat és
megállt.  Szédelegve hasára fordult és körbeszimatolt. Szőrét alaposan megrázta
és Máriáék felé intett orrával.

-- Ne mozduljatok! -- dörrent rájuk egy hang a távolból. A két kismanó és a
papucsállatka ijedten néztek a Folt közepe felé. A távolból egy hernyó feje
nézett vissza rájuk.

\secbreak

Minden manógyerek ismerte a Folt történetét. Hónapokkal ezelőtt Balcsutak, a
hernyó megtámadta Cseppcsurrant. Házakat ledöntve a főtérre rontott. A
manóharcosok szembeszálltak vele, de kemény koponyája megvédte. A főtérről a
város legszebb része felé indult a hernyó. Átgázolt a gondozott parkokon,
átmart a boltok cellulózablakain és egy falatra lenyelt nektárral teli
hordókat. Kemény fejével átszakította a királyi palota selyemfalait és a
kincstár felé húzta puha testét.

A palota népe menekült amerre látott, de György, Cseppcsurran királya a
trónteremben várta Balcsutakot.

-- Ez a levél a manóké. Ha szeretnéd élve elhagyni, fordulj vissza most és
menj, amerről jöttél -- mondta György és a falról leemelte a királyi pallost.
Ezt a fegyvert a fa nagyobb lakói ellen készítették és csak a legerősebb manók
tudták megforgatni. Ragyogó pengéje sokszorosa volt György magasságának, hegye
majdnem a palota mennyezetét érte.

-- Majd elhagyom -- sziszegte a hernyó, -- ha jól laktam, bebábozódtam és
szárnyaim nőttek!

Ezzel meglendítette fejét és György felé csípett rágóival. A király maga elé
emelte a pallost, hogy hárítsa a támadást. A kitin hangosan csattant a fémen és
szikrák zúdultak a palota padlójára. A hernyó ereje a terem vége felé lökte
Györgyöt, de sikerült neki az egyik oszlopban megkapaszkodva megváltoztatni az
irányát. Balcsutak oldalába került, lesújtott a pallossal és megszabadította a
hernyót a fejétől.

A céltalanul tekergő testet elkerülve György a hatalmas fej szemébe nézett. A
rövid harc dühe még az ereiben izzott. Már nem csak azért haragudott
Balcsutakra, amit a hernyó elpusztított, hanem azért is, amit magának kellett
elpusztítania miatta. A haldokló hernyóból nem lesz már pillangó. Ezzel a
gondolattal hatalmába kerítette a királyt a bűntudat. Izzadt tenyeréből
kicsúszott a pallos markolata és a palota padlóján koppant.

Mint minden manóházban, a palotában is a falevél felszíne volt a padló. A
király tróntermében a növényi sejteket borító vékony viaszrétegbe festett
pókfonállal mintákat varrtak a szorgos manókezek. Egy arany fonállal megrajzolt
hangyasereg útjában feküdt a súlyos pallos. György tekintete a hangyák sorát
követve ismét Balcsutakhoz ért.

-- Ne! -- kiáltotta György. De ha lett is volna egy cseppnyi jóakarat a
hernyóban, akkor sem volt már benne annyi élet, hogy hallgasson a királyra.
Rágói a mintás falevélbe mélyedtek, és keserű mérgét a sebbe csorgatta. A levél
elfeketedett a seb körül és a sötét folt növekedni kezdett.

György elesetten lépett hátra. Szemei előtt töltötte be a fekete méreg a palota
padlóját. Tehetetlenül figyelte egy ablakon át, ahogy a palota körüli utcákat
is egytől-egyig megfesti a feketeség, és úgy érezte, mintha a hernyó a szívét
marta volna meg és onnan áradna szét a keserűség a testében. Elfeketedtek
szemei előtt az utcák, a házak és az ég is.

\secbreak

Sok nap telt el azóta, és most meleg teával kínálta vendégeit a volt király.
Mária és Filep csendben kortyoltak a gőzölgő csészékből és a kerek ablakon át
nézték a pusztaságot. György kiáltása állította meg őket, amikor Mamusz után
indultak és majdnem ráléptek a veszélyes, törékeny területre. Itt állt hetekkel
ezelőtt a palota és itt vívta csatáját Balcsutak és György. Itt fecskendezte
mérgét a levélbe a hernyó és akik elmenekültek előle, nem jöttek vissza a
fekete utcákba. A méreg szerencsére nem volt elég, hogy az egész levelet
megölje, de a Folt körzetében elszáradt a levél és összedőltek a házak.

A hernyó teste kiszáradt és elvitte a szél, de a feje kapaszkodott a halott
levélbe. A palotát is elvitte a szél, de a király sem tudta hátrahagyni a
Foltot. Berendezkedett hát Balcsutak kiszáradt fejében, és most az ő szemein át
nézték a láthatárig terülő pusztát.

-- Hogy lehetne meggyógyítani a Foltot? -- kérdezte az örökké kíváncsi Filep.

-- Nincs ilyen varázslat sajnos -- válaszolt az öreg király. -- Majd a
tavasz mindent meggyógyít.

-- Már nagy gyerekek vagyunk -- szólt közbe Mária. -- Tudjuk, hogy a tavasz
nem is igazi. Csak a szüleink díszítették fel a házat reggelenként.

-- Ó, hát tényleg nagyra nőttetek! Filep lehet, hogy holnap már a királyi
pallost is megforgatja, csakugyan! No de biztos észrevettétek, hogy most
rövidebbek a napok, mint mikor kicsik voltatok. Kevesebbet játszhattok, és
többet kell tanulnotok.

A két fiatal bánatosan bólogatott.

-- Idővel elfogynak a nappalok, elszáradnak a falevelek és elbújnak a manók. A
ti dolgotok lesz, hogy amikor eljön a tavasz, előbújjatok és mindent
újraépítsetek. Ahogy építitek a városotokat, úgy nő majd a leveletek és
reggelenként eljátsszátok majd a tavaszt a saját kismanóitoknak.

Filep és Mária figyelmesen hallgatták György jóslatát. A felnőttek mindig az
évszakokról meséltek nekik, de egyelőre nem tudták eldönteni, hogy valóban
léteznek-e vagy sem.

Elfogadtak még egy teát Györgytől és meghallgattak egy történetet arról, hogyan
vette meg annak idején a hangyáktól hét tetűért azt a levelet, amin
Cseppcsurran épült. Elköszöntek és a király megtanította nekik, hogy jutnak át
biztonságban a Folt repedezett felületén.

-- Köszönöm a segítséget! -- mosolygott Filepre útközben a manólány. --
Mamusznak nem kell a cseppben raboskodnia hála nekünk!

-- Szívesen segítettem.

-- Ha akarsz még segíteni, vigyünk egy kis vizet a gombácskámnak.

\secbreak

A főtéren színes ruhákban kavarogtak a manók. Legtöbben vízért jöttek a
hatalmas harmatcsepphez, de sokan megálltak beszélgetni. Árusok kínálták
portékájukat és öreg manók melegedtek a csepp túloldalán, ahol a nagy lencse
összegyűjtötte a felkelő nap fényét.

Mária épp olyan ügyesen formált kapillárist a kezéből, mint a felnőttek. A
csepp felszínéhez tapasztotta markát és Filep gyorsan batyujába kapkodta a
kibuggyanó harmatgombócokat. Egy el akart gurulni, de Mamusz rögtön
felhörpintette.

Mária egy kicsike gombát nevelgetett a szobájában. Nagynénjétől kapta, és
tiszta szívből hitte, hogy mézédes húsa lesz, ha csak még egy kicsit gondozza.
Ebben csak az hihetett, aki sosem kóstolta még meg, gondolta Filep. A fiú egy
kis nyúlványát megízlelte, amikor vendégségben voltak Máriáéknál, de borzasztó
rágós és keserű volt. Azóta inkább kerülte a furcsa alakú csírát és most is
fontos felfedezői munkájára hivatkozva elköszönt a manólánytól.

\secbreak

Filep a levél szárához érkezett. Még egyszer visszapillantott a manóépületek
között ragyogó cseppre és érezte, most válik belőle felfedező. A szárra ugrott
és fenekén csúszva indult az ismeretlenbe. Eszébe jutottak nagymamája
történetei a földről. A manóanyó azt mesélte, még a hangyák is félnek odáig
lemászni, mert akkora szörnyek járják, hogy még őket is egyben lenyelik.
Szerencsére ezek a teremtmények túl nagyok ahhoz, hogy a faleveleken járjanak,
így a manóknak nincs félnivalójuk. De mi a helyzet, ha egy kismanó elhagyja a
falevelet?

A levél egy fiatal gallyon nőtt. Szára tövében nagy zöld hurkák húzódtak, hogy
a levél lengését csillapítsák. Filep érkezését is ezek tompították. Bár nem
volt választása a kérdésben, úgy döntött, három kis koppanás jobb volt, mint
egy nagy. Nagymamája szörnyeitől tartva rögtön a gally egy manóméretű
repedésébe bújt és onnan leselkedett. Nem látott szörnyeket se jobbra, se
balra.

A gally olyan vastag volt, mint egy egész háztömb a levélen és mindkét irányban
a láthatárig nyúlt. Felülete sokkal egyenetlenebb volt, mint a levélé. Egy manó
mindig találhatott itt egy rést, amibe gyorsan bevetheti magát. Ez kicsit
megnyugtatta Filepet. Előbújt a keskeny repedésből, körbejárta a levél szárát
és átgondolta, hogyan fog visszamászni a felfedezőút végeztével. A szárat erős
viaszréteg borította, amin jól csúszott a nadrág, de a kicsi manóujjak
belemélyedtek és jó fogást találtak.

Az is jól látszott, merre dől a szár, és Filep ebbe az irányba indult.
Kíváncsian nézett felfelé. Először látta a levél alját, aminek túloldalán
eddigi életét töltötte. A levél erezetét jól lehetett látni, és a Folt barna
köre is tisztán kirajzolódott. Filep előcsomagolta útinaplóját és lemásolta a
levél rajzolatát. Aligha látott még szebb térképet Cseppcsurranról!

Tovább sétálva az ágacska görbülete felett ingó fekete rovarcsápokat pillantott
meg a távolban. Cseppcsurranon nem kellett félnie a rovaroktól, de Filep nem
tudta, mire számítson a falevélen kívüli vadonban. Egy egyenetlenségbe húzódva
szemlélte a közeledő csápokat. A csápok egy nagy fekete hangyafej tetején
billegtek és egy hangyatest követte a fejet. Azt pedig kicsivel később még egy
hangyafej és még egy hangyatest. Három hangya közeledett Filephez.

A hangyák rendszeres vendégek voltak Cseppcsurranon is, bár a manóknak nem
mindig sikerült kiigazodni rajtuk. Hatalmas testüket fényes kitin borította,
keményebb talán még a királyi pallosnál is. Lépteik alatt megremegett a levél,
és rágóikkal a főutca összes házát egyszerre fel tudták volna emelni. Mégsem
okoztak gondot sosem. Minden nap eljött egy hangya, egy szó nélkül körbejárta a
levelet, megkóstolta a harmatcseppet és visszament ahonnan jött. Filep szülei
azt mondták, a hangyák szemetet keresnek, és ezért kell mindent tisztán
tartani. Mária azt hallotta, rossz gyerekeket keresnek, de Filepet mégse vitték
el, amikor rossz volt.

Filep úgy érezte, aznap nem volt rossz, sőt jó volt. Bátran a hangyák elé állt
és derékra tett kézzel várta őket. A hangyák messziről is nagynak tűntek, de
ahogy közeledtek csak nőttek és nőttek. Filep elbújhatott volna egy lábuk
mögött.

A hangyák vagy nem vették észre Filepet, vagy nagyon siettek valahova. A háznyi
fejek, torok és potrohok magasan a manófiú feje felett vonultak el. Meglepetten
forgott alattuk és hiába integetett és kiáltott, nem álltak meg.

-- Várjatok, én is arra megyek, menjünk együtt! -- kiáltott utánuk, de a
hangyák nem vártak. Filep utánuk iramodott és rávetette magát az egyik óriási
lábszárra. Szorosan ölelte, ahogy a láb a levegőbe emelkedett. -- Várj, állj
meg, le fogok esni! -- kérte a hangyát, de vagy süket volt vagy nem beszélte a
manók nyelvét. Filep persze ügyesen tudott mászni, és nagy nehezen sikerült
felküzdenie magát a hangya térdéig. -- Hova siettek ennyire? Talán a tél elől
rohantok?

Filep végig egyensúlyozott a hangya combján, átvágott a tort borító
szőrrengetegen és kis bukdácsolás után megérkezett a hangya fejére. -- Így ni,
-- mondta, és két kapaszkodásra termett szőrszál között letelepedett. -- Most
már mehetünk tovább.

-- Filep vagyok, egy felfedező. Te biztos sokfelé jártál már, Hangya bácsi.
Jártál a fa tetején is? Biztos nagyon messzire ellátni. Talán még a leghosszabb
ág végén is túlra. Bár nem tudom, ti hangyák mennyire láttok jól. Lehet, hogy
nem is láttok engem? Nem is hallotok. Még ha énekelek sem?

\begin{verse}

Mi az, málna a kezedben? \\*
\vin Megérett az alga! \\*
Hol találtad? Honnan szedted? \\*
\vin Egy korty vizet inna! \\
Ha megitta, mi lesz véle? \\*
\vin Elmegy a világ végére! \\*
Onnan ír majd levelet, \\*
\vin Köszöni a vizedet.

Mi az, pinty a házam felett? \\*
\vin Visszajött az alga! \\*
Lehet, hogy csak lepke repdes? \\*
\vin Jó ebédet adna! \\
Asztalom megterítettem, \\*
\vin Kövér algámat megettem. \\*
Ültesd el a spóráját, \\*
\vin Ablakom épp jól rálát.

\end{verse}

A dal végén az utolsó sor dallama lezárásként még egyszer megismétlődött mély
basszus hangon. A Filep alatti hangya dúdolta el újra.

-- Ó, hát mégis hallasz! Hogy hívnak téged? Hová megyünk? Miért nem szóltál
semmit?

-- Igen, hallak, Filep. Balázs vagyok. A birodalom határait ellenőrizzük. Ha
valaki betolakodott, valami elromlott vagy találunk valami értékeset, jelezzük
a többieknek. Fontos munkánk van, nem érünk rá beszélgetni és énekelni.

-- Értem -- mondta Filep. -- Kérdezhetek valamit? Mekkora a birodalom? Az
egész fa a ti birodalmatok?

-- Nem -- válaszolt Balázs. -- Innen lefelé hat ág és felfelé négy ág a
birodalmunk. A fának nincs vége és egy birodalomnak kell, hogy legyen határa,
különben nem lehetne megvédeni.

-- De igen, minden ágnak van vége és a törzs is egyre vékonyodik és a fa
csúcsán véget ér és csak a kék ég van felette. Ha végtelen sok ág lenne, ha
felnézünk, nem láthatnánk az eget a végtelen sok levéltől és a napfény sosem
érne el minket és örök sötétség lenne -- magyarázta Filep. -- Hát még ezt sem
tanultad meg az iskolában?

-- Hangyák nem járnak iskolába. Mi mindent tudunk, amint kikelünk. Nincs
szükségünk tanulásra.

-- Bár én is hangya lehetnék! De mi van, ha valaki kitalál valami újat? Azt is
rögtön tudni fogod? Vagy csak azok fogják tudni, akik utána kelnek ki?

-- Egy jó hangya nem talál ki semmi újat. A rossz hangyákat elvisszük a
birodalom határára és nem jöhetnek haza többet.

Filepnek ez nagyon nem tetszett, mert az egész felfedezésben az volt a jó, hogy
új dolgokat tanulhatott. De inkább hallgatott, mintsem hogy megharagítsa
útitársait. Különösen így, hogy egyben szélsebes hátasa is volt Balázs. Inkább
kényelmesen hátradőlt és beitta a magasban ringó levelek között áttörő
napsütést. Vajon mit csinálnak a kitaszított hangyafeltalálók? Biztos
szomorúak, hogy nem jöhetnek haza, de elmehetnek felfedezni. És ha feltaláltak
valami ügyeset, biztos még jobban is boldogulnak, mint a birodalom hangyái.
Lehet, hogy csodálatos városokat építenek a távoli falevelek között. Lehet,
hogy egész falevelekből fonnak maguknak palotát a fa legtetején.

\secbreak

Filep arra ébredt, hogy valami lökdösi. Olyasmi forma volt, mint Mamusz, de
szőrtelen és kemény. Ahogy kinyitotta szemét, négy hosszú fekete csáp
integetett felette. Felülve látta, hogy még mindig Balázs fején lovagol, de egy
másik hangyával állnak szemben és az ő csápjai ébresztették fel Filepet.

-- Hol vagyunk? Miért álltunk meg? Egy kicsit elaludtam -- mondta a manófiú.

-- Véget ért az ellenőrzőútunk. Ez a birodalom határa. Mindent rendben
találtunk és visszamegyünk a bolyba -- válaszolt Balázs.

Filep kíváncsi volt a hangyabolyra is, de mivel a hangyák nem hívták, nem akart
tolakodó lenni. Inkább felnézett a fa csúcsa felé és vett egy nagy lélegzetet.
A törzs a végtelenbe látszott nyúlni és kicsit aggasztotta Filepet, amit a
hangyák mondtak. De eltökélte magát, hogy felmászik a fa tetejére és megmutatja
a hangyáknak, hogy nem is tudnak mindent.

Filep megköszönte, hogy idáig elhozták a hangyák, és elbúcsúzott tőlük.
Megtudta azt is, hogy a hangyák birodalma felett egy harkálycsalád birodalma
kezdődik. Ez rovaroknak borzasztóan veszélyes, de Filep tudta, hogy a madarak
olyan hatalmas nagyok, hogy nem látják a manókat. Sokkal inkább tartott az
amőbáktól és más harapós egysejtűektől, amik a törzs egyenetlen kérgében
lappangtak.

A vastag kéreg megkönnyítette a mászást. Filepnek nem is kellett a kezét
használnia, akár ha egy lépcsőházban ment volna felfelé. A kéreg sötét zugaiban
itt-ott megmaradt nedvesség és Filep gondosan elkerülte őket. Mamusz vad
rokonai éltek ezekben a vízcseppekben és biztosan a manó bokájába kaptak volna,
ha a vízbe dugja a lábát.

Az egyik árnyék mélyén viszont valami fényesen akadt meg Filep szeme. Nem víz
csillant és nem is kitin. Ahogy Filep megkereste a kapaszkodókat és járható
felületeket a kis barlang felé, a fényesség a levegőben látszott lebegni.
Alakja egy manó alkarjához hasonlított és felülete meg-megcsillant a barlang
sötétjében forogva. Filep le se tudta venni róla a szemét és szokásos
óvatosságát maga mögött hagyva egyre közelített a különös tárgy felé.

A kéreg alatti apró űrben szinte teljes volt a sötétség. A titokzatos tárgy
síkjain a Filep mögötti világosabb világ tükröződött időnként. Filep a tárgy
felé nyúlt és saját kezét látta eltorzítva a kristály felszínén. Elvette és
kezében forgatta. Mitől lebegett a levegőben? Hogy lehet ilyen fényes? Miért
ilyen különös az alakja? A napfényben jobban meg lehet nézni\dots

Ahogy Filep megfordult, egy sötét lényt talált maga előtt. A barlangból
kivezető utat elállta egy csupaláb, csupaszem pókfióka. A felnőtt pókok még a
hangyáknál is nagyobbak, de a kis pók még egy manónál is alacsonyabb volt.

-- Tolvaj! -- sziszegte és Filep felé közelített. -- El akartad csenni a
kincsemet. De tetten értelek. És le kell sújtsak rád a törvény szigorával.

A pók fenyegetően felemelte négy lábát és szélesen széttárta őket. De Filepet
nem lehetett ilyen könnyen megijeszteni. A pókra vetette magát és birkózni
kezdtek. Közelharcban a nyolc hosszú láb nem vált a pók hasznára. Filep
lendülete hanyatt döntötte és hiába kalimpált velük. Rágóit csattogtatta, de
Filep kezei fejtorát a földre szegezték. Potrohából fonalat fonva próbálta a
vadul rátámadó manófiút megkötözni, de ezzel sem volt szerencséje. Ahogy egyre
több fonál tekeredett a birkózókra, egyre lassult a harc. Nem sokára a pók hét
lábát Filep jobb bokájára csomózta a fonalrengeteg. A nyolcadik láb végét Filep
markolta. A pók megszeppenve nézett fel a manóra.

-- Csak vicceltem -- mondta a legyőzött nyolcszemű. -- Tiéd lehet a látókő.

-- Köszönöm -- mondta Filep, és elkapta a pókfonálon lengő kristályt. -- Én
Filep vagyok, a felfedező. Téged hogy hívnak?

-- Gergő, a ravasz -- mutatkozott be büszkén a pókfióka és lábával megrázta
Filep kezét. -- A csapdaállítást gyakorolom. Te vagy a legnagyobb zsákmányom
eddig.

Filep nem akarta elkedvetleníteni a kis pókot és a csapda sikerességéről más
témára terelte a beszélgetést.

-- Mit tud a látókő? Gyere, menjünk ki a napra, nézzük meg jobban -- mondta
Filep és a lábára erősített pókkal a barlang kijárata felé lépdelt. A pók
szabad lábával igyekezett hozzájárulni a gyalogláshoz.

-- A látókő egy nagyon kicsi homokszem. A görbe felületei különös módon
torzítják, amit látsz. Ha belenézel, minden sárga, fejjel lefelé van, és sokkal
közelebb látszik, mint igazából -- magyarázta lelkesen az összecsomózott
pókfióka. Filep a szeméhez emelte és szétnézett a törzs kérgéről. Ha csak egy
kicsit is fordult, a látókőben minden szélsebesen mozdult. Ha viszont nagyon
óvatosan tartotta, a távoli falevelek úgy látszottak, mintha csak rajtuk járna.
Csak sárgában. És feje tetejére fordítva.

Filep nem győzött betelni vele. Egyik falevélről a másikra fordította a
látókövet és a távoli manókat, hangyákat, pókokat és papucsállatkákat
szemlélte. Felfelé nézve talált egy levelet, aminek alján tényleg manók éltek.
Házaikat a levélre tapasztották, tető helyett padlót kellett építeniük és a
házak között kifeszített fonalakon jártak. Az ágat tovább követve Filep egy
hatalmas madár lábujjait látta, majd az utolsó leveleket. Az ágon túl pedig egy
másik ágat. Egy másik fát. Itt már minden kisebbnek látszott, és Filep nem
tudta kivenni, hogy hogyan élnek a manók az erdő többi fáján. De nagyobb
állatokat ott is felfedezett a látókővel.

-- Nézd, Gergő, odaát is élnek pókok! Ismered őket? -- kérdezte új barátját.
De a kis pók helyén már csak egy üres fonalkupac hevert. A pókfióka elszökött,
míg Filep a látókövet csodálta. Nem találta a pókot sem a barlangban, sem annak
környékén. Óvatosan elcsomagolta a látókövet és a bokájáról leoldott
pókfonállal együtt hátizsákjába tette. Továbbindult felfelé, remélve, hogy
Gergő is arra ment, és találkoznak majd.

\secbreak

A kéreg könnyen mászható rostjait valamivel feljebb egy nagy lyuk szakította
meg. Egy kisebb levélnyi területen hiányzott a kéreg, és a fa testén is mély
hasadék tátongott. Persze egy felfedezőnek az a dolga, hogy kivizsgálja az
ilyesféle titokzatos helyeket, és a kis manó körbejárta a nyílást.

A hasadék mély volt, és Filep a harkályra gyanakodott. Ismerte a kopácsolás
hangját, és az iskolában elmagyarázták neki, hogy a harkály csőrével megkeresi
a fába alagutakat építő lárvákat. Hálásak voltak neki, amiért eltávolítja a fa
kártevőit, és azért is, hogy nem leveleket eszik. De más volt erről hallani és
egymagában ott állni a hatalmas bemélyedésben.

Gyanúja beigazolódott, amikor egy széles alagút szájára bukkant. Az alagút a
harkály fúrta lyuk falából indult jobbra és balra. Filep a jobboldali nyíláshoz
ért. Ez emelkedett, a szemközti alagút pedig ereszkedni látszott. Vajon
elcsípte a harkály a lárvát? Vagy sikerült az alagútba iszkolnia és tovább
bujkál a madár elől?

-- Halló -- kiáltott Filep az alagút hűvös homályába, de csak a visszhang
válaszolt. Nekilátott hát az alagút felderítésének.

A kezdeti enyhe lejtő után egyre meredekebb lett a járat. A függőleges részeken
ügyesen kellett Filepnek másznia, mert az alagút fala sokkal simább volt, mint
a fakéreg. Gyorsabban is tudott viszont haladni, mert nem kellett keresnie az
utat és egysejtűeket rejtő vízcseppektől sem kellett tartania.

Ahogy a harkály ütötte lyuktól távolodott, egyre sötétebb lett és olyan csend
vette körül, mint még soha. A járatot betöltötte egy állandó fuvallat, ami
szerencsére felfelé húzta Filepet, megkönnyítve a haladást. A kis manó tudta,
ez azt jelenti, egy kijárat várja az alagút végén.

Nem is volt teljes még a sötétség, amikor ismét világosodni kezdett a járat. A
gyenge fénynek sárgás színe volt, mintha csak a látókövön át szűrődne be az
alagútba. A fény egyre erősödött, míg Filep a forrásához nem ért. A fény a nap
fénye volt az alagút falán átszűrve. Az elvékonyodott fa lemezek átadták neki
sárgás színüket. A fénygyűrű közepén pedig át is lyukadt a fa, és az alagútban
áramló fuvallat ezen a manónyi kis nyíláson át távozott.

A sárgás fényben különös látvány bontakozott ki Filep előtt. A ragyogó kapuval
szemben egy hangya állt. Nem hat lábon, mint egy rendes hangya, hanem kettőn,
mint egy manó. A felső két pár lábát összekulcsolta a teste előtt, mintha
mozdulatlanul imádkozna a kiszélesedett alagútban. Kis szentélye falát különös
rajzolatok borították. A vonalakat a hangya karcolhatta a simára csiszolt fába.

Filep a szentély bejáratában ült le és merengve nézte a csendes hangyát. Nem
tudta megfejteni, mit ábrázolhatnak a vonalak, és kíváncsian remélte, a hangya
elmagyarázza majd, ha felébred. A hosszú mászásban meg is éhezett, és hogy
hasznosan teljen az idő, előcsomagolt két szép szem pollent a batyujából.
Persze a látókő is a kezébe akadt, de a hangya közelebb hozva csak végtelen,
aranyló kitinpáncélnak tűnt. A követ megfordítva minden messzire távolodott. A
kicsomagolt pollen elérhetetlennek tűnt, de Filep keze is megnyúlt, mint a
pókfonál, és szájához emelte a távoli falatot.

\secbreak

A hangyák nem pislognak. Szemüket is vékony kitin borítja, nincs szükségük
nedvességre. Így Filep nem tudhatta, figyeli-e a hangya, vagy alszik. De még be
sem fejezte az első pollent, amikor mozgásra lett figyelmes. A hangya széttárta
középső pár lábát, majd a felső párt és ebben az új pózban dermedt meg.

-- Kérsz pollent? -- kérdezte Filep, és a hangya felé nyújtotta a második
gombócot. -- Itt laksz? Száműztek a birodalomból? Mit találtál fel? Nagyon
szépek ezek a faragások. Ezért taszítottak ki?

A hangya nem válaszolt, de lassú mozdulattal elfogadta a csöppnyi falatot.
Komótosan szájához emelte és megette.

-- Ismersz más kitaszítottakat is? Mi mindent találtatok fel? Megtanítotok
engem is? -- Filep szeme előtt a fa tetején épült titkos hangyaváros képe
lebegett. Ez a legendába illő helység lett volna felfedezőútja koronája, és
mindennél jobban vágyott rá, hogy a hangya elismerje létét.

-- Az igazságot találtam meg. A fényt. Ez a ragyogás szentélye. A Nap minden
élet forrása -- válaszolt a hangya. -- A hangyák birodalma is elveszne a Nap
sugarai nélkül. Mégis eltaszítottak maguktól. Nem akarták megépíteni a Nap
templomát. Egyedül kell elvégeznem a munkát.

-- Én szívesen segítek! -- ajánlotta Filep.

-- A Nap a fa tetején vár minket. De én nem tudok már tovább menni. Kérlek,
vidd ezt a dalt a fa csúcsára.

\begin{verse}
\napdal
\end{verse}

Filep lejegyezte útinaplójába a dalt és elköszönt a hangyától.

\secbreak

A szentély kapuján átlépve Filepnek hunyorognia kellett. Magasan volt a fán,
sokkal közelebb a Naphoz, mint Cseppcsurran. Elégedettséggel töltötte el a
pollen és hogy az öreg hangya is hitt a fa csúcsában. Könnyű léptekkel
emelkedett célja felé.

A vidám lépteket egy hirtelen hanyattesés szakította félbe. Ahogy Filep
feltápászkodott, tenyere alatt furcsa csúszós formákat érzett. Körülötte
mindent sárga baktériumok borítottak. Csúszós nyálukkal összemászkálták a fa
kérgét, és orrukkal Filep cipőjét bökdösték. A manófiú a tekergő nudlikat
félrerugdosva visszahátrált a kéreg száraz részére. Felnézve látta, hogy a
baktériumok a felfelé vezető út jó részét beborították. A csúszós nyálkával
nyerték ki táplálékukat a kéregből. Filep nem érzett bőrén semmi csípőset,
tehát a nyálka nem volt veszélyes rá. Vízszintes talajon óvatosan átvágott
volna a hemzsegő baktériumok között. De az egyenetlen kérgen képtelenség volt
így továbbmászni.

Filep más útvonal után nézett. A törzs görbülete miatt nem tudta felmérni, a
baktériumtelep meddig terjed ki, de a hangya szentélyével egy szintben látott
egy vastag ágat. Erre kimászva már jobban fel tudta mérni a helyzetet. A
baktériumok nem véletlenül nőttek azon a területen. Felettük volt a harkályok
odúja, és az ő hulladékaikon éltek a csúszós parányok. Nem láthatta, hogy a
törzs másik oldalán meddig terjed ki a járhatatlan terület, de Filep inkább nem
akarta megkockáztatni az utat. A kis sárga baktériumok ártalmatlanok voltak, de
élhetett a területen veszélyesebb fajta is.

Gondolataiba merülve sétált tovább az ágon. A törzstől távolodva egyre jobb
képet kapott az ágak elrendeződéséről. Ha valahogy át tudna jutni egyik ágról
egy másikra, elkerülhetné a harkályok közelségét és folytathatná útját.

Filep ismét megbotlott és most előre bukfencezett. A levegőben egy megpendült
húr hangja remegett és még akkor sem ült el, mikor a manó talpra állt és az
észrevétlen akadályt kereste tekintetével. A hang és a rezgés nyomra vezette.
Egy átlátszó, manóujj vastagságú fonál volt az ághoz kifeszítve. A fonál ferdén
emelkedett a magasba, de Filep akárhogy meresztette a szemét, nem tudta a másik
végéig követni a finom szálat. Abban bízott, hogy talán ezen az úton eljuthat
egy magasabb faágra.

Gondolatban meghosszabbította szál vonalát és próbálta megtalálni az ágat,
amihez felső vége rögzítve lehet. A gondolatbeli vonalon azonban valami máson
akadt meg a szeme. Egy hatalmas fekete pók látszott a levegőben lebegni. A
finom fonálon szaladva jártak a lábai, és mielőtt Filep átgondolhatta volna,
hogy a pók közeledik-e vagy távolodik, már ott is volt felette.

-- Nohát, mi akadt a hálómba -- búgta a pók Filep felett tornyosulva. --
Csak nem egy ügyeskezű manó?

-- Bocsánat, nem vettem észre a hálót. Filep vagyok -- kezdte bemutatkozását.

-- A felfedező. Igen, tudom. Gergő játszótársa. Én a mamája vagyok.
Megkérdezzem, hogy le akar-e jönni játszani?

-- Köszönöm, de igazából én szeretnék felmenni. Hova vezet ez a fonál?
Elérhetem rajta a következő ágat?

-- Természetesen! Szívesen látunk, és ingyen használhatod a hálónkat. Olcsón
tudok térképet is adni, ha szükséged van rá -- ajánlotta sejtelmesen a pók.

Filep ismerte János történetét. A manólovag harcba indult egy gonosz pók ellen,
aki sanyargatta a levél lakóit. Hosszú vándorlás után megtalálta a pók hálóját
és mászni kezdett. A pók látta a félelmetes kardot és a háló másik végébe
menekült. János üldözőbe vette és ügyesen mászott a háló fonalain. De a pók
mindig elkerülte, és a háló egyre ragacsosabbá vált János alatt. Csak a pók
tudta, melyik fonál ragad és melyik nem.

A mesében János kardjával levágta a hálót és elkergette a pókot a fáról. De
Filepnek sem kardja nem volt, sem a hálót nem akarta levágni.

-- Mit kérnél a térképért cserébe? -- kérdezte.

-- Tudásért tudást -- válaszolta a pókmama. -- Taníts meg olyan szép kabátot
kötni, amilyen rajtad van. Van sok fonalam és télre szeretnék Gergőnek kötni
egy kabátot, de nem tudom, hogyan kell a szálat vezetni.

Így történt, hogy Filep tanította meg kötni a pókot.

\secbreak

Mikor a pók már ügyesen kötött, nekilátott egy egyszerű kendőnek. Háromféle
szálat váltogatva bonyolult mintát szőtt bele. Ahogy dolgozott, sorról sorra
bontakozott ki Filep előtt a pókháló térképe. Hét hosszú fonál pányvázta ki
különböző ágakhoz. Köztük számtalan sugárirányú szál feszült és ezeket sűrű
csigavonalban borították a ragadós fonalak. Hogy a pókok mindenhová
eljuthassanak beragadás nélkül, a háló egyes részein további járható fonalakat
szőttek a mintába.

A járható fonalak térképe nagyon hasznos volt, Filep figyelmét mégis más
kötötte le. A hét hosszú pányva végében a pók finom fonalvezetéssel különböző
képecskéket helyezett el. Filep nem ismerte fel mindet. Ahol a pókkal
találkozott, ott egy harkály kandikált ki odújából. Egy hasonló magasságban
lévő állomásnál egy levéltetű legelészett. Ugyanannak a fonálnak a másik
végében egy manó mosolygott. Tőle balra egy mókus rágcsált valamit a háló
felfüggesztésénél.

Ahogy készült a térkép és sorra jelentek meg a kis képek, Filep szeme tágra
nyílt. Ennyi felfedeznivalóról nem is álmodott. Egy nap sem lenne elég ezt a
sok rejtélyt mind megfejteni. El kellett hát döntse, melyik fonálon hagyja el
elsőként a hálót. Még mindig a fa csúcsa volt a kaland végcélja, így a magas
ágra rajzolt manóarchoz vezető fonalat választotta.

Filep nem volt pók, és nagy volt a távolság egyik ágtól a másikig. Bár Gergő
elkísérte egy szakaszon, a mászás végtelen hosszúnak tűnt. Lassacskán azonban
kezdte sejteni, hova vezet a fonál. Nem egy ág kérgére tapadt a vége, ahogy a
harkályok közelében, hanem egy levél aljára. És ahogy közelebb ért, látta, nem
is csak egy ponton volt rögzítve. A levélhez közelítve szétvált a fonál három
szálra, és a levél három különböző pontját célozta meg. Filep a levél csúcsához
legközelebb esőt választotta.

Mikor elég közel ért, hogy lássa a manók épületeit, feltárult előtte a
mozgalmas város. Ahogy a látókővel korábban felfedezte, a levél lakói alulra
építették házaikat. A házak alján díszes padlózatok tündököltek. Színes
korlátokkal kerített teraszokon és őket összekötő függőhidakon sétált a
település apraja és nagyja. Sok helyütt vékony fonalak lógtak egyik porta és
egy másik között közlekedés vagy üzenetküldés céljából. Filep a
teherszállításnak is érdekes módjával ismerkedhetett meg. Időről időre hosszú
selyemszálon lendültek át nagy csomagok a levél egyik végéből a másikba.
Félelmet nem ismerő manók álltak a lengő terheken és testsúlyukkal irányították
pontosan célba szállítmányukat.

A pókfonál, amin Filep mászott, egy nagy épület belsejébe vezetett. A szál
körüli kürtő három emeletet kötött össze egy légtérbe. Minden emeleten pallók
nyúltak ki megkönnyítve a fel- és leszállást. Filep a mászástól kimerülten az
első emelet pallójára lépett és máris a forgalmas dokképület belsejében volt.
Erős manók hada töltötte be az emeletet. Nagy bálákat cipeltek fejük felett és
hangosan beszélve irányították a rakodás menetét.

-- Hol a többi alga? -- kiáltott le valaki a második emeletről. -- Hé,
kismanó, kérdezd már meg, melyik raktárba vitték az algákat! Mindjárt tovább
kell küldenünk őket a központba.

Cseppcsurranon nem volt ilyen szervezett a kereskedelem, de ahhoz Filep is
hozzá volt szokva, hogy felnőttek között üzeneteket közvetítsen. Beszaladt a
rakodómunkások közé és megkérdezte hova vitték az algákat. A hármas raktárba
vitték.

-- A hármas raktárba vitték! -- kiáltott fel Filep az emeletre.

-- Miért nem a kettesbe? -- mérgelődött az emeleti manó. -- Végülis mindegy.
Szólj a tetőtérben a csörlősöknek, hogy kezdhetnek tekerni. Nem várhatunk
tovább.

Ebből sem értett sokat Filep, de a pallóról visszakapaszkodott a pókfonálra és
mászni kezdett. A fonal nem a levél felületén ért véget. Egy nagy csörlőre
tekeredett fel, vastagon beborítva a tengelyt. A tengelyhez egy nagy
kormánykerék volt rögzítve, melyet az emelet mindkét oldaláról több manó is
elérhetett.

Filep a várakozó munkásokhoz futott a pallón át, és átadta az üzenetet.
Egyszerre mindenki megelevenedett és buzgón tekerni kezdték a nagy kereket. A
pókselyem megfeszült és tovább csavarodott a csörlő köré. Közelről úgy
látszott, mintha a pók hálóját húznák közelebb magukhoz, de Filep érezte, nem a
háló mozog. Ahogy rövidítették a fonalat, maga a levél lendült mozgásba. A
csörlő minden fordulatával egy árnyalattal meggörbítették az egész levelet, az
egész várost.

Lenyűgözte a munkások hada és a szorgos kezekkel rakodott áruk folyama, és
Filep a fonaldokkban segédkezett egy ideig. A berakodás és kirakodás közötti
rövid szünetekben megismerkedett a helyi manókkal és a levél történetével.
Együttműködtek Gergő családjával és a pókfonal mentén kereskedelmi hálózatot
építettek ki. Gergőék bőségesen ellátták őket pókselyemmel és együtt dolgoztak
a csörlőrendszerek létrehozásán. Cserébe a manók különleges csemegékkel,
fűszerekkel, hírekkel és színes kristályokkal halmozták el a pókokat.

A műszak végén ebédre is meghívták Filepet a munkások. Itt előkerült a manók
legnagyobb büszkesége, a pókpálinka is. Ezzel Cseppcsurranon nem találkozott
Filep, és érdeklődését látva részletes magyarázatba kezdtek társai. Nyolclábúak
és kétlábúak ugyanolyan jó kedvvel fogyasztották ezt a párlatot. Az aranyló
cseppek hajnalban csapódtak le a levél alján és a manóházak padlószerkezetén.
Manócsapatok biztosítófonalakkal és pókokhoz illő ügyességgel gyűjtötték az
apró cseppeket nagy hordókba. Innen kerültek elő a bódító kortyok, amikor csak
összegyűltek és mulattak a manók.

Legféltettebb titkuk az arany cseppek eredete volt. Meleg levegőben oldva
érkezett a nedű a levélhez és a hideg felület kényszerítette folyékony formába.
A levegőt az új nap fénye melegítette meg, és az alattuk lévő ág egy leveléről
emelkedett fel. Ezen a levélen szelíd levéltetveket legeltettek a manók. A
tetvek mézharmatot adtak és a manók egy gombát használva megerjesztették az
édes gyöngyök egy részét. A felkelő nap melege ezekből az erjedt cseppekből
párologtatta el a szeszt, ami aztán a fenti város hordóiban kötött ki.

Filep nem kóstolta, de azt mondták, olyan íze van, mint a napfénynek. Egy korty
után úgy érzi a manó, mintha a nyári napon sütkérezne. Erről az italról kapta
nevét a városuk is, Szolperla. Valamilyen nyelven ez azt jelenti, ``a nap
gyöngyei'', állították a helybéliek.

A párlat nem csak gömbölyű volt és sárga, de egy szikrával meg is lehetett
gyújtani. Ilyenkor világított és melegített, és a lángba tett kezet meg is
égette. A tűz feltalálásáról egy kerek fekete folt emlékezett meg a levélen,
hasonlóan Cseppcsurran száraz Foltjához. De azóta kitanulták a tűz használatát
a manók, és éjszaka mindig egy cseppnyi láng lobogott a város alatt. Fénye
segített hazatalálni a vándoroknak és sok óvatlan rovart a pókháló csapdájába
csalt. A pókok eleinte nem bíztak a veszélyes jelzőfényben, de a gazdag
zsákmány hamarosan megváltoztatta a véleményüket.

Filep megköszönte az ebédet és a sok érdekes történetet, majd Szolperla
felderítésére indult. A dokk forgataga után jól esett a város hídjain sétálni
és az alatta elterülő leveleket nézni. Tudta, az egyik levél Cseppcsurran kell,
hogy legyen, de még a látókő segítségével sem tudta megtalálni, honnan indult
reggel útjára.

\secbreak

Nem sokára Filep azon kapta magát, hogy Szolperla gyermekseregével fogócskázik.
A játék több ügyességet követelt, mint ahogy Cseppcsurranon játszották. A
legkisebb manók is ismertek minden hidat és minden kósza fonalat. Kifeszített
selyemszálakon egyensúlyoztak át, teraszokról lelógó rojtokon lendültek, és
akárhányszor úgy tűnt, a mélységbe vetik magukat, mindig megkapaszkodtak végül
egy kiálló peremen vagy lelógó szöveten.

Filep még a függőhidak kuszaságán is alig igazodott ki, így sokszor jutott rá a
fogó szerepe. Egy magánál idősebb játszótársát kergette épp, és bár a fiú
váratlan rövidítésekkel egyre nagyobb előnyhöz jutott, Filep nem adta fel az
üldözést. Közeledtek a levél szárához, és tudta, hogy ott véget ér Szolperla
labirintusa. A fiúk előtt elfogyott az imbolygó hidak, teraszok és a közéjük
szőtt pókfonalak hálója. Nem volt tovább hova futni, és Filep egy jó vetődéssel
átadta a fogó szerepét.

Persze sarkukban volt a helybéli gyerekek egész hada, és most, hogy új fogó
iramodott feléjük, visongva rebbentek szét. Filep egyedül maradt Szolperla
legszélén. Még fülében dübörgött a kergetőzés vad tempója, de tekintete a levél
szárához fordult és szíve is megállapodott. Ha vacsora előtt fel akart érni a
fa tetejére és vissza Cseppcsurranra, nem tölthetett több időt játékkal.

A törzs egyre vékonyodott és a vízszintes ágak egyre rövidültek. A fa kérge is
fiatalabb és simább lett. Egy manó apró keze persze bárhol könnyen talál
fogást. A simább és újabb felület legalább azt jelentette, hogy nem kellett
sötét üregekben lappangó egysejtűektől tartania. Minden ág és gally után
kevesebb levél volt Filep felett és végül elfogyott a fatörzs. A törzs végén
két levélszár nőtt és Filep mászni kezdett azon, amelyik a magasabb levélhez
vezetett.

A levélen alulról semmi különös nem látszott, de Filep érezte, hogy valami
izgalmas kell, hogy várja a másik oldalán. Az nem lehet, hogy az egész fa
legmagasabb levelén ne találjon semmi különöset. Vajon a kitaszított hangyák
varázslatos városa nyílik meg előtte? Vagy maga a Nap lakik idefent egy izzó
palotában? Udvarában manók forgataga, akik különös ruhákban sétálnak fontos
papírokkal a kezükben és nincs idejük Filepre?

Mikor elérte a levél szintjét, visszaszámolt háromtól, majd mindenre
felkészülve felugrott a színére.

\secbreak

A fa legfelső levele ferdébb volt, mint a többi és Filep kis híján
visszabukfencezett róla. Visszanyerte egyensúlyát, de ezen kívül semmi
izgalmasat nem látott. A levél makulátlan zöld volt, de teljesen kopár. A
középső erén végigsétálva akár Cseppcsurran főutcáján is lehetett volna kora
tavasszal, még mielőtt a manók odaköltöztek.

A hosszú mászás után jól esett kinyújtózni a puha meleg felületen. Jól esett
rajta tótágast állni is és lebukfencezni a főutcán. És körbe-körbe szaladgálni
is, fél kört kaptatón, fél kört ereszkedőn megtéve. Jól esett felmászni a levél
csúcsába, ahol édesapja és édesanyja várta volna, ha ez a levél Cseppcsurran
lett volna. És végül jól esett a fa legmagasabb pontjáról szétnézni az egész
világon.

Saját szobája ablakából is látott távoli leveleket és tudta, a legtávolabbiak
egy másik fa ágain nőttek. Tudta, hogy a madarak átrepülnek egyik ágról a
másikra és fáról fára is átszállnak. De ez nem készítette fel az erdő
látványára. Filep fája egy nagyobb domb oldalában állt, remek kilátással a
hullámzó tájra. Fák végtelen sokasága borította a dombokat. Több fa volt az
erdőben, mint levél egy fán. A fa véges volt, de az erdő határtalannak
látszott.

És minden fa különböző volt. Egyik magasabb volt, másik alacsonyabb. Volt
amelyiknek bogyói voltak, volt aminek tobozai, de még a leveleik sem voltak
egyformák. Filep csak ámult és próbálta elképzelni, milyen az élet a különös
formájú leveleken. Volt, ami kerek volt, és volt, aminek három vagy öt csúcsa
is volt. Sok falevél nagyobb volt Cseppcsurrannál. De sok levél kevésbé volt
zöld. Más fákon mintha már beköszöntött volna az ősz. Sárga, vörös és barna
színekben pompáztak a domboldalak, akár ha átragadt volna rájuk a lenyugvó Nap
fénye. A Nap felhőkön átragyogó korongja Filep eszébe juttatta az öreg hangya
dalát.

\begin{verse}
\napdal
\end{verse}

A dal hallatán a Nap is előbújt a felhők közül, és fénye aranyba borította a
dombokat. Egy pacsirta rebbent fel valamelyik közeli fáról, és éles
csiviteléssel hívta dalra és játékra társait. Volt, aki csatlakozott hozzá, és
az égben kergetőztek, volt, aki dalával kísérte őket. De hirtelen mind
eltűntek.

Filep a szeméhez emelte a kezét, és az ujjai között nézett a távoli dombok
felé. Új nap bújt elő a láthatár mögül. Ingatag fényében mintha lángba borultak
volna az őszi színekbe öltözött lejtők, és sötét árnyakká váltak a felhők. A
madarak dala elült, az erdő minden lénye a félelmetes látványt figyelte.

A vakító fény lassan emelkedett, és nem sokára mély moraj töltötte be a tájat.
Filep a levélre lapulva figyelte a különös jelenséget. A dübörgő hang egyre
erősödött, és Filep rémülten látta, hogy az erdőn egy hullám söpör végig. A fák
meghajoltak az útjában, az elsárgult levelek magasra szálltak róluk és madarak
hada röppent fel riadtan. A hullám az új nap felől gördült végig a tájon, és
megállíthatatlanul közeledett Filep felé.

Filep nem tudta pontosan, mi történik, de ösztönei megelégelték a higgadt
megfigyelést. Gondolkodás nélkül felpattant és rohanni kezdett. El a
legmagasabb falevél hegyétől, el a mindent felfordító hullám útjából.
Végigszaladt a levél kopár főutcáján, lecsúszdázott a sima száron és a törzs
rendíthetetlen kérgéhez tapadt. Ekkor érte el a hullám.

Az egész fát megrengette egy borzasztó dörgés. A levegőt homokszemek kavargása
töltötte meg. Közöttük egész falevelek forogtak, sőt Filep egyben leszakított
gallyakat is látott. Egy madár menekült a forgatagból, de útjába kerültek a
kiszámíthatatlanul vergődő ágak. Filep minden erejével a kéreg apró repedéseibe
kapaszkodott, és szorosan zárt szemekkel remélte, látni fogja még családját.

Manónyi homokszemek martak a faágba körülötte. Egy barna falevél borította el,
és egy pillanatra eltakarta a vakító fényt, de aztán továbbrepítette a szél, és
megpróbálta Filepet is magával vinni. Apró páracseppek záporoztak körülötte, és
ahogy a fa törzse meghajolt, az egész táj mozgásba lendült. Valami megragadta
Filepet, és mindketten a mélybe bukfenceztek.

\secbreak

Szerencsére Filep a széllökés nagy részét a kéreghez lapulva vészelte át, így
nagyjából egyenesen estek lefelé. Elsuhantak egy ág mellett, de a következő
nagylelkűen az útjukba nyújtotta egy levelét. Egymás nyakába-lábába gabalyodva
érkeztek a zöld felszínre. A világ tovább forgott körülöttük, míg a falevél
remegése lassan lecsillapodott.

Filep felült és egy öreg kék manó ült fel vele szemben.

-- Ó, jaj, hol lehetek -- sopánkodott a kékbőrű hölgy. Megigazgatta finom
ruháját és teljesen kusza hajába is visszasimított egy tincset.

-- Köszönöm, hogy segítettél földet érni, manócska -- hálálkodott Filepnek.
-- Ez aztán különös időjárás, nem igaz? Legjobb lesz, ha megkeressük szüleidet
és főzünk neked egy jó teát. Egészen elsápadtál, látom, teljesen kiment a kék
az arcodból.

Filep meglepetten tapogatta meg az arcát. Valóban nem volt kék, de nem is
szokott kék lenni.

-- Filep vagyok, a felfedező -- bökte ki jobb híján.

-- Ó, nagyon szép név. Én Mazsi vagyok. Melyik levélről indultál
felfedezőútra, Filep? Azt hiszem, most már mindent láttunk és ideje pihenőt
tartanunk.

Filep egyetértett ezzel, és amint a fa megállapodott, útjukra indultak
Cseppcsurran felé.

Mazsi olyan kicsi volt, mint Filep, pedig még a felnőtteknél is öregebb volt.
És nagyon kék. Egy távoli fáról repítette el a szélvész, és azt mondta, ott
mindenki kék. Nagyon sokat tudott az erdőről és még azt is tudta, hogy az új
napot rakétának hívták. Dübörgése mostanra már eltávolodott, és ha a levelek
között megpillantották, már tűhegynyire zsugorodott vakító lángja.

-- Hogy fogsz hazamenni, Mazsi? -- kérdezte Filep. -- Talán a leghosszabb
ágakról át lehet ugrani a szomszédos fára, de napokig tarthat így az út.

-- Hetekig! -- helyesbített Mazsi. -- Ajtónkon kopogtat már az ősz. Nem
lehet ilyenkor olyan hosszú utat megtenni. Az elszáradt leveleken nincs már mit
enni és minden fa barnába fog borulni még mielőtt haza érnék. El kell vackoljak
veletek, Filep, vagy\dots

-- Vagy építenünk kell egy rakétát! -- javasolta Filep.

Mazsi egy örökzöld fáról jött, ahol a levelek keskeny tűk és a manók a
belsejükbe vájják lakásaikat. A lakásokat hosszú csigalépcső köti össze, de a
gyerekek az ablakokon is ki-be mászkálnak. A ragacsos gyanta télen sem fagy meg
és kiváló mérnökeik a lakások fűtését is megoldották. Mazsi is mérnök volt és
bár sosem hallott róla, hogy manók rakétát építettek volna, nem tartotta
lehetetlennek.

-- A legfontosabb kérdés az üzemanyag. Ami elég hevesen ég, gyorsan el is
illan, így a természetben nem fogunk könnyen találni ilyet -- gondolkodott
Mazsi.

-- Pontosan tudom, hol szerezhetünk ilyet! -- mondta Filep, és lecsúszott a
levél szárán.

\secbreak

Filep a levél szára tövében várta Mazsit, de úgy festett, a manóanyó óvatosabb
módját választotta a leereszkedésnek. A látókövön át látta Filep, ahogy
óvatosan ereszkedik, de a kék manók fáján egész másak voltak a levelek és Mazsi
nem boldogult ügyesen a viaszos felszínen. Két kapaszkodás közé gyakran
beillesztett egy bukfencet és nem sokára már a szár alsó oldalán lógott hosszú
kék sáljáról. Filep elkezdett visszamászni, hogy segítsen neki, de még fél úton
sem volt, amikor Mazsi visítását hallotta. Útitársa sála leoldódott a szárról
és most az aláhulló manó nyomában lobogott. Filepnek egy szemvillantás alatt
kellett döntenie. Úgy döntött, elrugaszkodik a szárról, röptében megragadja
Mazsit és másodjára is együtt pottyannak le valahova.

A hosszú sál lassította az esésüket, de kiszámíthatatlanná is tette. A rakéta
moraja alig hallatszott már, de még kavargott a levegő és most a két manót is
megforgatta. Megkerülték a levél szárát, elsuhantak a faág mellett, tettek két
kört alatta is, kicsit felemelkedtek, majd keringőzve tovább hullottak.

-- Ez aztán a mulatság, nem igaz, Filep? -- visongott Mazsi.

A világ tett még néhány kört körülöttük, majd váratlanul megállapodott. A kék
sál is megdermedt a levegőben, mintha láthatatlan kezek fogták volna meg. A
manók egy ragadós pókfonálon kötöttek ki.

-- Vigyázz, Filep, ez egy pókháló! -- sivalkodott most az apró kék manó. --
Menekülnünk kell! Meg kell szabadulnunk! Ha ideér a pók, végünk!

Mazsi rettenetesen kapálózott, de hiába. Kétségbeesett erőfeszítése viszont
magukra vonta a pók figyelmét. Láthatatlan pókfonálon közeledett feléjük a nagy
fekete test.

-- De jó, hogy újra látlak, Filep -- mondta a pók. -- Nagyon aggódtam érted.
Micsoda szél volt ez! Annyi manó hullott a hálómba, hogy alig győztem
kiszabadítani őket. Szerencsére fogtam egy szép nagy bogarat is -- kuncogott.

Mazsi arcán a rettegés teljes értetlenségbe fordult, majd hitetlenkedve vonta
fel szemöldökét, és szája széles mosolyra görbült.

-- Ez Mazsi -- mutatta be Filep, -- egy örökzöld fáról fújta ide a szél.

Mazsi is megtalálta a hangját és lelkesen üdvözölte a pókot. Kiderült, az ő
fáján manóevő pókok éltek és rendkívül veszélyes volt a hálójukba cseppenni.

-- Persze sokkal jobb a barátság -- tette hozzá Mazsi kicsit zavarban. Nem
szeretett volna új táplálkozási ötleteket adni a póknak.

-- Teljesen egyetértek -- mondta Gergő mamája, -- és nem is finomak a manók.

Miután kiszabadította Filepéket a ragacsból, megmutatta nekik Gergő félkész
kabátját is. Jól elbeszélgettek a kötés apró trükkjeiről, és a pók még
pálinkával is megkínálta Mazsit.

-- Nagyon sok hordó hullott Szolperlából a hálómba. Nekem hagyták őket
cserébe, hogy a manókat kiszabadítottam a hálóból. Igyunk a nagy szerencsére,
hogy mind átvészeltük ezt a szélvészt.

-- Ez igen -- csettintett nyelvével Mazsi. -- Ez kiváló üzemanyag lesz a
rakétánkhoz.

A pók azt mondta, sosem fogyasztana el ennyi hordóval, és üzletet ajánlott a
manóknak. Megkapják mind az üzemanyagot, ha cserébe egy pókfonalat kötnek a
rakétára. Így a pók át tud majd sétálni a távoli fára. És ha a két manónép
között kereskedelem épülne ki, ő vámot szedhet majd.

A két manó nagyon lelkes volt. Elfogadták az ajánlatot, és kaptak tizenegy
hordó pókpálinkát. Ahogy megérkeztek a háló aljába, Mazsi elkérte a látókövet
és az útinaplót, és mérésekbe és számításokba kezdett. Milyen messze van az
örökzöld fa? Milyen ívben legjobb repülni? Mennyit nyom egy manó és hány hordó
pálinkára lesz szükségük?

Filep először követte az eredményeket, de aztán elkalandozott a figyelme. A
faág felszínén valami mozgott a távolban. Vissza akarta kérni a látókövet, hogy
megnézze, mi az, de Mazsi nem adta. A hálóba kapaszkodva próbált jobb rálátást
nyerni. Megpróbálta fülét a kéreghez tapasztani, hogy a lábdobogást hallgassa.
Valami szőrös és apró nyargalt feléjük.

Pár pillanat múlva már Mamusszal kergetőztek Mazsi körül. A papucsállatka akadt
valahogy nyomára, és Filep nagyon örült a társaságnak. Egymás után nyargalni
sokkal izgalmasabb volt, mint a bonyolult számításokat követni. Nem sokára
Mária tűnt fel a láthatáron. Mamusz Fileptől Máriához szaladt, Máriától pedig
Filephez, míg a manófiú és manólány el nem érték egymást.

-- Hát megvagy, Filep -- örvendezett Mária. -- Cseppcsurranon mindenki
nagyon aggódik érted. Még szerencse, hogy nem a fa tetején voltál, amikor
mindent felkapott a szél! Most az erdő túlsó végéről szedhetnénk össze\dots

Filep is nagyon örült Máriának. Elmesélte, hogy bizony épp a fa csúcsán volt,
amikor a felszálló rakéta szele elérte a fát. És elmesélte azt is, hogy mi a
rakéta, és bemutatta Mazsit is, aki kedvesen felnézett Máriára, de nem
szakította félbe a számításait.

-- És ez itt a mi rakétánk üzemanyaga -- mutatott a tizenegy hordó pálinkára
a pókháló tövében.

\secbreak

-- Meg vagytok bolondulva? Ha a távoli rakéta ekkora szelet csinált, akkor a
tiétek tőből kicsavarja a fát! -- mondta Mária.

-- Jaj, Filep, megijeszted Máriát -- szólt közbe Mazsi. -- Abban a távoli
rakétában több hordó üzemanyag volt, mint ahány levél van az erdőben. A mi
rakétánkban csak száz hordónyi lesz.

-- Száz hordónyi? -- kérdezte meglepetten Filep. -- De hát csak tizenegy
hordóval kaptunk a póktól.

-- Tudom, Filep. Sajnos úgy fest, találnunk kell még nyolcvankilenc hordóval
valahol.

Felmentek hát Szolperlába. A város sok hídja szakadtan lógott és a színes
korlátokat is megtépázta a széllökés. Az egyik ház is félig elvált a levél
aljától, és Filepék látták, ahogy manók csapata ügyesen kimenekíti a lakókat és
megpróbálja kötelekkel visszahúzni az alakját vesztett lakást. Egy munkás a
lengő teherszállító szálon ragadt és most a város alól nézett fel aggódón. A
dokk erős épülete szerencsére sértetlen volt, de bent dobozok és hordók százai
gurultak ki a raktárakból és a munkások szorgosan próbálták visszaállítani a
rendet.

Filepék segítettek szétválogatni egy raktár tartalmát, és meghallgatták milyen
károkat szenvedett a város. A manók hozzászoktak a váratlan fuvallatokhoz, és
legtöbben meg tudtak kapaszkodni valamiben. Akik leestek, azokat a pók hozta
vissza. De sajnos a pálinkás hordók elszabadultak és mind kigurultak a
raktárakból és a mélybe vesztek. A munkavezető szobájában volt egy megkezdett
hordó, és ezt Filepnek adta cserébe a segítségért.

-- Ha így megy tovább, tavaszra sem lesz meg a száz hordónk -- kesergett
Mária. -- Pedig már úgy beleéltem magam, hogy én is egy rakétán utazhatok.

-- Ha elveszett Szolperla pálinkakészlete -- gondolkodott hangosan Filep, --
az a biztos megoldás, ha magunk pároljuk le erjedt mézharmatból. Szolperla
alatt van egy levél, ahol a tetűnyájat legeltetik. Talán tudnak mézharmatot
adni.

Kis csapatuk leereszkedett hát a város alatti levélre, hogy felmérje az ottani
készleteket.

-- Bizony növesztek még négy lábat, ha egész nap a pók hálóján mászunk fel és
alá -- mondta Mazsi, ahogy a levélre érkeztek. Filep és Mária már egy
manócsoport lábai között furakodott előre, és Mazsi is követte őket.

A manók egy megvadult tetűt álltak körül. A rovar kisebb volt egy hangyánál, de
még így is a manók fölé tornyosult. Mindenki vékony pókfonalat tartott a
kezében, és igyekeztek a kör közepén tartani a makrancos állatot. A tetű
nyerített és próbálta levetni a fonalakat, de egy bátor manó a hátára mászott
és rágóit összekötözve igyekezett megbékíteni. Filepéket egy fiatal manóasszony
gyorsan kiterelte a körből.

-- Óvatosan, kedveskéim -- mondta, ahogy elvezette őket a fenevaddal birkózó
manók gyűrűjétől. -- Még valaki rátok tapos. Hogy mi történik itt? A tetvek
általában nagyon szelíd állatok, de megvadulnak, ha elszakadnak a nyájtól. A
nyáj sajnos kitört a karámból, amikor a szél megdöntötte a fát. Ez a szegény
borjú valahogy lemaradt, és a nővérem a törzs közelében meglasszózta. Egy tucat
erős manó segített neki visszahozni, de most már nem soká meg fog békélni. 

-- Mi lesz velünk, Jutka? -- kérdezte az asszonyt egy idősebb fiú. -- A nyáj
nélkül nem lesz mit ennünk. Ha eljön a tél, mit csinálunk?

-- Egész évben a télre készültünk, tele van az éléskamra -- nyugtatta meg
Jutka a fiút. -- De tavasszal összehúzhatjuk a gatyamadzagot, ha ezt az egy
borjat legeltetjük a levélen.

-- Én Filep vagyok, a felfedező -- szólt hozzá félénken Filep. -- Talán
segíthetünk megkeresni az elkóborolt nyájat.

-- Hova valósi vagy, Filep? -- kérdezte Jutka.

-- Cseppcsurranon lakom.

-- A hangyabirodalomban laksz hát. Akkor tudnod kell, hogy a hangyák még
napnyugta előtt befogják a nyájat, és bármily szépen kérjük, nem fogják
visszaadni. Válaszolni se fognak, és azt se fogod tudni, hogy látnak-e téged.

Filep tudta, hogy a hangyák látják, és Balázs lehet, hogy válaszolna is. De
tudta, Jutkának igaza van, a hangyák sosem adnák vissza a tetűnyájat. Eszébe
jutott viszont az öreg napimádó hangya, és felébredt benne a remény, hogy ő
talán jó tanáccsal fog szolgálni.

Nekiláttak hát sokadjára is a pókhálónak. A nagy kapaszkodás közben Mária törte
meg a csendet.

-- Mi lesz, ha nem találunk több üzemanyagot? Felhígíthatjuk ezt a tizenkét
hordót? Vagy megkérjük az örökzöld fa lakóit, hogy küldjenek nekünk? Vagy
játszunk valami mást?

-- De már csak nyolcvannyolc hordó hiányzik -- mondta Filep. -- Egész biztos
vagyok benne, hogy a hangya segíteni fog. Az üzemanyag szinte készen áll az
induláshoz.

-- Értettem, Filep kapitány! És mi kell még a rakétához? -- fordult most
Mazsihoz.

-- Kell egy erősfalú ház, amiben repülni fogunk. Egy távollátó, hogy célba
találjuk. És kell egy fúvóka -- sorolta Mazsi.

-- Egy fúvócska? -- kérdezett vissza Mária. -- Az micsoda?

-- Ha meggyújtjuk az üzemanyagot, a lángja megy mindenfelé. Olyan térben kell
meggyújtsuk, hogy csak hátrafelé mehessen a láng, így fog minket előrefelé
nyomni -- magyarázta a kék manó. -- A fúvóka egy különös cső, aminek egyik
szája keskeny, itt csurog be az üzemanyag, a másik szája pedig széles, itt tör
ki a láng.

-- De Szolperlában azt mondták, a láng mindent eléget -- gondolkodott el
Filep. -- Nem fogja a fúvókát is elégetni?

-- Én is hallottam egy mesét nagymamámtól -- tette hozzá Mária, -- amiben
Jánost utoléri a pók, és tüzet okád rá.

-- Ez aztán nem semmi -- hüledezett Mazsi. -- Tűzokádó pók? Talán túl sok
pálinkát ivott? És miből fonta a hálóját, hogy ne égjen el?

-- Épp ez a lényeg -- folytatta Mária. -- A háló elégett, a levelek elégtek,
az egész fa is elégett és még a föld is elfeketedett. Végül a koromban és
feketeségben ott állt János a pókkal szemben.

-- Hogyhogy ők nem égtek el? -- kérdezte Filep. -- Azt mondod, csináljuk a
fúvókát manókból vagy pókokból?

-- Azért nem égtek el -- magyarázta Mária, -- hogy folytatódhasson a
történet.  Viszont a föld sem égett el. Csináljuk a fúvókát földből!

Ezzel az okfejtéssel senki nem tudott vitatkozni, és megegyeztek a tervben.
Filep dolga lesz a pálinkakészlet feltöltése. Mazsi a rakéta testét építi meg,
Mária pedig fúvókát formál földből. Egy darabig viszont még együtt ment az
útjuk, és a pók hálójától együtt cipelték el a hordókat. Fejük felett mind négy
hordót egyensúlyozva meneteltek a faágon a törzs felé. Mazsi a tűlevelű fán
maradt családjáról mesélt a gyerekeknek út közben.

\secbreak

-- A legnagyobb unokám pedig Róbert -- fejezte be a felsorolást Mazsi. -- Ő
akkora lehet, mint te, Filep! Vajon miben mesterkedhet épp\dots

Míg Mazsi utolsó unokájához ért, megérkeztek a hangya szentélyének kapujába, és
le is rakodták a nehéz hordókat.

-- Alig várom már, hogy találkozzatok velük, Mária! Biztos remekül meglesztek
egymással. Gyertek, nézzük meg, mit tud mondani Filep hangyabarátja! --
mondta, és mind beléptek a míves kapun.

A szentély üregét most is varázslatos arany fény töltötte be, és apró
fénypontok keringtek az illatos levegőben. Bár többen voltak, még nagyobbnak
tűnt most a csend, mint mikor Filep egyedül járt itt. Mazsi magához ölelte a
gyerekeket és sokáig szó nélkül nézték a hangyát. A fafaragások félkörében most
is felső lábait összeillesztve imádkozott. A szentély mozdulatlanságát csak a
kapun áttörő fény útjában ringó falevelek árnyai bontották meg. De a levelek
nagyon messze voltak, és hiába rajzoltak örökké változó mintákat a hangya
páncéljára. A szentély őre halott volt. Homlokából és vállából hosszú száron
aranysárga lándzsák törtek elő. Sárga tőrök emelkedtek ki a lábaiból is.

A manókat megilletődöttségükből egy aláhulló sárga szilánk térítette magukhoz.
A szilánk akkora volt, mint Filep és ahogy a padlóra esett, arany porfelhő
szállt fel körülötte. Mazsi habozva engedte el karjából a manófiút, de Filep
már fel is emelte a különös töredéket. Nagyon könnyű volt, de erősnek is tűnt.
Felszínét ezer lyuk borította, mintha valaki a hangya faragásait akarta volna
ügyetlenül lemásolni. A levegőben megforgatva sűrű porfelhőt hagyott útjában.
Hogy ne fessen mindent sárgára, Filep a pókháló térképébe bugyolálta.

\secbreak

Mária csúszdázott le elsőnek az elhagyatott alagútban. Mikor visítása elhalkult
a sötétben, Mazsi követte. Az ő hangja is elhalkult, majd manókacagás
csilingelt a sötét járatban. Filep erre a jelre várt, és egymás után leengedte
a pálinkáshordókat a csúszdán. Minden hordó érkezését bukfencezés és
manónevetés hangjai jelezték, míg el nem fogytak a hordók. Filep is
leereszkedett, majd elbúcsúzott Máriától és Mazsitól. Ők ketten megismételték a
csúszdázást a szú járatának második szakaszában és Filep elbúcsúzott a
hordóktól is. A terv szerint Cseppcsurranon fogja őket viszontlátni, ahol Filep
nyolcvannyolc hordójával és Máriáék fúvókájával kiegészítve hozzákezdenek a
rakéta megépítéséhez.

Filep elővette látókövét és hosszasan pásztázta az alatta elterülő leveleket. A
hangyák legelőjét kereste, de pillantása megakadt egy másik levélen. A látókő
segítségével újra Cseppcsurran utcáin érezhette magát. A házak tetejét
felborzolta a szélvész, és a levél nevét adó cseppnek nyoma sem volt a főtéren.
De az épületek mind álltak, és miután Filep visszafelé bejárta a reggel
megkezdett útját, szülői otthonánál kötött ki. Nem látott át a tetőn, de
remélte, szülei otthon vannak és nem aggódnak miatta. Nem akarta őket tovább
váratni a kelleténél, így hát tovább fordította a látókövet.

A szomszédos leveleken is manók éltek, de az egyiken épp körbejárt egy hangya.
Az ágon talált még kettőt, de ők is csak őrjáratukat végezték. Követte útjukat
egy másik ágra is. Ezen végigpásztázva talált egy rügyet, amit gyanús foltok
borítottak. Árnyékban volt, így Filep nem látta jól, de gyanította, itt
legelhet az eltévedt nyáj.

Filep fürgén ereszkedni kezdett a kérgen és közben átgondolta, hogyan győzi meg
a hangyákat, hogy visszaadják a manók tetveit. Jutkának igaza volt, a szép
szóra biztosan nem fognak hallgatni. De ha meglasszózná a hangyákat,
lepányvázhatná őket, mint a manók tették a fiatal tetűvel. Vagy a rakéta
hangját utánozva megijeszthetné a nyájat, és amikor felmenekülnek egy levélre,
a királyi pallossal levágja az egészet. Ha jól időzíti, egy széllökés pont
felrepítheti őket a magasba. Pókfonállal hurkot vetne egy faágra, és ahogy a fa
körül kering, beleütközne a pókhálóba. Innen az összes tetű egyenesen
lebucskázna az alattuk fekvő karámba.

A terv egyes részei még hiányoztak, például a királyi pallos György
nappalijában volt a falon. De nagy vonalakban tökéletes volt. Filep máris
magabiztosabban lépdelt a rügy felé. Innen már azt is látta, hogy nem csak a
rügyet borították tetvek, de a mellette növő levél szárát is. Hirtelen egy
hangya előzte meg Filepet. Hatalmas lábain elrobogott Filep mellett és észre
sem vette a harcra kész manót. Filep lábai meginogtak, ahogy a nagy tömeg
elhaladt mellette, de nem fordult vissza.

Mi van, ha tényleg nem látják a hangyák? Talán elcsalhatja a legnagyobb tetűt,
a többi mind követni fogja, és a hangyák bambán néznek tovább.

Nem így történt. Akárhogy tolta vagy húzta a nyáj fejét, az egy lépést sem
mozdult. Filep próbálta megemelni a monstrum lábát, de ennyi erővel egy
hangyával is birkózhatott volna. Felmászott a hátára és a potrohára csapott
inkább. A tetű ezt sem vette észre. A fejére ült és a szeme előtt integetett.
Lábával az orrát böködte, és a fülébe kiabált, de a tetűt csak a növényi nedv
érdekelte. Hogy is sikerült Balázzsal szót értenie? Hátha a manódal ennek a
behemótnak is tetszeni fog.

\secbreak

Míg Filep a fa legnagyobb tetűjét próbálta mozgásra bírni, Mária és Mazsi
Cseppcsurranra érkeztek. Négy hordót egyensúlyozott mindkettejük a feje tetején
és két másikat görgetett maga előtt lábával. Így nem csoda, hogy nagy feltűnést
keltettek a levélre érkezve.

-- Mi ez a sok hordó? -- gondolta a falu szélén egy algatermesztő, és a
tetőjavítást félretéve nézte az érkezőket.

-- Nézd, kék néni! -- mutatta testvéreinek egy kismanó. De hiba volt ezért
megtorpannia, mert bátyja és húga hátulról elgázolták és egy kupacban birkóztak
tovább.

-- Mi ez a pálinkaszag? -- nyitotta ki ablakát a suszter. A rakéta lökése
összekeverte a cipőit, és jó részük még mindig nem találta meg a párját.

-- Hogy másztak vajon fel a levél szárán ennyi hordóval? -- gondolkodott egy
fiatal manó a főtéren. Azért illedelmesen segített nekik lerakodni, és mire
végeztek, körbeállták őket az algatermesztők, kismanók és suszterek is.

Mária megosztotta velük a rakétaépítés tervét, és akinek nem volt fontosabb
dolga, felajánlotta segítségét. Mazsi vezetésével ez a csapat hozzálátott, hogy
a megrongálódott tetőkből és egyéb törmelékből megépítse a rakéta testét. Mária
pedig felszaladt Filep szüleihez, hogy megnyugtassa őket.

Filep édesapja és édesanyja nagyon hálásak voltak a hírért, és
gyümölcscukor-kristállyal kínálták Máriát. A kristályon még érezhető volt a
bogyó íze is, amiből kivonták, és nagyon jól esett Máriának a cipekedés után.
Elfogadott még egyet az útra, de cserébe meg kellett ígérnie, hogy sötétedés
előtt hazahozza világlátott fiukat.

Új lelkesedéssel szaladt végig a főutcán és elrobogott Mazsiék mellett is, akik
épp próbálták egy növényi rosttal megakadályozni, hogy tákolmányuk összedőljön.
Mikor mégsem sikerült, a robaj megtorpantotta Máriát.

-- Menj csak, Mária -- mondta Mazsi. -- Mire visszaérsz, kész leszünk. Ez
még csak egy vázlat, a végleges építmény sokkal erősebb lesz.

Mária lelket akart tölteni Mazsiékba, és nekik adta a finom gyümölcscukrot. Egy
kicsit még hallgatta, ahogy a rakéta szerkezetéről beszélgetnek, aztán útnak
indult a föld felé.

Kevés mese szólt a földről. Nagymamája azt mondta, azért, mert a manók nagyon
régen a fákra költöztek. Hogy mit csináltak azelőtt? Ő a dédmamájától azt
hallotta, hogy egy pompás városban éltek. Akkora tornyai voltak, mint egy fa.
És hatalmas gépeik is voltak, amik mindent megcsináltak helyettük. A gépeknek
olyan formája volt, mint egy manó, de olyan vastag volt a karjuk, mint egy
faág. A gépek sokáig szolgálták őket, és a manók lusták lettek és
elfelejtették, hogyan kell dolgozni. Azt is elfelejtették, hogyan kell a
gépeket megjavítani, ha meghibásodnak. Egy nap a gépeknek meghibásodott a szeme
és nem látták többé az apró manókat. Nem tudták tovább etetni őket, nem
építettek nekik házakat és nem védték meg őket a rovaroktól. Csak erre vártak a
hangyák és bevették a pompás várost. A manók menekültek amerre láttak. Mária
ősapja és ősanyja az erdőbe menekült. Nem tudták, hol lehet élelmet találni és
nem tudták, mit szabad megenni. Az első év nagyon szűkösen telt, de nem adták
fel, és újra kitanulták, hogyan kell megélni.

A törzs egyre szélesedett, és a kéreg repedései is mélyebbek lettek. Mária
útját homokszemek tömbjei lassították. Egyik-másik tömb akkora volt, mint egy
felnőtt manó. Mária mindegyik szem tövét megvizsgálta.  Az esőtől felvert
sárcseppek tapasztották a homokszemeket a kéreghez és Máriát a sár érdekelte.
Az egyik tömb alatt kuporogva elővette kvarckését. A fényes, törhetetlen pengét
is attól a nagynénjétől kapta, akitől a kedvenc kis gombáját. A késsel kikapart
egy maréknyi agyagot a homokszem alól. Kezében forgatta az apró kristályokat,
és próbálta kiválasztani a legtisztábbakat. Nyelvét hozzáérintve megkóstolta
őket, majd egyiket fancsali képet vágva eldobta. A másikat szoknyája zsebébe
pottyantotta és ment tovább.

\secbreak

A vezértetű ezalatt csupán három lépést tett. Egyet előre, egyet jobbra és
egyet hátra. Filep sem erővel, sem dallal nem tudta haladásra bírni. Próbált
már a látókövön át is farkasszemet nézni a makacs rovarral. Hátha a
felnagyított manó szavára jobban hallgat. De ez sem vált be. Most hátizsákja
mélyén keresgélt remélve, hogy talál valamit, amivel hathat hátasára.

Fél tucat élénk színű kvarckristály. Nem lehetett velük írni a finom kitinre,
megenni sem akarta őket a tetű és hiába zsonglőrködött Filep, erre sem
mozdította meg egy csápját sem.

Egy nagy csomó pókfonal. Ezt sok küzdelemmel a tetű lábaira és testére csomózta
Filep, de a pókselyem nagyon rugalmas, és kicsit sem akadályozta foglyát a
mozgásban. Nagyszerűen lehetett viszont hintázni rajta.

Két gyűrött muslincaszárny. Ezeket kihajtogatta és a tetű fejére erősítette a
pókfonál segítségével, mintha lelógó fülei lennének. Ha a tetű fejét forgatta
volna, hosszú műfülei viccesen lengtek volna.

Egy növényi rostokból szőtt bojt. A rostokat különböző színekre festették és
Filep édesanyja mutatta meg, hogyan kell összefonni őket. Ezt megmutatta a
tetűnek is és összefonta három sörtéjét, de hiába.

Ott volt még maga a finom növényi membránokból szőtt hátizsák. Jobb ötlet híján
Filep ezt saját fejére húzta, és a tetű hátán szaladgált tapsolva és kiáltozva,
míg le nem esett. Szerencsére estében beleakadt a lábak közé kifeszített
pókfonálba.

Végül a táskán kívül ott volt a térképbe bugyolált aranysárga szilánk. Filep
nem tudta pontosan, mi az, de félt tőle. Biztos volt benne, hogy az öreg hangya
segíteni fog neki, de nem kapott mást tőle csak ezt a poros töredéket.
Felnőttektől hallotta, hogy mindennek lelke van. Talán a hangyák lelke sárga. A
vén hangya lelke túl nagyra nőtt és kitört a testéből.

Kicsomagolta a szilánkot és egyik kezéből a másikba vette. Érintése tenyereit
is ragyogó sárgára festette.

-- Filep, csomagold vissza -- szólalt meg egy hangya a háta mögött. Filep
Balázsra ismert rá, egykori hátasára, aki a Cseppcsurrantól a hangyabirodalom
határáig elvitte. De az is lehet, hogy minden hangyának ugyanolyan a hangja.

-- A manók tetűnyája elszökött -- magyarázta Filep -- és azért jöttem, hogy
visszatereljem őket. Remélem, nem baj.

-- Én vittem fel a beteg hangyát a birodalmon túlra -- mondta a hangya. --
Mindennek mennie kell, amit megérintett az aranygomba. Te sem maradhatsz itt és
ez a tetű sem. Rosszak vagytok. Tiszta sárga a tenyered. Ne nyúlj semmihez!

-- Mert ha valamihez hozzányúlok? -- feleselt Filep.

-- Minden rossz lesz! Ne nyúlj semmihez!

A hangya meg akarta ragadni Filepet, de láthatóan tartott a sárga szilánktól.
Megkerülte a tetűt, hogy a másik oldalról próbálkozhasson, de visszariadt
amikor Filep felé fordult.

-- Ez a tetű már a tiéd, csak csomagold el a gombát.

Filep elcsomagolta a lélekgomba szilánkját, és lemászott a tetűről sárga
tenyérnyomok sorát hagyva maga mögött. A hangyák mind tisztes távolságot
tartottak tőle. Bármerre lépett a hangyák elhátráltak. Amikor elég messze volt
a vezértetűtől, azt gyorsan felkapták és elvitték a rügyről.

Filep aranyló ujjait karmokként maga előtt tartva ijesztgette a hangyákat.
Nagyon élvezte, hogy végre nem néznek át rajta. Épp ellenkezőleg, most mind rá
figyeltek. Szaladgálni kezdett a hangyák között, és sikerült úgy megijesztenie
egyiküket, hogy az leesett az ágról. Ujjával sárga csíkot húzott a zöld
felszínre és kinevette a hangyákat, akik nem merték átlépni.

A tetvek viszont nem féltek tőle, akárhogy próbálta ijesztgetni őket. Még arra
se hederítettek, ha sárga pecséteket hagyott rajtuk. Persze a hangyák ezektől a
megbélyegzett tetvektől is tartottak és nagyon óvatosan elszállították őket.
Filepet bántotta, hogy megijeszti a hangyákat, de nem tudta abbahagyni a
játékot. Addig szaladgált és tapicskolt, míg az egész nyájon sárga foltok nem
virítottak, és a hangyák kötelességtudón el nem vitték őket.

-- Bocsánat -- mondta Balázsnak. De a hangyák elhátráltak a rügyről és Filep
egyedül maradt.

\secbreak

Mária elérte a fa tövét. Itt kiszélesedett a törzs és a gyökerek megindultak a
föld alá. Sikerült útközben találnia egy kevés finom agyagot, de még legalább
kétszer ennyire volt szüksége. Mielőtt a felszínre ereszkedett volna,
szétnézett egy vastag gyökér tetejéről. Félelmetes gondolat volt elhagyni a
fát. Körülötte mindent avar borított. Száraz, halott levelek, amikről gombák
rágták le a lemezeket. A legalsó levelek voltak legrosszabb állapotban. Ezek
talán még a múlt évben hullottak le. Ha valaki közel merészkedne hozzájuk,
tavalyi manóvárosok romjait is fellelhetné.

Mária hirtelen zajra lett figyelmes. Egy fürge gyík szaladt az avarban, egész
leveleket lökve félre útjából. Egy ekkora állat le tudná szakítani
Cseppcsurrant a faágról. A gyík a lemenő nap irányába szaladt, és a Nap
korongja előtt egy még nagyobb állat körvonalai sötétlettek. Egy őz szakított
le éppen egy friss gallyat a szomszédos fáról. Ahogy komótosan elrágta, Mária
csak remélni tudta, hogy nem éltek manók a levelein.

Összeszedte bátorságát, és tovább ereszkedett a gyökéren. A kéreg itt hiányos
volt, és helyenként a fa belső felszínének hosszú vájataiban gyalogolt. A
kanyon oldalában még ott magasodtak az öreg kéreg maradványai. A kéreg csúcsai
még a fa fiatal korából származtak. Számtalan tél és nyár koptatta és
szárította őket. Elvesztették már a friss kéreg gombaelleni védelmét is, és
zuzmók színes fodrai vetettek árnyékot Máriára.

-- Manócska! Kislány! Hova, hova? -- kiáltott le egy hang a zuzmóból. Ha nem
kiált, Mária észre sem veszi a kövér szürke atkát. Hízott teste a zuzmó felső
rétegének gombafonalakból szőtt foteljében pihent. Nyolc lába a mélyedés
peremén lógázott. Egyikkel lustán felhasította a gomba sűrű szőttesét és az
alsó rétegből egy maréknyi zöld algát húzott elő.

-- Megmondom, ha válaszolsz egy találós kérdésemre! -- kiáltott fel Mária. Az
atkák híresek voltak róla, hogy sosem aggódtak semmiért. Nem voltak sokkal
nagyobbak, mint egy manó, de olyan erősek voltak, mint egy hangya. Nem kellett
hát senkitől félniük, és nem lehetett előlük elrejteni semmit, amit
megkívántak. Kényelmes életüket békésen töltötték, és mindig kaphatóak voltak
egy kis játékra. -- Miből vannak a felhők?

-- Ó, kicsi manó, ilyet kérdezni egy atkától -- nevetgélt odafent az atka, és
bekapta az előhalászott algacsomót. -- Unokatesóm híres utazó volt. Mindenfelé
elvitte a szél, még a felhők felett is járt -- folytatta teli szájjal. --  A
felhő nem más, mint rengeteg vízcsepp, amiket a szél a levegőben tart. Akkorák
lehetnek, mint a kicsike fejed, manócska. Ha megnőnek akkorára, hogy
beleférnél, nem tudja már fenntartani őket a szél és lecsöppennek. Ez az eső!

-- Én is utazó leszek, ha nagy leszek -- mondta Mária, -- és felmegyek a
felhők fölé. Megnézem majd, hogy igazad van-e. Ha tévedtél, visszajövök!

-- Megvárlak itt, drágám. De most mondd csak, hova mégy? A manók odafent
laknak a leveleken. Csak nem leejtettél valamit?

-- Agyagot keresek. Rakéta fúvókát építünk belőle.

-- A gyökerek között akarsz keresgélni? -- lepődött meg az atka. -- A
lámpásodat nem felejtetted otthon? Vak sötét van odalent, manócska. Tudod mit,
ha megválaszolod az én találós kérdésemet, adok egy lámpást. Egy tölcsérgomba
lemezei közt találtam, egy szentjánosbogár veszíthette el. Most a tiéd lehet,
ha jól válaszolsz. Mit jelent, ha sárgán villog a jánosbogár?

-- Kanyarodik -- vágta rá Mária. Meg is kapta az atkától a lámpát. Gombaszagú
szőrcsomó volt és a szőrök halványan derengtek az alkonyi fényben.

Mária megköszönte a hasznos ajándékot és folytatta útját. A gyökér görbült
egyet jobbra, egyet balra, majd egyet lefelé. Mária óvatosan követte a nagy
fekete göröngyök közé. Először semmit nem látott, de szeme gyorsan hozzászokott
a sötétséghez. A levegő dohos volt, érezte a földben dolgozó gombák és
baktériumok szagát. Sok nedvesség is megrekedt a föld gazdag szerkezetében, és
Mária nagyon félt, hogy belecseppen egy ilyen sáros medencébe. Vége lenne szép
ruhájának és az eddig összeszedett agyagnak is.

Ahogy a gyökér a föld alá ért, hajszálgyökerek eredtek belőle. Ezek
szétfutottak a talaj titokzatos világába és magukba szívták mindazt a
nedvességet és tápanyagokat, amikre a fának szüksége volt. Mária egyelőre
tovább ereszkedett. A talaj felső rétegében sok volt a bomló növényi anyag, és
az ezeken hízó apró lények mindent elleptek. Nehéz és visszataszító lett volna
félresöpörni őket, hogy hozzáférjen az ásványi rögökhöz.

Mélyebben Mária útját állta egy rég levetett kitinburok. Egy rovarlárva viselte
valaha, aki a gyökeret rágcsálva cseperedett fel, míg le nem vetette ezt a
bőrt. A gyökér azóta vastagodott és szorosan belenőtt az elhagyott burokba.
Mária a zölden fénylő szőrcsomót sapkaként a fejére tette és elővette éles
kvarckését. Átdöfte az akadályon és gyors mozdulattal kerek lyukat vágott
rajta. A hártyát felgöngyölve piros fényt látott meg a mélyben. Meglepetésében
elesett és belecsavarodott a finom kitinhártyába. Birkózott jobbra és balra,
hogy kiszabaduljon, de mire levetette a lárvabőrt, belepottyant a maga által
vágott lyukba.

Hogy megállítsa esését, Mária kését a mellette elsuhanó gyökérfalba döfte.
Lelassult, majd megállt és egy hajszálgyökérre érkezett. A kés szántotta
hasadékból egy tiszta vízcsepp csöppent fejére és megborzongott. Honnan
jöhetett a piros fény? Most nem látszott sehol.

Mária végigegyensúlyozott a keskeny hajszálgyökéren. Nagy, hűvös rögök közé
vezetett a fehér ösvény. A rögök falait fekete gombafonalak szőttese vonta be,
és szálai a rögök közti térben is ott kígyóztak. Az éles kés nyitott új utat
Mária számára a sötétségbe.

Már épp megszokta volna, hogy motolla módjára aprítja a fekete fonalakat, mikor
hirtelen elfogytak. Keze az üres levegőt csapkodta egy pillanatig, majd megállt
és körülnézett. A fonalrengeteg jobbra visszahúzódott, mert baloldalt nem
talált emésztenivalót. A baloldali rög csupaszon és szürkén derengett. Agyag! A
kvarckés újra munkához látott és Mária zsebei tovább gömbölyödtek.

\secbreak

Filep közben a Szolperla alatti birtokra érkezett. Nagy volt itt az izgalom,
mert találtak egy nagy tetűt, aminek fejére valaki szárnyakat kötözött. És
ahogy a tanya felé terelték megláttak még egyet. Amikor azt meglasszózták, egy
hangya épp felhozott még egy tetűt. Letette birodalmuk határán és nem engedte
visszamászni. Útját állta egész addig, míg a manók el nem terelték birtokukra.
Ez ment egész este, és a hangyáktól kétszer akkora nyájat kaptak vissza, mint
amit elvesztettek. És minden tetű oldalán öt apró ujj nyoma sárgállott.

Mikor meglátták Filep sárga tenyerét és meghallgatták a történetét, őt
ünnepelte a manók apraja és nagyja. Rögtön megterítettek egy hosszú asztalt, és
mindenki hozott valami finomat az éléskamrából. Ekkora nyájjal nem félték már a
telet. Savanyított peteleves és spékelt algaroládok voltak előéltelnek. Füstölt
tetűcombból vágott hajszálvékony szeletekbe csomagolt sajtgombócok követték
rántva, és amíg az ünneplő manók falatoztak, elkészült az édesség is. Egy nagy
pollent vágtak félbe, és kétféle nektárral töltötték meg. Az egyik felébe
meleg, fűszeres tölteléket tettek. A másikba hígabb, hideg nektárt csorgattak.
Ebbe egy csepp drága gyümölcsízt is kevertek, és a konyhát málnaillat lengte
be.

A lakoma egész este tartott, de Filepnek fontosabb dolga volt. Jutka segítségét
kérte, és a fiatal asszonnyal együtt megfejték az összes tetűt. Annyi
mézharmatot adtak, hogy az összes hordó megtelt.

-- Mekkora nyájat hoztál nekünk, Filep -- merengett Jutka. -- Több hordót
kell építsünk. Nagyobb karám is kell majd. Sok lesz a munka. A fiatalokat nem
engedhetjük fel Szolperlába kipróbálni a szerencséjüket. Tán még a levél sem
lesz elég nagy. Tavasszal két levelet kell keresnünk. Ha sok petéjük lesz, még
hármat is talán.

Miután kifogytak a hordókból, kosarakba, asztalterítőkbe és lepedőkbe
gyűjtötték a mézharmatot. Összességében több volt, mint Cseppcsurran
harmatcseppjében a víz. Már csak az maradt kérdéses, hogyan tudják ezt mind
megerjeszteni és lepárolni.

-- Jó édes mézet adtak az új tetvek -- mondta egy bajuszos öreg manó, -- és
melegek még az esték. Ha most hozzáadjuk az élesztőt, reggelre megpezsdíti.

-- De hogy fogjuk lepárolni? -- kérdezte egy másik manó, aki ikertestvére
lehetett volna. -- Nem lehet ennyi harmatsört kézzel elmorzsolni.

Tovább vitatkoztak, és Filep a pálinkafőzés folyamatának több trükkjével
ismerkedett meg, mint valaha hitte volna. A harmatsört például, hogy gyorsan
párologjon, apró gyöngyökre szokták morzsolni a manók. Ehelyett próbáltak
kitalálni valamit, de egy dologban mindenki egyetértett. Ha reggelre meg
akarják erjeszteni a mézharmatot, most bele kell keverni az élesztőt. Ebben
segített is az összes manó, aki épp szünetet tartott az ünneplésben. Filep
pedig haza indult.

\secbreak

Mária zsebei már úgy elnehezedtek a sok agyagtól, hogy le akarták húzni a
szoknyáját. Leleményesen nadrágtartót font néhány fekete gombafonálból, és
azzal tartotta fenn. A szürke agyagrög felszínét karcolások tömege borította,
és a kitartó munka három ujjnyi mélységben lecsupaszította. Az agyag nagyon
finom volt. Sem homokszemek, sem növényi maradványokat emésztő gombafonalak nem
bukkantak elő belőle. Pontosan erre volt szüksége Máriának a fúvókához, és
nagyon örült, hogy rátalált erre a rögre.

Egyik zsebe már teli volt, és a másikból sem hiányzott sok, mikor a kés
nyomában Mária meglepetésére piros fény fakadt. Valahonnan jönnie kellett a
halvány fénysugárnak, és hogy megismerje forrását, a kést megforgatva
megnagyobbította a lyukat. Hiába leselkedett, még csak pirosan megvilágított
agyagot látott, így tovább dolgozott, míg a lyuk tenyérnyire nem nőtt.

Arcát a nyíláshoz tapasztva látta már, hogy a rög nem tömör. Egy arasznyi
vastag agyagfalon vágott tölcsérszerű lyukat. Mária oldalán manótenyérnyire
nyílt, de a másik oldalon mintha csak ujjával bökte volna ki. Mintha ujjával
egy manó nappalijának tapétáját bökte volna ki. Egy kanapé hátát látta. Az
ülések egy kandalló felé néztek. Préselt cellulóztömbök parázslottak benne, és
meleg fénybe burkolták a szobát. A kandalló peremén kis képek álltak szépen
bekeretezve, felettük egy trófea ijesztő féregfeje meresztette bajuszát
Máriára. A piros csíkos tapétás falak kicsit befelé görbültek, de voltak azért
sarkai a szobának. Az egyik sarokban formás gomba nőtt egy agyagcserépből. A
sárga gombafej mintha világított volna a piros tapéta előtt.

A kanapéról egy manó állt fel, és Mária felé fordult. Körvonalai ragyogtak a
parázs fényében. Magas volt és erős. Széles vállai felett egy kopasz fej
kerekedett, de kétoldalt megtörte ívét a vastag bajusz. Épp olyan bajusza volt,
mint a trófeának. Egyik szemét fekete posztó takarta. A másikkal szigorúan
nézte a lyukat. Ahogy Mária hátrahőkölt, a szigorú szemben megvillant a zöld
szőrsapka visszavert fénye. A manólány sikított és futásnak eredt.

Nem futott messzire, mert arcába csapódtak a megtépázott gombafonalak,
elvesztette egyensúlyát és majdnem leesett a keskeny hajszálgyökérről. A
fonalak hálójában kapaszkodott meg, és bennük fogódzva sietett tovább. Úgy
dobogott a szíve, hogy nem hallott semmi mást. De úgy képzelte, a bajuszos manó
ledöntötte a falat és őrjöngve vágtat mögötte. Nem is mert hátrapillantani,
nehogy a zölden villogó tekintettel kelljen szembenéznie. Csak a gyökér felé
botladozott, el innen, fel a levelek közé, vissza Cseppcsurran jól ismert
világába.

Azt hitte, már örökre elveszett a gombafonalak fekete szövevényében, amikor
meglátta a gyökér sápadt falát. De alighogy szaladni kezdett, egy rémes árny
rajzolódott ki a gyökér előtt. Egy hosszú, manóderék-vastagságú féreg kígyózó
tömegén lovagolt a bajuszos alak. Egyetlen szeme hidegen verte vissza a zöld
szőrlámpás fényét. Mária ereiben megfagyott a vér és mozdulni sem engedte a
félsz. A rém piros fényű lámpát húzott elő kabátja alól és magasra emelte.

A vörös fényben épp olyan gonosznak látszott, ahogy Mária képzelte. Vagy
legalább majdnem olyan gonosznak. Kicsit kevésbé gonosznak, mert a vastag
bajusz alatt szája széles mosolyra húzódott.

-- Hó, Anakonda! -- csillapította hátasát. -- Megijeszted a vendégünket.

\secbreak

Kezdett már lenyugodni a nap, és a kéreg réseinek hosszúra nyúlt árnyékaiban
Filep is ijesztően kígyózó alakokat látott.

Nem sokkal azelőtt a pók hálóján még nem félt semmitől és fütyörészve lendült
egyik pókfonalról a másikra. A faragásokkal teli szentélyben viszont
elcsendesedett. Ahogy remélte, megkapta a segítséget a különös hangyától, de
nem értette pontosan, mi történt. A hangyák szerint rosszat csinált, a manók
szerint jót. Nem tudta, szülei mit szólnak majd hozzá. Ahogy forgatta magában a
gondolatokat, egyszer megnyugodott, másszor bűntudat gyötörte. De ahogy nyúltak
az árnyak egyre fogyott a büszkeség és nőtt a bűntudat lelkében. Vajon tele
van-e a hangyák éléskamrája? Lesz-e mit enniük tavasszal?

Mozgalmas alkonyatkor a kéreg világa. Előbújtak már az éj lopózó formái, de még
nem vonultak vissza a nappal éber lényei sem. Mindenki vadászott vagy rá
vadásztak ebben az órában. Egysejtűek iszkoltak medveállatkák elől. Atkák
birkóztak fonalférgekkel. Gombák fonalai csaptak le az elesettekre. És a
lelkiismeret örökké változó árnyai üldözték Filepet.

Szaladt a hosszú árnyak és nyirkos hangok elől, ahogy csak a lába vitte. Csak
érjen minél hamarabb Cseppcsurranba. Fáradt volt már, de ha lassított, közelebb
jöttek a hangok. Az egyik árnyékban ragadós gombafonalakba akadt a lába. Míg
kirángatta, utolérték az árnyak. És meg is előzték. Változatos formájú
egysejtűek és más apró lények hada csúszott és mászott el mellette. Ügyet sem
vetettek a manóra, mintha csak ők is Cseppcsurranba siettek volna, hogy
vacsorára otthon legyenek. Filep kiszabadította lábát és értetlenül nézett a
szőrös, nyúlós, csápos sokaság után.

Váratlanul árnyék vetült rá hátulról. Hát ez elől menekültek az egysejtűek! Egy
nagy sárga hullám emelkedett a kéreg szirtjei fölé és sebesen közeledett Filep
felé. Lassított, megállt majd újra lendületet vett. Filep nem nézte tovább,
futásnak iramodott. A nyálkagomba egy sötét, nyirkos zugban húzhatta meg magát
nappal, és mikor árnyékba borult a kéreg, eljött az ő ideje. A hatalmas sejt
lüktetve lökte magát előre, és elemésztett mindent, ami az útjába került. Ezért
menekültek a kéreg apró lakói, amerre láttak. Egy tucat manó sem tudott volna
megálljt parancsolni neki.

Filep igyekezett az utolsó napsütötte foltokon maradni, átugrálni az
árnyékokat. De egy nagy kéregszirthez közeledve muszáj volt a sötétbe lépnie.
Veszélyes volt az egyenetlen terepen vakon szaladni, de nem lassíthatott le. A
nyálkagomba felgyorsult a hűvös árnyékban, és már Filep sarkában volt minden
lüktetésével. Filep szorosan a szirt fala mellett futott, ahol a nyálkagomba
nehezebben érte el. De véget ért a fal, és Filep tovább rohant a nyílt téren.
Már majdnem elérte az árnyék végét, amikor egy gödörbe lépett, és megbotlott.
Nem is tudott rögtön felpattanni, mert a gödör nem engedte a cipőjét. A
nyálkagombának csak ennyi kellett, már lendült is Filep felé a hullám.

Hirtelen egy szőrgombóc bukkant elő a semmiből, és a hullámnak ugrott.
Gumilabdaként visszapattant, de a hullám is megtorpant egy pillanatra az
ütközéstől. A nyálkagomba felszínén remegések gyűrűje áradt szét. De mielőtt
messzire érhettek volna, a szőrgombóc újra rárontott.

-- Futás, Mamusz! -- kiáltott Filep, mikor sikerült kiszabadítania lábfejét.
Papucsállatka és manófiú egymás mellett szaladtak tovább a sárga hullám elől.
Szerencséjükre Cseppcsurranhoz közeledve simább volt a kéreg, és kevesebb az
árnyék. Itt a lenyugvó nap fénye megállásra kényszerítette a nyálkagombát, és a
két jóbarát boldogan kergetőzve ért haza.

\secbreak

A főtéren rég elpárolgott a harmatcsepp. Elpárolgott a korán kelő manók csapata
is. Nem voltak árusok, nem jött senki megtölteni a kulacsát. Sötétedéskor
kiürültek az utcák, becsukódtak az ajtók és ablakok. Filep csak egy öreg manót
látott a háza előtti padon ülni, de ő sem fogadta köszönését. Elaludhatott, és
csak akkor ébred majd fel, ha megcsípi az ősz előszele.

Mégsem volt üres a főtér. Nagy halom cellulóz és kitin borította. A kupacot
növényi rostok és pókfonalak szőtték át. Körülötte hevert rendetlenül a
tizenkét pálinkáshordó. És az egyik hordóból fülrepesztő horkolás hallatszott.
A horkolás ütemére lustán előre és hátra billent a hordó.

-- Hahó -- mondta bátortalanul Filep.

A horkolás üteme megtört ennek hallatára és fordítva kezdett billegni.
Mamusszal a sarkában közelebb lépett és előre hajolva belesett a hordóba.

-- Mazsi! -- kiáltott.

-- Hagyjál -- horkantotta a hordó. Mamusz megörült az ismerős hangnak és
beugrott Mazsi mellé. Nem lehetett kényelmes egyiküknek sem, mert nagy
forgolódás következett. A végén gurulni kezdett a hordó és a halomban álló
építőanyagokba ütközött. Mazsi Mamusszal a fején kibukfencezett belőle és
elterült a levél felszínén.

-- Filep kapitány -- kapta fel fejét egy pillanattal később, -- jelentem
kipróbáltunk két tucat különböző változatot a rakétatestre. És -- tette hozzá
alig hallhatóan -- kiürítettünk egy tucat hordót közben.

-- Ez remek hír, Mazsi kormányos -- válaszolt Filep. -- Megvan hát a
tökéletes szerkezet terve?

-- Még\dots vannak ötleteink. És Filep, sikerült visszakérni a hangyáktól a
nyájat?

-- Igen, sőt, még többet is adtak -- büszkélkedett.

-- Ez remek hír, kapitány! -- mondta. Majd elkerekedett a szeme. -- Micsoda?
Tényleg visszaadták a tetveket? Mit csináltál, Filep?

Filep elmesélte hát a hangyák legelőjén esetteket. Közben felültek a szétesett
építmény tetejére és élvezték a friss levegőt. Mamusz a lábukhoz kuporodott és
nevéhez híven melegen tartotta a manók apró lábujjait. Mazsi elgondolkodva
hallgatta, mennyire féltek a hangyák az aranyló portól és hogyan hordták fel az
egész nyájat a manóknak. Mikor Filep a lakomát írta le, igencsak megéhezett, de
tarisznyájában már csak egy apró szárított algát talált.

-- Nyálkagomba ilyen magasan a fán? -- hüledezett, mikor Filep a történet
végére ért. -- Szerencséd volt, hogy Mamusz ott termett. Én is találkoztam egy
nyálkagombával, amikor annyi idős lehettem, mint te, Filep. Az öcsém, Zsoli,
még kisebb volt, de igazi vadóc. Együtt megleptünk egy bolhát, és a hátán
körbelovagoltuk a fát. De a sima hátáról könnyű lecsúszni, és Zsoli addig
bohóckodott, míg az egyik ugrásnál ledobott minket a bolha. A kéreg egy árnyas
hasadékába estünk, és annyira nevettünk, hogy nem láttunk a könnyektől. Egymást
ölelgetve dülöngéltünk, de Zsoli hirtelen abbahagyta a nevetést. Mikor
észrevettem, hogy már egyedül nevetek, én is kinyitottam a szemem. Egy atka
maradványai hevertek mellettünk. Valami teljesen lecsupaszította a kitinvázat.
Lábaink alatt a kérget egysejtűek oszló sejthártyái borították. Nem tudtuk,
milyen véget érhettek ezek az apró lények, de nagyon megijedtünk. Lábujjhegyen
osonva igyekeztünk elhagyni a hasadékot, mikor hirtelen\dots

Mazsi meséje közben eltűntek a csillagok az égről és elmélyült a sötétség. A
legijesztőbb résznél villám cikázott az égen, és fényében rettenetesen kígyózó
alakot pillantottak meg Cseppcsurran főterén.

-- Á! -- sikoltott Mazsi és Filep, és ahogy hátrahőköltek a rém látványától,
lebucskáztak a kísérleti rakétatest tetejéről. A kiáltásra és dörgésre
felébredt a háza előtti padon alvó öreg manó is, és szemét meresztette a
sötétbe. A következő mennydörgésnél ő is meglátta a félig kígyó, félig manó
alakot, padját feldöntve bemenekült a házba és becsapta az ajtót. Egy
szemvillanással később az ablakot is.

-- Ez ijesztőbb, mint a nyálkagomba -- állapította meg Mazsi az egyik
hordóból.  Filep óvatosan visszakúszott a halom csúcsára és kidugta fejét, hogy
meglesse a rémesen tekergő formát. A felhők eltakarták a holdat és a
csillagokat, és minden fekete volt. De Filepnek jó szeme volt, és a különböző
dolgok kicsit különböző feketék voltak.

A tekergő forma egy nagy fonalféreg kellett, hogy legyen. A hátán három sötét
alak lovagolt. Az egyik egy nagy, erős manó lehetett, a másik egy kicsi,
törékeny manó. De kettejük között ült a legnagyobb alak. Mindenfelé formátlan
karok, csápok és karmok nőttek belőle. Tényleg ijesztőbb volt, mint a
nyálkagomba, de Filep bátran közelebb osont. Suttogást hallott a fekete lovasok
felől.

-- Láttad, Mazsi hogy lebukfencezett? -- kérdezte az egyik.

-- Igen. És Filep arcát láttad a villám fényében? -- kérdezte a másik
visszafojtott kuncogással a hangjában.

Filep rögtön megismerte Mária hangját, de nem akart egy jó ijesztgetésből
kimaradni. A sötétben bárki meglephetett bárkit. Mária és két titokzatos társa
a féreg hátán a tér közepe felé csúsztak, Filep pedig kis kerülővel a hátuk
mögé került. Fejére húzta üres hátizsákját és így ragadta meg a mit sem sejtő
Máriát.

Nagy sikítozás, birkózás és kergetőzés lett ebből. Mikor mindketten úgy
kimerültek, hogy nem tudtak már többször megijedni, és Mazsi is előbújt
hordójából, Mária bemutatta útitársait.

-- Ez Anakonda -- mutatta az örökké tekergő férget. -- Ez pedig János.

Az erős manó lemászott hátasáról, és egy vörösen izzó lámpást húzott elő. Fénye
bal szemét izzó ragyogásba borította, de jobb szemét fekete nemez borította.
Nagy orra alatt hosszú bajusz takarta mosolyát.

-- Nem a híres vitéz vagyok -- mondta, -- csak névrokona. Bocsánat, hogy
megijesztettünk. Mária ötlete volt.

-- Valakit meg kellett ijesszek azután, ahogy én megijedtem -- mondta Mária.
-- János a föld alatt él, és félelmetes vadász. Mikor találkoztunk, úgy féltem
tőle, hogy el is akartam rohanni. Pedig csak beszélgetni akart. Sok izgalmas
dolog esett meg vele a föld alatt, de ő az egyetlen manó ott, és nincs kinek
elmesélnie a történeteket.

-- És ki a másik útitársad? -- kérdezte Mazsi.

Mária előhúzta zöld szőrlámpását, és megvilágította az utolsó lovast. A torz
alak nem manó volt és nem is más parányi lény. Egy nagy halom agyag volt
mindenféle végtagok formájába gyúrva. Háncsszalagokkal kötözték Anakonda
hátára, és a zöld fényben minden félelmetes vonásának nyoma veszett. Valaki még
mosolygós arcot is formált neki.

-- Ő Gyuri -- büszkélkedett Mária. -- Eddig János nappalijának volt a fala,
de ma úgy döntött, fúvókának áll!

Ebből persze nem sokat értett sem Filep, sem Mazsi. Mária mindent elmesélt az
otthonukhoz vezető úton. Mária szülei házukba hívták Jánost, Filep szülei
asztalukhoz ültették Mazsit. A két kicsi házat a levél hegyében megtöltötte a
beszélgetés. A gyerekek elmesélték kalandjaikat, a kedves utazók pedig
összehasonlították a földalatti rögök és a távoli fenyőfa világát Cseppcsurran
zöld levelével. Lassan elhalványult a lámpások fénye, és mindenkiben megérett a
kíváncsiság a következő napra. Elköszöntek hát egymástól, és mindenki ágyba
bújt.

\secbreak

Filep fordult kettőt az ágyában, majd jobb ötlete támadt. Éjjeli szekrényéről
felkapta látókövét és kiosont a ház mögé. Lábát a mélybe lógatva a falevél
peremén ült már Mária is. Kölcsönadta neki a látókövet, és elgondolkodva nézte
az alattuk elterülő fekete leveleket.

-- Mária, te hova repülnél el legszívesebben?

-- Mazsi fenyőjére! Megígérte, hogy megtanít gyantatortát sütni.

-- De azután? Ha működik a rakéta, bárhová elrepülhetünk vele.

-- Ó, milyen igaz -- mondta most Mária is elgondolkodva. -- Van az a
málnabokor a nagy kő tövében. Tudod, ahonnan mindig finom illatokat fúj a szél.
Gondolod, hogy oda is elrepülhetnénk vele?

-- Hát persze! Vigyáznunk kellene leszálláskor, hogy ne akadjunk fenn a hegyes
tövisein. De a levelei biztonságosak. És onnan hová repülnél?

-- Vissza Cseppcsurranra. Egy egész bogyót elhozhatnánk és mindenki kaphatna
belőle.

-- Jó, de utána felrepülünk az öreg fa csúcsára a dombtetőn. Onnan olyan fákat
is látnánk, amiket innen nem lehet. És addig mennénk, míg elérjük az erdő
szélét.

-- És lehet, hogy ott találnánk egy másik erdőt. Ami csupa málnabokor. És
szederbokor.

Így tervezgették az utazásukat, míg el nem álmosodtak.

\secbreak

Filep azt álmodta, hogy kicsi volt. Olyan kicsi, mint egy szempillája. A
szobájában volt, és a szőnyege puha rostjainak rengetegében kalandozott.
Hirtelen besötétedett, és ahogy felnézett látta, hogy Mária ült az ágy szélén.
És Mária szobájában voltak, nem Filepében.

Filep felmászott a szőnyeg tetejére és egyik rostról a másikra ugrott. Pirosról
sárgára, sárgáról fehérre, fehérről kékre. És rájött, hogy ha jó sorrendben
ugrik a fonalakra, egyre világosabb lesz a cipője. Mária elővette a látókövet
és fordítva tartotta maga elé, hogy megvizsgálja a világító cipőket. Már csak a
cipők világítottak, és a parányi Filep árnyéka óriásként vetült a szoba zöld
falára. Olyan nagy volt, hogy tenyerébe tudta venni Máriát.

Filep, az óriás egy hangya hátán lovagolt. Aranysárga palástja volt és a hátasa
is aranysárga volt. De Mária egy gyíkot nyergelt fel, és a levél alatti síkon
szembeszállt az arany hangyák seregével. A gyík farkával tucatnyi hangyát
elsöpört, de aztán kibékült velük. Felsegítette a kis hangyákat és
nyeregtáskájából elővett málnabogyókkal etette őket. A hangyák cserébe
segítettek neki megépíteni egy hatalmas manóalakú gépet. Nagyobb volt a talpa,
mint egy egész levél. A cipőjét egy tucat hatalmas csigaházból építették meg.
Lábszára egy fatörzs volt és olyan magasra nyúlt, hogy a földön állva elérte a
fa alsó ágait. A nyakán pedig Filep mosolygó feje volt.

-- De mire jó ez a hatalmas gép? -- kérdezte Mária.

-- Fogócskázni! -- kiáltotta Filep és a gyíkra vetette magát. De a gyík fürge
volt és csak lehullott sárga és vörös leveleket markoltak az óriás kezei. A
leveleket a szeméhez tartotta, hogy meglesse, kik laknak rajtuk, de nem látott
semmit. A gyík megette a leveleket. Filep körül szaladgált és útjában
feltakarította az erdő padlóját. Ahogy felszippantotta az összes levelet, egyre
nőtt és nőtt. Mikor elég nagy lett, Filep is a hátára ült és a fák között
lovagoltak.

Nem sokára akkora lett, mint egy fa, és Filep a hátáról a fa tetejére lépett. A
fa legfelső levelén egyensúlyozott, és egy rakéta emelkedett fel a völgyből.
Fénye beragyogta az egész erdőt és az összes levél leesett. Filep is esni
kezdett, de a rakéta gyorsan irányt változtatott és elkapta estében. Akkora
volt, mint az óriásgyík és bonyolult csörlők, kötelek és karok borították.
Mária kinyitott egy csapóajtót, és beengedte Filepet. Felrepültek az égbe és
elrepültek a Napra és a Holdra. A Napon ősz volt és minden levél sárga volt. A
Holdon tél volt és fehér hó borította a fákat. Csak egy fa volt zöld, Mazsi
fenyője. És a kedves manó ott integetett nekik az egyik tűlevél oldalán vágott
kerek ablakából. Ügyesen leszálltak a küszöbére, de a Holdon minden fejjel
lefelé van, és amikor kiszálltak, Filep esni kezdett a Föld felé. Mária nem
értette, miért esik fel a Föld felé, és ivott egy teát Mazsival.

Filep csak esett és esett, és egyre kisebb lett. Mikor leesett a Holdról, még
akkora volt a talpa, mint Cseppcsurran. De mikor a fa ágai közé ért, már csak
akkora volt, mint egy hangya, és mire ráesett a háztetőre, akkorára összement,
mint egy manó. Pont beleesett az ágyába és felült. Anyukájától ő is finom teát
kapott reggelire.

\secbreak

Mazsi a főtéren ült egy kávézó teraszán. Kis kerek asztalát egy cellulózpapír
borította terítőként. Már lelógtak az asztal szélén a számítások, de tovább
járt kezében a ceruza. Egyik képletből következett a másik, és egyre pontosabb
képet kapott a különböző rakétatestek viselkedéséről. Kiszámolta, milyen erők
hatnak a test egyes részeire felszállás közben, ha így helyeznek el benne
gerendákat vagy úgy. Nagyobb súlyú test erősebb lehet, de a saját súlyát
nehezebben tartja meg. Mikor fiatal volt, az iskolában megtanulta, hogyan bírja
a terhelést a cellulóz és a kitin számtalan formája. Mérnöki munkája közben sok
megoldást megismert a szerkezetek megerősítésére, és most papíron kiszámította,
melyik hogyan szolgálna. Az előző esti sikertelen kísérletek után biztosabbnak
érezte papíron elvégezni ezt az elemzést.

-- Lássuk, hol lesz ez a derivált nulla -- motyogott, amikor Filep az
asztalához érkezett. -- Mindjárt kész leszek a számításokkal, Filep. A
rakétatest miatt nem kell aggódnod. Délre legkésőbb kész vagyok. Iszom még egy
kávét.

Filep nem akarta zavarni. Éjszaka esett az eső, és a cseppek elmosták a félig
megépített rakétát és az üres hordókat. A főtér közepén csak egy hatalmas
vízgömb ült. Mindenfelé szórta a felkelő nap fényét, de különösen bőkezű volt
egy paddal a főtér szélén. A pad körül árnyékot vetett a házakra, de amit a
házaktól elvett, azt mind a padnak adta. Mikor a nap felkelt, még csak a korán
kelő idős manók sütkéreztek rajta, de egyre több gyerek csatlakozott hozzájuk.
Végül az öreg manóknak megfájdult a füle a visítozó apróságoktól, és átadták
nekik a padot. Ahogy a Nap magasabbra emelkedett, a meleg folt lassan közelebb
húzódott a tér közepéhez. Le is kúszott a pad nagyobbik feléről, és a gyerekhad
megpróbálta utána tolni a padot. Egy szőke manófiú megragadta a lehetőséget egy
kis napozásra, míg a többiek erőlködtek, de ennek nagy kergetőzés lett a vége.

Mikor megbékéltek, Filep is segített a munkában, és egykettőre mindenki
kényelemben élvezhette a sűrített napfényt. Az erőfeszítésben megizzadtak, de
homlokukról az apró cseppek pillanatok alatt elpárologtak a melegben.

-- Ez az! -- kiáltott fel Filep. -- Itt könnyen lepárolhatjuk a mézsört!

-- A mit? -- kérdezte a szőke manófiú. De Filep már indult is a pásztor
manókhoz.

\secbreak

Cseppcsurran szélén Máriába és Jánosba botlott. Mária ült Anakonda hátán és
János a gyeplőt kezében tartva tanította a lovaglásra. Filep először látta
világosban a fonalférget. Bőre olyan sima volt, mint egy csepp víz, és élénk
piros. A nyerget ragadós gombafonalak tartották az izmos testen, és János
elmagyarázta, hogyan lehet Anakondát a fejére rögzített szerszámmal irányítani.

Mária után Filep is megtanulta, hogy kell megülni a szelíd férget. Közben
elmondta Máriának és Jánosnak, hogyan tervezi a főtér nagy cseppjének
gyújtópontjában lepárolni a rakéta üzemanyagát. Mária pedig elmagyarázta, hogy
fogja elkészíteni a fúvókát.

-- Két szelíd fonalféregre lesz szükségünk. Egyikük egy kis körben köröz, a
másik pedig körülötte egy nagyobb körben, ellenkező irányban. A nedves agyagot
kettejük közé töltjük és sima bőrükkel vékony falú gyűrűt formálnak belőle.
Ahogy több agyagot teszünk hozzá, a férgek szorítása kipréseli a nedvességet és
a megszilárduló agyagfal emelkedni kezd közülük. Mindkét férget gondosan
vezetjük, és lassan tágítjuk vagy szűkítjük a két kört. Így pontosan olyan
alakra tudjuk formálni a fúvókát, amilyet Mazsi megtervezett.

János és Filep elismerően összenéztek. Mária tényleg gondosan kieszelte a
fúvóka gyártásának módját. És ugyanígy készíthetnek majd nagy tartályokat,
hordókat is, ha akarnak. Több férget és elegendő agyagot használva akár
épületeket is.

Anakonda hátán felkerekedtek, hogy befogjanak néhány fonalférget a fúvóka
megformálásához. Az élénkpiros féreg még három manóval a hátán is szélsebesen
siklott az ág egyenetlen felszínén. A manógyerekek élvezték a vágtát, de rövid
idő múlva elérték céljukat.

A fa törzsének északi oldalán még ebben a magasságban is moha nőtt. A moha
szárainak kusza erdejében megrekedt a nedvesség, és a sötét víz nyüzsgött az
élettől. Medveállatkák lubickoltak, amőbák tornáztak, és az egyik nagyobb
cseppet egészen megtöltötte egy szúnyoglárva. A nedves göröngyökön vad
fonalférgek surrantak árnyékból árnyékba.

-- Rajokban mozognak, látjátok? -- mutatta János a férgek csapatát. A moha
alatti sötét, nyirkos világ az ő világa volt. Minden állatról eszébe jutott egy
izgalmas kaland, amit Filep és Mária lenyűgözve hallgattak. -- Ez a sárga
csíkos raj túl vad. Látod a fogaikat? Inkább letépné a karom, mint hogy a
hátára engedjen. Azok a fehérek ott a cseppben kezesek és nagyon jó úszók, de a
szárazon nem boldogulnak. De ott magasan a moha szárai között\dots Látjátok a
piros villanásokat?

A sötét szárak erdejét zöld fény töltötte be. A moha levelei a napsütésben
fürödtek és sehol nem látszott az ég kékje. Mária és Filep meresztette a
szemét, és bizony időnként felbukkant valami piros és mindjárt el is tűnt.

-- Azok Anakonda rokonai. Fürge, erős férgek. Nagyon szeretik az algát.
Gyertek, gyűjtünk egy kis algát. Ha csapdába ejtjük a férget és kézből
megetetjük, rögtön jámbor hátasunk lesz.

Egy világos cseppből manóökölnyi zöld gombócokat halásztak ki. Filep meg is
kóstolt egyet és jó húsos volt, csak egy kicsit kesernyés. János azt is
megtanította nekik, melyik gombafonalból tudnak jó szerszámot csinálni. Ez a
gomba veszélyes a vad férgekre is, mert ragacsos szálaiba könnyen
beleragadhatnak. Óvatosan feltekertek egy hosszú gombafonalat, és a moha
levelei felé másztak.

A piros férgek kerülték őket eleinte, de mikor meglátták a szép zöld
algagombócokat, nem tudtak ellenállni. A manók körül köröztek, és villámgyors
mozdulatokkal kikapták kezükből az algákat. János hurkot csomózott a gombafonál
végére, és a másik kezével maga előtt tartott alga mellé emelte. Mikor a féreg
lecsapott az algára, megrántotta a kötél másik végét, de a féreg vagy elkerülte
a hurkot vagy túl gyors volt. János meglazította a hurkot, és újra
próbálkozott. Másodjára szerencséje volt, és a ragadós fonál megszorult a féreg
derekán. János engedett a kötélből, és a féreg most már pórázon keringett
körülöttük.

-- És most óvatosan be kell húzzuk -- mondta János. -- Mindhármunkat könnyen
elhúzna magával, de a moha szára nála is erősebb. Ide hurkolom így -- mutatta.
-- És figyeld ezt a csomót. Így tudunk húzni rajta, de ha a féreg húz a másik
irányból, arra nem enged.

Rafinált csomója segítségével egyre közelebb húzta a férget. A raj többi tagja
megérezte a csapdát, és elillantak. A magára hagyott féreg ijedten próbált
szabadulni, de elfáradt, és megadta magát. Reszketve várta, mit hoz a sors.
Filep egy algát tartott az orra elé. A féreg bizonytalanul megszimatolta, majd
elfogadta. Filep megetette, és János tanácsát követve megvakargatta a hátát.
Ettől megnyugodott, és hagyta, hogy a hátára ragadt hurkot egyszerű nyereggé
alakítsák és kantárt kössenek a fejére.

Filep Rakétának nevezte el hátasát. Anakonda barátságosan megszaglászta, és
kicsit megnyugodva a fonalféreg megengedte Filepnek, hogy pórázon vezesse.
Kicsit karcsúbb volt, mint Anakonda, de így is több manóval felért az ereje.
Nem sokára már a hátára is hagyta Filepet felülni, és vadul cikáztak a moha
szárai között, míg Rakéta meg nem szokta a kantárt.

Mikor már jól megértették egymást, hozzáláttak, hogy Máriának is befogjanak egy
férget. Filep fürge hátasával segített János és Mária felé terelni egy piros
féregrajt.

-- Sok szerencsét! -- kiáltott Máriáéknak. -- Cseppcsurranon találkozunk.
Lehozom a mézsört, és mire elkészül a fúvóka, meglesz az üzemanyagunk is.

-- Sok szerencsét, Filep! -- kiáltották a piros férgek gyűrűjéből a Rakétán
távolodó manó után.

\secbreak

Filep lábai kicsit remegtek, amikor leszállt Rakétáról a pásztorok levelén. A
féreg sebessége ijesztő volt a kéreg hasadékaiban, de biztonságosan
megérkeztek. A levél manói megcsodálták a fényes piros hátast, és jókedvűen
köszöntötték Filepet.

-- Filep kapitány -- szólította meg Sándor, a mézharmat erjesztésének egyik
bajuszos mestere. Mazsitól vette át a kapitányi cím használatát, de hangjában
játékos megbecsülés csengett. -- Épp az imént vettem szemügyre a hordókat. Jó
büdösek lettek, megerjedt a mézharmat rendesen. Csak az a baj, hogy nem fogunk
tudni ennyi mézsört estig sem elmorzsolni.

-- Nem is kell -- mondta Filep, miközben Rakétát az istállóba kötötte. --
Azt gondolom, Cseppcsurranon le tudjuk párolni. Van egy nagy harmatcsepp a
levelünk főterén, ami összegyűjti a Nap fényét. Ha a gyújtópontjába teszünk egy
hordót, mindjárt felmelegszik, és felszáll belőle a szesz.

-- Mit szólsz hozzá, Sándor? -- kérdezte Sándor társát, akivel nem csak a
pálinkafőzés munkáját, de még nevét is közösen viselte. -- Ki lehet így
párolni a szeszt?

-- Működhet, ha közben kevergetjük a levet. Ha eltaláljuk a jó hőmérsékletet,
el fog tud válni a víztől a pálinka.

Tanakodtak még egy kicsit, de hamarosan azt tervezték már, hogy hogyan
szállítsák le a száz hordót. Fél tucat tetű hátán el tudták volna vinni, de a
hangyák birodalmán át ez veszélyes lenne. Filep nem tudta, hány hordót bírna el
Rakéta, de biztosan sokkal kevesebbet, mint egy tetű.

-- Ez aztán a fonalféreg! -- szólalt meg Gergő Filep háta mögött. Fejjel
lefelé lógott egy finom pókfonálon, és nyolc szemével rácsodálkozott Rakétára.

-- Jó reggelt, Gergő! -- üdvözölte Filep. -- Remélem, nem áztál meg az
éjszakai esőben.

-- Behúzódtunk az ág alá. Eső előtt mindig megtelik a háló élelemmel, úgyhogy
cseppet sem bánjuk a kényelmetlenséget. Édesanyám most főzi a finom ebédet.
Játszunk addig valamit?

-- Láttad, Szolperlán hogy lendítik a manók a nagy terheket a levél egyik
végéből a másikba pókfonálon? Játsszunk ilyet! Mi leszünk a manók, ezek a
hordók pedig a terhek. Cseppcsurranra kell levinnünk őket.

-- Cseppcsurranra? -- kérdezte Gergő lassan himbálózva.

Együtt kidolgozták a tervet. Cseppcsurran nem pontosan a legelő alatt volt, de
látszott a széléről. A levél egyik végébe halmozták a mézsörrel teli hordókat,
edényeket, terítőket és lepedőket, és Gergő takaros csomagba kötötte őket. A
levél másik végéből egy hosszú pókfonalat font a csomagig, és a laza fonalat a
levél peremén körben lelökdösték, hogy alul fusson.

-- Készen állsz, Gergő? -- kérdezte Filep. Együtt a levél széléig tolták a
szállítmányt, és már csak egy taszítás kellett az induláshoz.

-- Gyerünk, indulás! -- válaszolt Gergő. Még egyet löktek a csomagon, és
átbillentették a holtponton. A mézharmattal teli hordók és batyuk egyvelege
lassan lefordult a levél szélén. Ahogy fordult, Filep és Gergő felkapaszkodtak
rá, és mikor megfeszült a pókfonál, ők rajta hintáztak.

-- Jó irányban lengünk, Gergő! -- kiáltotta Filep suhanás közben. -- De még
nem éri el Cseppcsurrant a pályánk.

-- Még csak most indultunk el -- válaszolt Gergő. -- Figyeld csak, hogy
hintázik egy pók!

Gergő egyre több pókselymet toldott a szálukhoz. Mindig a hintázás
holtpontjánál adott a fonálhoz, így pályájuk egyre nőtt. Jó érzékkel még a
levegő mozgását is kihasználta, és úgy feszítette meg és lazította el a
szállítmány tartókötelét, hogy egyre gyorsuljanak.

-- Ez az! -- ujjongott Filep. -- Ott van alattunk Cseppcsurran!

-- Kikötésre felkészülni! -- mondta Gergő, és mikor következő alkalommal a
célpontjuk felett jártak, félelmet nem ismerve ledobta magát a szállítmányról.
Filep ijedten nyúlt utána, de mindjárt megértette, mit csinál az apró pók. Egy
új fonalat indított a lengő teherről, és mikor Cseppcsurranra érkezett, másik
végét a levél felszínéhez erősítette.

-- Kapaszkodj! -- kiáltotta onnan, de túl későn. Az új szál megfeszült,
megfékezte a csomag lengését a falu főtere felett, és Filep lebukfencezett
róla. A manó egyenesen a szikrázó harmatcseppbe esett volna, ha a szállítmányt
nem borítja pókfonalak gombolyagja. A szövevényben sikerült megkapaszkodnia, és
most a csomag alján himbálózott.

A több száz hordónyi mézsör szállítmány nagyobb volt, mint a harmatcsepp, és a
főtéren minden szem rajta függött.

-- Filep, ne ficánkolj -- szólt rá Gergő, és elkezdett felmászni. -- Ha
elszakad a fékező szál, ott lenghetsz egész nap. Én meg megtanulhatok úszni.

Filep szófogadóan abbahagyta a mozgolódást, és szomorúan lógott a csomagról.
Gergő megérkezett, és egy új szálat ragasztott egy hordóra. Leereszkedett
rajta, a légáramlatokat kihasználva elkerülte a cseppet, egy ház tetejéhez
csomózta a pányvát, és visszamászott rajta. Kihúzott így még három szálat, és
tökéletesen rögzítette a szállítmányt. Mikor Filep érezte, hogy már egy lepke
sem tudná leszakítani, átmászott egy pányvára és leereszkedett a levélre. A
pókok szerettek egész nap egy selyem szálon lógni, de Filep nem erre született.

-- Nem hiszek a szememnek! -- fogadta Mazsi. -- Ennyi pálinkával egy tucat
rakétát megtölthetünk, Filep. És milyen ügyesen ide lendítettétek\dots

-- Ez még nem pálinka, csak mézsör -- javította ki Filep. -- Ki kell még
párolnunk belőle a szeszt és összegyűjteni valahol. És az ott\dots Mária?

Mária úgy festett, mintha Gyuri, az agyagszörny elevenedett volna meg. Az első
fúvóka, amit építettek túl lágy volt, és amikor elkezdett beomlani, Mária
megpróbálta egyben tartani. Sajnos elölről kellett kezdeni, de nem vesztettek
mást, csak egy kis időt és annyi agyagot, amennyi Máriára tapadt.

-- Igen, én vagyok. Az pedig ott Nyár -- mutatott a készülő fúvóka körül
keringő agyaggal borított féregre, -- és odabent a társa, Bükk. Nagyon
óvatosan alakítjuk a formát, mert nem szabad egyenetlennek lennie. De nem
sokára készen leszünk.

-- Megnézhetem, milyen a fúvóka belülről? -- kérdezte az éppen földet érő
Gergő.

-- Nem -- vágta rá Mária. -- Nem segítettél Filepnek felmászni.

-- De hát kényelmesen lógott. Én egész nap úgy lógok.

-- De ő nem pók!

Gergő és Mária folytatták a vitát, míg Nyár és Bükk megformálta az agyagot.

\secbreak

-- Befejeztem a számításokat, Filep -- mondta Mazsi. Leültek a kávézó
teraszán mindent elborító jegyzetek közé. Mazsi egy lap híján mindent lesöpört
az asztalról. Ezt az egyet szépen kisimította és Filep felé fordította. --
Ezek az eredmények a különböző szerkezetekre. Sajnos nullánál kisebb számot
kapunk szinte minden építményre. Ezek szétesnének a levegőben. De ez az egy
megfelelő.

A számoszlopok mellett apró ábrák is mutatták, melyik szerkezet hogy fest.

-- Egy gömb! -- találta meg Filep a győztes változatot. -- Egy gömböt kell
építenünk?

-- És nem akármilyen gömböt. Csak hidegen kovácsolt kitin lesz elég erős. Ha
csak lemezekből szegecseljük vagy ragasztjuk, a kitin is darabokra hullana --
mutatott a számoszlop következő sorára.

-- Tudok is egy ilyen gömböt.

-- Micsoda? Ki épített ilyet?

-- Balcsutak, a hernyó -- mondta Filep.

Együtt felkerekedtek hát, és a király színe elé vonultak. Az esti eső
átnedvesítette a Folt halott felszínét, így ingoványos volt, de nem kellett
hirtelen repedésektől tartaniuk. Balcsutak levágott feje körül néhány fekete
baktérium rágicsált. Szétrebbentek, amikor Filep és Mazsi az ajtóhoz ért és
bekopogott.

-- Nohát, Filep! -- mondta György, mikor kinyitotta az ajtót. -- Nem
számítottam ma vendégre. Olyan pocsolyás a Folt. De gyertek be. Én György
vagyok, Cseppcsurran egykori királya.

-- Mazsi vagyok, Filep barátja. A tegnapi rakéta széllökése repített ide egy
távoli fenyőről. A gyerekek segítenek hazatérni. Mi is rakétát építünk, és
elrepülünk az én otthonomba.

György szája tátva maradt ennek hallatán.

-- Én is sokat utaztam fiatal koromban. Még a szomszéd fa egyik ágán is
jártam.  De a fenyőfák a völgy túloldalán vannak! Ha vissza nem fúj az a szél,
ami idehozott, nincs visszaút.

Mazsi csak mosolygott, és egy papírtekercset húzott elő. György az asztalához
ültette őket, főzött egy édes teát, és meghallgatta a nagy tervet. Bár nem
iskolában tanult, éles esze volt, és sok komoly kérdést tett fel. Mazsi örült
neki, hogy valaki segít leellenőrizni a számításait, és lelkesen elmagyarázott
mindent. György egyre szélesebben mosolygott, de legjobban annak örült, mikor
megtudta, hogy Balcsutak feje lenne a legjobb rakéta test.

-- Azt mondjátok, ki se kell tennem a lábam a házamból, és elrepülünk egész a
fenyőkig?

-- Azért pár átalakításra szükség lesz -- mondta Mazsi. -- Az alsó szintet
meg kell töltsük pálinkával\dots

-- Igen, feltétlenül. Ez nem akadály. Hol van most az üzemanyag?

-- A főtér felett lóg hordókban. De még le kell párolnunk.

-- Induljunk akkor gyorsan. Mazsit már biztos nagyon várják otthon.

Filep és György elindultak a lepárlást megszervezni. Mazsi a hernyó fejében
maradt, és hozzálátott a szükséges átalakításokhoz.

\secbreak

Cseppcsurran lakosai segítettek előző nap Mazsinak a nehéz lemezeket és
gerendákat rendezgetni. És érdeklődve figyelték Mária, a két féreg és az agyag
küzdelmét reggel óta. De eddig nem vették komolyan a munkájukat, és legtöbben
nem is tudták, mi a célja. Most egyre többen álltak meg a főtéren, és a
hatalmas mézsörszállítmányt szemlélve kérdezgették egymást.

-- Mi van azokban a hordókban?

-- Nem tudom, de onnan fentről hozták, a legelőről.

-- Akkor biztos mézharmat. De ennyi mézharmatot tavaszig sem iszunk meg\dots

-- Nem is meginni hozták! Elégetik mindet egy repülő házban.

-- Igen, azt hallottam a Holdra akarnak felszállni, mert ott örökké tél van.

-- Miért akarnak a Holdra szállni, ha ott örök tél van?

Így cserélt egyre több manó kérdéseket és feltételezéseket a főtéren. De
oldalba bökték egymást és elcsendesedtek, amikor meglátták Györgyöt. Pompásan
festett lila páncéljában, nagy fehér szakállával. Mosolyogva intett Máriának,
majd a manókat szólította meg.

-- Balcsutak, a hernyó megitta a vizünket, megrágta a házainkat és megmérgezte
a levelünket. A palotám helyén csak egy barna folt maradt. Ma mégis hálásak
lehetünk neki, mert egy igazi kincset hagyott hátra cserébe. A kiszáradt fejét.

Ezen a ponton kérdő tekintetek övezték Györgyöt, de kezét felemelve folytatta a
magyarázatot.

-- Nyakunkon az ősz, és ma vagy holnap Cseppcsurran is veszíteni kezd zöld
színéből. Sárgába és pirosba borul minden fa, és menedéket keres minden manó.
De egy távoli fán ezek a napok is olyanok, mint bármelyik másik. A fenyők
tudomást sem vesznek a télről. A manók gyantával szigetelik lakásaikat, és
tovább dolgoznak egész tavaszig. És Balcsutaknak hála ezen a télen nekünk is
lesz hol laknunk. Egy rakétát építünk a fejéből, és átrepülünk a völgy felett.
Hosszú pókfonállal kötjük össze a két fát, amin mind átkelhetünk. A fenyőt lakó
kékbőrű testvéreink ünnepelni fogják a mézharmattal és fűszerekkel megrakott
karavánjainkat, és megtanítanak minket lakásokat építeni a végtelen sok tűlevél
egyikében.

A manók közelebb húzódtak és reménnyel telt meg a szívük. Senki nem várta a
telet. A sötét és hideg minden élet ellensége volt, és keserű szenvedés volt
megvárni a tavaszt apró menedékhelyeiken.

-- És minek fognak legjobban örülni távoli rokonaink? -- folytatta György.
-- A pálinkának! Mazsi, akit nagy szerencsénkre a fenyőtől idáig fújt a
tegnapi szélvihar, elmondta, tetvek nem élnek meg a tűleveleken. Így a kék
manók ritkán ízlelnek édeset, és nem ismerik a szesz jótékony melegét. Csoda
hogy túlélik a telet. Főzzünk hát annyi pálinkát, amennyit csak tudunk, töltsük
fel Balcsutak fejét az aranyló folyadékkal, és még ma építsük meg a két fa
közti hidat! A pálinka lesz egyben a rakéta hajtóanyaga is, de ami megmarad az
út végére, azt a fenyő lakóinak ajándékozzuk.

Az ital a ritka ünnepek jelképe volt Cseppcsurranon. Maguk nem készítettek
pálinkát, és Szolperlától idáig hosszú volt az út. A kereskedők érkezése és
szállítmánya kiszámíthatatlan volt. Az ital említése így lelkesedéssel töltötte
el a főtéren összegyűlt manókat, és már csak arra voltak kíváncsiak, hogyan
lehetnek részesei a tervnek.

-- Filep, az apró felfedező már leleményesen ide szállította a pálinka
alapjául szolgáló mézsört, és kieszelte a lepárlás folyamatát is. Kérlek
benneteket, kövessétek az utasításait. Filep, hogy fogjunk hozzá?

Filep sosem volt még ilyen büszke, de akármennyire is kihúzta magát, csak azok
látták, akik az első sorban álltak. Felmászott hát a gyújtópontban álló pad
tetejére, és a ragyogó fényből kezdte irányítani az önkénteseket.

Első lépésként a gyerekek felmásztak a pókfonálon a hordókhoz. Alattuk
összegyűltek a felnőttek, és elkapdosták a ledobált hordókat. A kusza csomagból
az egyik kislány egy lepedőbe csomózott mézsör cseppet is ledobott, de ez
kinyílt amikor elkapták, és erjedt lével borította be a szerencsétlen
önkéntest. Jobban vigyáztak hát, és a batyukat láncban adták le a levélen
állóknak.

Ahogy fogyott a szállítmány, Gergő szorosabbra húzta a fonalakat. Mikor az
összes mézsör földet ért, egy pontba összecsomózta őket.

-- Cseppcsurran fölött egy kicsit keletre van egy vastag ág -- mutatta Filep.
-- A levelek már melegednek ilyenkor reggel, de az az ág még árnyékban van. Ha
alatta párologtatjuk ki a szeszt a hordókból, az le fog csapódni a hideg
kérgén. Gyertek, segítsünk Gergőnek megfeszíteni a szálakat, hogy az ág alá
forduljon Cseppcsurran!

Ahogy Filep Szolperlán látta, a pókhálót feszítve nekiláttak Cseppcsurran
elmozdításához. Puszta kézzel hiába erőlködtek a legizmosabb manók, nem tudták
úgy megrövidíteni a fonalat, hogy Gergő átköthette volna. János még Anakondát
is bevetette, de hiába.

-- A levél széle a faág alatt van. Nem lenne egyszerűbb oda hordani a mézsört?
-- kérdezte a véletlenül leöntött segítő. Ő volt Cseppcsurran pékje, de most a
szokásos finom zsemleillat helyett a cefre szaga áradt belőle, és mindenki
tartotta tőle a távolságot.

-- A harmatcsepp nélkül hiába visszük oda a sört, nem fogjuk tudni lepárolni
-- magyarázta Filep. -- Mi a könnyebb, a cseppet elvinni a levél szélére,
vagy valahogy megfeszíteni ezeket a pókfonalakat?

Tanakodtak egy darabig, de mivel egyik megoldásra sem volt pontos tervük, nem
tudták eldönteni, melyik az egyszerűbb. Sietnük kellett viszont, mert ha a nap
felmelegíti a faágat is, nem tudják hol összegyűjteni az üzemanyagot. Két
csapatot formáltak hát. Filep a Szolperlán megismert csörlők elvét igyekezett
bevetni a fonalak ellen, Peti, a pék pedig a cseppet próbálta meg elgurítani,
mint egy zsemlét.

Gergő váltig állította, hogy egy jó hálóban minden szál egyforma feszes, de
Filep mégis megvizsgálta mindegyiket. Még egy ház tetejére is felmászott, hogy
a tetőhöz ragasztott szálat megrángassa, és itt végre érzett egy kis
engedékenységet.

-- Biztos akkor lazult meg, amikor kirakodtuk a szállítmányt -- állította
Gergő.  Senki nem hibáztatta ezért, sőt boldogok voltak, hogy találtak egy
szálat, amivel talán el tudnak bírni.

A csapat egyik tagja hozott egy szőrszálat, amit egy hernyó hullajthatott el.
Sima, két ujjnyi vastag, erős rúd volt. Ha sikerül a pókfonalat köré tekerni,
elkezdhetik felcsavarni rá, és lassan a faág felé húzni a levelet. Addig
erőlködtek a háztetőn, hogy a szőrszálat a fonálba öltsék, míg Filep le nem
esett.

Anakonda megszaglászta a kiterült manófiút. Filep huncutul rávigyorgott és már
vitte is fel a gyeplőnél fogva a háztetőre. Erre már a ház lakója, egy csinos
fiatal hölgy is kijött a portájára, és onnan nézte őket derékra tett kézzel.

-- Hallottam, hogy valamit játszotok a tetőmön, de arra nem számítottam, hogy
egy fonalféreggel próbáltok egy kifeszített pókfonalat ráölteni egy hernyó
szőrére\dots

Pontosan így történt pedig, és egy kis küzdelem után sikerrel is jártak. Mind
megizzadtak, de az első tekerés után már nem kellett attól tartaniuk, hogy
kiugrik a szőrszál a fonálból. A hosszú rudat két végéről egyszerre négy manó
is meg tudta fogni, és minden erejüket beleadva csavarták a pókselymet. Minden
fordulat egy kicsit közelebb vitte őket a faág sötét ívéhez.

-- Működik! -- örvendezett Filep. -- Szólok Petinek, hogy nem kell
elgurítaniuk a cseppet.

Az ügyes pék nem járt annyi szerencsével, mint Filep. Az ő csapata épp
izgatottan szaladgált a harmatcsepp körül és próbálta kitalálni, hogyan húzzák
ki Petit a cseppből. A szegény manó odabent lebegett fejjel lefelé, és próbált
a társai felé evickélni.

-- Filep, mit csináltok? -- dörrent rájuk János mély hangja. -- Ez a manó
mindjárt megfullad! Hol van Anakonda?

-- A háztetőn -- mutatta Filep.

János nagyot nézett, majd hosszú füttyentéssel magához parancsolta a
fonalférget. Egy rövid füttyentéssel felmutatta a kezét és egy harmadik füttyel
Petire mutatott. Anakonda rögtön értette a feladatot. Hegyes orrával úgy kelt
át a csepp felszínén, mintha ott sem lett volna. Egy szempillantás alatt
Petihez úszott. Köré tekeredett, egészen befonta, és lendületet vett, hogy
kiússzon a cseppből. Az orrával most is ügyesen átlyukasztotta a felszínt, de
megakadt, amikor a manót próbálta kihúzni. Szerencsére ott termettek a manók,
és kantáránál fogva kicibálták mindkettejüket.

Peti sok vizet kiköpött, de nem esett baja. Nem győzött hálálkodni Anakondának.
Mikor teljesen magához tért, felpattant, és a pékségből algapogácsákat hozott a
megmentőjének.

\secbreak

Két manó megfogott egy hordót, a csepp gyújtópontjában összegyűlt fénybe vitte,
és ahogy Sándor javasolta Filepnek, alaposan rázni kezdte. A rázás miatt
egyenletesen melegedett a mézsör, és amikor elég meleg lett, pezsegni kezdett.
Kicsit felhabzott és bódító szagot árasztott a barnás lé, majd összeesett és
abbahagyta a pezsgést. Elvitték a napfényből, és a következő páros már érkezett
is a következő hordóval.

Filep és Gergő a Cseppcsurran feletti faág aljáról nézték a folyamatot. Gergő
éles karmaival meg tudott kapaszkodni az ág felszínén is, Filep viszont nagyon
óvatosan a kéreg egy lemezének a peremén ült és onnan lógatta a lábát. Az
árnyékban tényleg hűvös volt, és Gergő bánhatta, hogy még nem készült el a
télikabátja.

Miután a manók elkezdték a párologtatást, hamarosan az ágon is lehetett érezni
a szúrós szagot. De a hideg felülettel érintkezve a szesz lecsapódott és apró
gömböket formált. Filep nagyon figyelt, mégsem vette észre őket rögtön. A
harmat szemmel nem látható finom cseppeket formált. Ahogy a cseppek sűrűsödtek,
összeértek egymással és nagyobb cseppeket alkottak.

-- Gergő, nézd! Csupa arany gyöngy vagy!

-- És a te hajad is! Minden szála végén egy kicsi gyöngy van!

Mindketten összesöpörték a gyöngyöket, és egy ragyogó gömböt kaptak, ami épp
elfért Filep markában. Ahogy a kezében forgatta, alig láthatóan nőtt a gömb. És
a haján is új cseppek jelentek meg. A kéreg csúcsain is összegyűlt a pálinka,
és akkora gömböket formált, mint amilyet Filep a kezében tartott. Ahogy
hozzáérintette saját gömbjét egy kiszögellésen növekvőnek, egyesültek, és a
kéreg peremén máris egy nagyobb csepp remegett.

Mindkettejüknek tetszett a játék, de hamarosan rájöttek, hogy szükséges is amit
csinálnak. A kis cseppek elpárolognak, a nagy cseppek megmaradnak. Gergő kis
hálót csinált, és úgy mászott végig a kérgen, hogy két lábával addig húzta maga
után a hálót, míg egy jókora aranyló gombócot össze nem söpört. Filep a kéreg
belső üregeit járta és az ott összegyűlt gyöngyöket kergette ki. Összehordták
egybe a cseppeket, és ahogy a manók sorban párolták a hordókat, egyre nőtt az
ágon a legnagyobb csepp. Mikor nagyobb lett, mint Filep, Gergő a biztonság
kedvéért néhány laza fonállal megerősítette. Ez nem akadályozta a növésben, de
biztosította, hogy csak akkor cseppen le, amikor készen állnak.

-- Ennyi volt -- kiáltott fel György Cseppcsurranról. -- Ez volt az utolsó
hordó.  Mazsi azt mondja, szép nagy a csepp!

Csakugyan szép nagy volt. A szokatlan illatra előbújtak a fa teremtményei, és a
manók világának apraja és nagyja mind az aranyló cseppet nézte. Az árnyékban
gömbölyödő forma épp akkora lehetett, mint az alatta szikrázó harmatcsepp
Cseppcsurran terén. A lenti cseppet felfelé bámuló manók gyűrűje vette körül. A
kisebb utcákon is megálltak a manók és a szokatlan látványban gyönyörködtek.
Volt, aki még a háza tetejére is felmászott és onnan nézte a levél feletti
cseppet.

De a fa kérgéből is számtalan csillogó szem leste a kitartó munka gyümölcsét.
Pókok, atkák, medveállatkák, hernyók, lárvák és még felnőtt rovarok is
kíváncsian leskelődtek. A muslincák izgatott táncot lejtettek körülötte, és
szárnyaik zümmögése töltötte meg a levegőt.

-- Peti, kezdhetitek kiengedni -- kiáltott le Filep. -- Irány a második
állomás!

A háztetőn a pókfonál tekeréséért felelős csapat újra megragadta a szőrszálat.
Most az ellenkező irányba tekertek, de így sem volt könnyű a feladat. Ha nem
fogják erősen, az egész csomó kibomlik, és a levél kiszabadul.

-- És megérkeztünk -- szólt Filep. Az üzemanyag cseppje pontosan Balcsutak
feje fölött volt.

\secbreak

-- Megérkeztek -- szólt Mária a hernyó szeméből. A két hatalmas szem alatt
György napfényes olvasószobákat rendezett be. A kényelmes díványokat most
megsűrűsödött mézharmat ragasztotta az emelet felszínéhez. Mazsival ketten az
összes bútort és dísztárgyat rögzítették valamihez. Mária át is nevezte a
szobákat. Az olvasószobából pilótafülke lett. Az alsó szinten a konyhából
gépterem lett. Itt ügyködött éppen Mazsi a fúvóka beillesztésével.

-- Még egy pillanat -- mondta Mazsi. -- Még egyszer leellenőrzöm a
szigetelést, aztán feltölthetjük.

A fúvókát is mézharmattal ragasztották be. Mazsi végig kopogtatta a ragasztást,
és elégedetten látta, hogy jól megkötött. A fúvóka nyílását papucsállatkák
gyapjújával tömték meg, és odakintről egy hosszú kanóc vezetett hozzá. Mazsi
ezt is leellenőrizte, majd felmászott az emeletre.

-- Gyere Mária, jobb lesz ezt kívülről néznünk.

Az emeletről egy sebtiben felácsolt lépcsőn leszaladtak a Folt barna
ingoványára, elhúzódtak a nagy koponyától, és ugrálni és integetni kezdtek.

Erre a jelre vártak odafent. Hogy egyenesen essen a drága csepp, kivárták, míg
egy pillanatra teljesen megállt a levegő, és két oldalról egyszerre kioldották
a pókfonalakat. A csepp nyúlni kezdett, majd elvált a kéregtől és egy
pillanattal később elöntötte a hernyó átalakított fejét. A tökéletes célzásnak
hála az üzemanyag szinte mind a helyére került. A tetőn vágott nyíláson át
megtelt a felső emelet, majd a lépcsőházon át lefolyt a gépterembe.

Persze azért Mária és Mazsi is kaptak a szétfröccsenő főzetből. Visítozva
ugrándoztak és rázták hajukról az illatos cseppeket. Mazsi megállt egy
pillanatra, megnyalta a szája szélét és elégedetten bólogatott.

Filep és Gergő egy fonálon leereszkedtek közéjük, és eljött a búcsú ideje.

-- Mindjárt kész az ebéd -- mondta Gergő. -- Édesanyám hálófogságra ítél, ha
nem leszek otthon időre. Itt ez a fonál. Ragasszátok a rakétára, és mi
gondoskodunk majd róla, hogy bármeddig is menjetek, ne szakadjon el. Köszönöm a
játékot, és jó utat!

-- Jó utat, Filep -- mondta édesanyja. -- Nagyon rendes vagy, hogy segítesz
Mazsinak. Érezd magad jól a fenyőn, de siess azért haza!

-- Jó utat, Mária -- mondta édesapja. -- Micsoda munkát végeztél! Tanulj
sokat Mazsitól. Mamuszt vidd magaddal, ő vigyáz majd rád.

-- Jó utat, Mazsi -- mondta János. -- Remélem, hamarosan újra találkozunk,
és megosztjuk a legújabb kalandjainkat egymással!

-- Gyerünk, beszállás! -- mondta György. -- Már átitatta a tömítést az
üzemanyag.  Indulnunk kell.

Beszállás előtt még integettek a rakéta körül összegyűlt manóknak, de aztán
mindenki visszavonult a Foltról, nehogy megpörkölje őket a hajtómű tüze. A Folt
szélén meggyújtották a kanócot, és kéz a kézben várták a felszállást.

-- Tíz, kilenc, nyolc, hét, hat, öt, négy, három, kettő, egy\dots

A tűz elérte a rakétát, és mély morajlás hallatszott Balcsutak fejéből. Lassan
felerősödött a hang, és vakító fény öntötte el Cseppcsurrant. Mindenki
fedezéket keresett a levélen, és a fejen belül megszorították a kötéseket az
utazók. A rakéta hirtelen lökéssel útjára indult.

A tűz kiszárította a Foltot és a közepét meg is kormozta, de nem tett nagyobb
kárt semmiben. Egyre gyorsabban emelkedett az ágak között, és egy pillanattal
később már csak egy pontnyi ragyogó fényt láttak a levélről. A ráragasztott
pókfonál mintha az égbe szökött volna.

-- Repülünk! -- ujjongott mindenki Balcsutak fejében.

-- A röppályánk is jó, csak három fokot kell balra kanyarodnunk -- mondta
Mazsi, ahogy izgatottan összehasonlította számításait és méréseit. György
megtekert egy kart, ami hajszálnyit más irányba fordította a fúvókát.

Filep és Mária a pilótafülkéből nézték az alattuk kibontakozó tájat, és
elteltek a felfedezés örömével.

\begin{center}
Vége
\end{center}

\cleartoverso
\cleartorecto

\end{document}
