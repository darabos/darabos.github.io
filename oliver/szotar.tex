\documentclass[a4paper]{memoir}
\usepackage{charter}
\usepackage[utf8]{inputenc}
\usepackage{t1enc}
\usepackage[hungarian]{babel}
\usepackage[final]{microtype}
\usepackage{multicol}
\usepackage{ifthen}
\isopage[12]
\checkandfixthelayout

\title{Molmi Holmi}
\author{Olivér--magyar szótár}
\date{}

\newcommand{\ol}[1]{\textbf{#1}} % Olivér formázás.
\newcommand{\e}{\hangpara{1.5em}{1}} % Szótár bejegyzés.
\newcommand{\oliver}[1]{\e \ol{#1} \hspace{1ex}} % Olivér bejegyzés.
\newcommand{\oli}[1]{\hspace{1ex} \ol{#1} \hspace{1ex}} % További Olivér kifejezés.
\newcommand{\magyar}[1]{\ifthenelse{\equal{#1}{1}}{}{\hspace{1ex}\ }\textbf{#1.}} % Magyar jelentés száma.
\newcommand{\pelda}[1]{\textit{``#1''}} % Példa mondat.

\begin{document}

\maketitle

\section*{2018 február}

\begin{multicols}{2}

\oliver{á} evés

\oliver{bémbé} BMW

\oliver{busz}

\oliver{cecece} Mercedes

\oliver{cici}

\oliver{cincin} egér

\oliver{cica} emlős állat

\oliver{csücsü} leülni

\oliver{emmi} ez mi

\oliver{Kia}

\oliver{krokodil}

\oliver{kuku} kukucs

\oliver{Lilli} Félix

\oliver{nínó} autó

\oliver{olló} Volvo

\oliver{pusse} Porsche

\oliver{sziattó} sziasztok

\oliver{Tesla}

\end{multicols}



\section*{2018 március}

\begin{multicols}{2}

\oliver{aú} hajó

\oliver{arra}

\oliver{ádi} Audi

\oliver{be-e-e-e} bárány

\oliver{Bentley}

\oliver{bimbi} pingvin

\oliver{bamba} lámpa

\oliver{buszedli} busz

\oliver{cicije} cici

\oliver{cicikli} bicikli

\oliver{cincia} egér

\oliver{dendő}
  \magyar{1}~rendőr
  \magyar{2}~mentő

\oliver{débu} Daewoo

\oliver{én is} én is kérek

\oliver{fiá} Fiat

\oliver{Honda}

\oliver{kaka}
  \magyar{1}~kacsa
  \magyar{2}~sapka

\oliver{kinkankó} pillangó

\oliver{kukac}
  \magyar{1}~kukac
  \magyar{2}~csiga

\oliver{Lilix} Félix

\oliver{mamaó} Alfa Romeo

\oliver{ó} Ford

\oliver{Oli, Anya, Apa, Lilix} Citroën

\oliver{ondó} motor

\oliver{ope} Opel

\oliver{ótiti} Volkswagen

\oliver{páus} pelenka

\oliver{márt} Smart

\oliver{mini} Mini Cooper

\oliver{miau} cica

\oliver{olló}

\oliver{sisa} sisak

\oliver{Skoda}

\oliver{szia} \textit{(ezt mondja a matchbox)}

\oliver{Tádé}

\oliver{táó} mesefilm (``Tayo'')

\oliver{tátáé}
  \magyar{1}~teteje
  \magyar{2}~kabrió

\oliver{tásszusz} Lexus

\oliver{tetti} Mercedes

\oliver{toji} Toyota

\oliver{vauvau} kutya

\end{multicols}

\end{document}
