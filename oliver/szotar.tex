\documentclass[a4paper]{memoir}
\usepackage{charter}
\usepackage[utf8]{inputenc}
\usepackage{t1enc}
\usepackage[hungarian]{babel}
\usepackage[final]{microtype}
\usepackage{multicol}
\usepackage{ifthen}
\isopage[12]
\checkandfixthelayout

\title{Molmi Holmi}
\author{Olivér--magyar szótár}
\date{}

\newcommand{\ol}[1]{\textbf{#1}} % Olivér formázás.
\newcommand{\e}{\hangpara{1.5em}{1}} % Szótár bejegyzés.
\newcommand{\oliver}[1]{\e \ol{#1} \hspace{1ex}} % Olivér bejegyzés.
\newcommand{\oli}[1]{\hspace{1ex} \ol{#1} \hspace{1ex}} % További Olivér kifejezés.
\newcommand{\magyar}[1]{\ifthenelse{\equal{#1}{1}}{}{\hspace{1ex}\ }\textbf{#1.}} % Magyar jelentés száma.
\newcommand{\pelda}[1]{\textit{``#1''}} % Példa mondat.

\begin{document}

\maketitle

\section*{2018 február}

\begin{multicols}{2}

\oliver{á} evés

\oliver{bémbé} BMW

\oliver{busz}

\oliver{cecece} Mercedes

\oliver{cici}

\oliver{cincin} egér

\oliver{cica} emlős állat

\oliver{csücsü} leülni

\oliver{emmi} ez mi

\oliver{Kia}

\oliver{krokodil}

\oliver{kuku} kukucs

\oliver{Lilli} Félix

\oliver{nínó} autó

\oliver{olló} Volvo

\oliver{pusse} Porsche

\oliver{sziattó} sziasztok

\oliver{Tesla}

\end{multicols}



\section*{2018 március}

\begin{multicols}{2}

\oliver{aú} hajó

\oliver{arra}

\oliver{ádi} Audi

\oliver{be-e-e-e} bárány

\oliver{Bentley}

\oliver{bimbi} pingvin

\oliver{bamba} lámpa

\oliver{buszedli} busz

\oliver{cicije} cici

\oliver{cicikli} bicikli

\oliver{cincia} egér

\oliver{dendő}
  \magyar{1}~rendőr
  \magyar{2}~mentő

\oliver{débu} Daewoo

\oliver{én is} én is kérek

\oliver{fiá} Fiat

\oliver{gangon} SsangYong

\oliver{gogogó} robogó

\oliver{Honda}

\oliver{ita} Nissan

\oliver{kaka}
  \magyar{1}~kacsa
  \magyar{2}~sapka

\oliver{kakaó} ital

\oliver{kankó} SsangYong

\oliver{kinkankó} pillangó

\oliver{kukac}
  \magyar{1}~kukac
  \magyar{2}~csiga

\oliver{Lilix} Félix

\oliver{mamaó} Alfa Romeo

\oliver{ó} Ford

\oliver{óbembe} hóember

\oliver{Oli, Anya, Apa, Lilix} Citroën

\oliver{ondó} motor

\oliver{osziszi} Suzuki

\oliver{ope} Opel

\oliver{ótiti} Volkswagen

\oliver{páus} pelenka

\oliver{márt} Smart

\oliver{mini} Mini Cooper

\oliver{miau} cica

\oliver{olló}

\oliver{oóoó}
  \magyar{1}~vonóhorog
  \magyar{2}~Land~Rover

\oliver{sisa} sisak

\oliver{Skoda}

\oliver{szia} \textit{(ezt mondja a matchbox)}

\oliver{Tádé}

\oliver{táó} mesefilm (``Tayo'')

\oliver{tapad} gepárd

\oliver{tát} Seat

\oliver{tátáé}
  \magyar{1}~teteje
  \magyar{2}~kabrió

\oliver{tásszusz} Lexus

\oliver{tetti} Mercedes

\oliver{toji} Toyota

\oliver{ungenge} úthenger

\oliver{vauvau} kutya

\end{multicols}

\section*{2018}

\begin{multicols}{2}

\oliver{Bambi, bambi, bóhihi} jókedvű dalocska

\oliver{bububékát}

\oliver{bummbele játszani} játszani

\oliver{cicianya}

\oliver{"Danone" Ferrari}

\oliver{fikukuku} kipufogó \pelda{Anya, homman a fikukuku a Bububishinek?} Anya, hol van a kipufogója a Mitsubishinek?

\oliver{funka} fánk

\oliver{hi-hü-hü} vonat

\oliver{misma} csizma

\oliver{mungáj} kormány

\oliver{ojeoje} Land Rover

\oliver{ombembe} hóember

\oliver{ongás} Jaguar

\oliver{padzag} Mazda

\oliver{pelelő} repülő

\oliver{pumba} lámpa

\oliver{szopér} pucér

\oliver{zazazu} Subaru

\end{multicols}

\oliver{Anya képzeld el: gepárd!} meseköltés Félix mintájára, röviden

\oliver{Az a szép, az a szép, akoni akoni.}

\oliver{Dédöjö mami mami} lelkes dalocska

\oliver{Látom Olit!} tükörben

\oliver{Nincs teteje Bentley autó. Paprió.}

\section*{2019}

\oliver{Megolvasod ezt?} Elolvasod ezt?

\oliver{Ez nem jó nekem.}

\oliver{Nem kérem Apát!}

\oliver{Én vagyok a fiad.} fontos közlendő

\oliver{Vászkodo vandi swish swish swish.} ``The wheels on the bus go [...] swish swish swish.''

\oliver{Te tütűdjél!} dudálj

\oliver{Be jöhet menni?} Be lehet jönni?

\oliver{Ki megette?} Ki ette meg?

\oliver{jeleje} jel az utó elején

\oliver{az én anyám}

\oliver{az én apám}

\oliver{Tyotyi a kukim}

\oliver{Ez régen lámpa volt, most már út.} a "rámpa" szó értelmezése

\oliver{jöjjél ide} gyere ide

\oliver{allagvé} nem jöttünk rá

\oliver{Ez olyan finom, hogy meg se tudom enni.}

\end{document}
