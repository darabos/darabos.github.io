\documentclass[10pt]{memoir}
\usepackage{pifont}
\usepackage{charter}
\usepackage[utf8]{inputenc}
\usepackage[final]{microtype}
\clubpenalty=10000
\widowpenalty=10000
\raggedbottom
\setstocksize{6.87in}{4.25in}
\settrimmedsize{6.87in}{4.25in}{*}
\isopage[12]
\checkandfixthelayout
\setlength{\emergencystretch}{2em}

\renewcommand{\pfbreakdisplay}{\bigskip \ding{166} \bigskip}
\newcommand{\secbreak}{\fancybreak{\pfbreakdisplay}}
\newcommand{\sunsong}{
  Shine your light on your cold creation. \\*
  The sky above the clouds is filled with your illumination. \\
  Shine your light! A precious ray in winter, \\*
  Distilled to a drop of delight, it is my heart's tinder. \\
  Shine your light somewhere else in the nighttime. \\*
  The patient spirit waits for you and holds on to your lifeline. \\*
  Shine your light and I will shine my life.
}

\author{Daniel Darabos}

\date{2014}

\title{The Tree Far Away}

\begin{document}

\begin{titlingpage}
  \centering
  \vspace*{0.2\textheight}
  {\Huge \thetitle}\\[\baselineskip]
  {\large\itshape by \theauthor}\\[\baselineskip]
  \vfill
  {\itshape Written in 2012. \par
  Translated in 2015 \par from Hungarian to English. \par
  Printed in February 2015.}
  \vspace*{0.1\textheight}
\end{titlingpage}

\vspace*{0.4\textheight}
\noindent
Philip lived with his parents in a small house. His parents had built their
house at the very end of a young leaf. The tip of the leaf swayed the most when
the wind blew, so his mother asked Philip every day to be careful. But on quiet
evenings they all sat together at the edge of the leaf and dipped their feet
into the endless abyss.

Thousands of leaves stretched out below them.

``Is every leaf like ours?'' Philip asked his father.

``On this twig, yes. But on the branch beneath us the fairies speak a foreign
language. Above us they knit their houses on the bottom of the leaves because
it is cold and the surface is more slippery. Each branch is different. There
are places where fairies have never been.''

\secbreak

Philip became an explorer the very next day. He packed a dozen pollens, two
folded fruit fly wings, and a blank travel diary and set off toward the stalk of
the leaf.

The leaf where Philip lived was called Dewington. At every sunrise beads of dew
grew on the roofs of the houses. When they grew as large as a pollen, they
tumbled into the gutters. There they merged into spheres the size of a fairy's
head and bounced off onto the veins of the leaf. The veins were the road
network of Dewington, and every morning hundreds of dew spheres rolled across
them. The kids played ball with the spheres while the adults got annoyed
whenever they knocked down a bench or a pot. By the time the sun had wholly
risen above the horizon, the water droplets all flocked to the center of the
leaf and merged into a giant sphere. The sphere sparkled in the light of the
new day and all the inhabitants of the leaf made a stop at Dewington's main
square in time to draw from the fresh dew water.

Philip too was headed for the main square. The chitin shop was just opening up
as he reached the corner of the main street. He took a longing look at the
wares. Among the shiny blades, thin needles, and hard plates in the window
today there was an armor fashioned of radiant scales. It would have been
perfect if he went hunting, Philip thought, but the hard suit would have just
constrained him as an explorer.

``Stop it!'' someone shouted from the right. As soon as Philip turned in the
direction of the desperate voice, a dew ball the size of a loaf of bread hit
him in the face. He reached after the bounding drop but lost his balance and
fell over onto the leaf.

Mary ran past, helped him up, and they chased after the drop of water together.

``Babouche is in this drop!'' the fairy girl cried and pounced on the drop. She
reached it but could not grab it from both sides, and the ball bounced from her
hands. Philip saw now why they were chasing after it. Mary's paramecium was
trapped in the ball of water. Parameciums are good swimmers, but if the small
drop were to merge into the large one in the main square, Babouche would have
to spend the rest of the day there.

The drop glided smoothly over the vein of the main street. Philip was a good
runner and able to keep up with it, but he knew it would not be easy to stop
the ball. There is nothing more slippery than the surface of water droplets. If
they could not squeeze it in somewhere, they would never pull Babouche out of
it. ``We have to get it onto the Stain!'' Philip shouted and shoved the drop
into an alley.

The ambience here was nothing like on the main street. There were no inviting
shop windows, no fresh scents drifted from restaurants, and the drop on the
main square towering above the houses did not fill this street with its
radiance. The murmur of life faded away behind them. As they chased the drop of
dew through the passage they could soon smell the Stain ahead.

They ran past the last houses and found themselves in an endless brown desert.
The withered leaf creaked and crackled loudly beneath their feet, but they did
not want to lose Babouche now. The drop started to slowly shrink. The dry, dead
surface gradually soaked up its moisture.

A dried flake collapsed under Mary's foot. She fell, and more flakes broke under
her, outlining her fallen shape. Philip turned back and took an anxious step
toward Mary.

``Run on! Babouche is headed for the middle of the Stain!'' Mary said.

``We cannot run on. The leaf is entirely dead here. We have to be very
cautious,'' Philip said and helped her up. They pushed on more carefully now.
They still saw Babouche and also that the drop was almost completely gone. Once
the dew no longer covered him, Babouche tumbled to a stop. Still dizzy, he
righted himself and sniffed around. He thoroughly shook his hair and nodded
toward Mary.

``Do not move!'' a voice boomed from the distance. The two little fairies and
the paramecium looked anxiously to the center of the Stain. From the distance a
caterpillar's head stared back at them.

\secbreak

Every child knew the story of the Stain. Months before, Balthazar, the
caterpillar, attacked Dewington. Knocking over houses on his way, he barged
into the main square. The fairy warriors fought bravely, but his hard skull
had protected him. The caterpillar headed from the main square to the most
beautiful part of the city. He waded through well-kept parks, pierced the
cellulose shop windows, and swallowed barrels full of nectar. He tore through
the silk walls of the royal palace and dragged his soft body toward the
treasury.

The palace staff panicked and fled, but George, King of Dewington, received
Balthazar in the throne room.

``This leaf belongs to the fairies. If you would like to leave alive, turn
around now and go back to whence you came,'' George said and lifted the royal
sword off the wall. This weapon was forged against the larger inhabitants of
the tree. Only the strongest of the fairies were able to wield it. The shiny
blade was many times George's height. The tip almost reached the ceiling of the
palace.

``I will be leaving,'' the caterpillar hissed, ``once I have fed, pupated and
grown wings!''

He swung his head and snapped his mandibles toward George. The king raised his
sword to block the attack. The chitin clashed against the metal, and sparks
poured onto the floor of the palace. The strength of the caterpillar threw
George toward the end of the hall, but he managed to grab a column and change
his direction. He flanked Balthazar, cut down with the blade and severed the
head of the caterpillar.

George sidestepped the aimlessly twisting body and looked into the eyes of the
huge head. The rage of the short battle still burned in his veins. He was not
only angry at Balthazar for what the caterpillar had destroyed, but also for
what George had to destroy because of him. The dying caterpillar would never
become a butterfly. At this thought the king was consumed with guilt. The hilt
of the sword slipped from his sweaty palms and banged on the floor of the
palace.

As in all fairy homes, the floor of the palace was the leaf itself. Industrious
fairies had sewn patterns into the thin layer of wax that covered the plant
cells in the throne room. The heavy sword lay across the path of an ant caravan
embroidered in golden thread. George's eyes followed the line of ants to
Balthazar.

``No!'' George shouted. But even if the caterpillar had a drop of goodwill in
him, he had not enough life left to heed the king's command. His mandibles had
dug into the decorated leaf and a bitter venom was seeping into the wound. The
leaf turned black around the wound and the dark stain started to grow.

George stumbled back. The black venom filled the floor of the palace right in
front of his eyes. He watched helplessly through a window as the streets around
the palace were painted black one by one. He felt as though the caterpillar had
poisoned his heart and the bitterness had spread through his body. The streets
turned black, the buildings turned black and the sky did too.

\secbreak

Many days had passed since then. The one-time king now served hot tea to his
guests. Mary and Philip sipped quietly from the steaming cups and looked at the
wasteland through a round window. George's voice had stopped them from chasing
after Babouche onto a dangerously fragile region. Weeks earlier, the palace
stood here, and here fought Balthazar and George. The caterpillar injected his
poison into the leaf right here, and those who fled from him never returned to
the blackened streets. Thankfully, there was not enough of the poison to kill
the entire leaf. But within the Stain the leaf withered and the houses
collapsed.

The caterpillar's body had dried up and drifted away. The head, however, still
clung to the dead leaf. The wind took the palace as well, but the king could
not leave the Stain behind. He took up residence inside Balthazar's dried head,
and they looked out through his eyes over the wasteland.

``How could the Stain be healed?'' the ever-curious Philip asked.

``No such magic, unfortunately,'' the old king replied. ``But the spring will
heal all.''

``We are big kids now,'' Mary interjected. ``We know the spring is not real. It
is only our parents who decorate the house in the morning.''

``Oh, you have really grown up! Philip may indeed be swinging the royal sword
come tomorrow! But I am sure you have noticed that the days are shorter now
than when you were small. You can play less and have to learn more.''

The two youngsters nodded ruefully.

``With time the days run out, the leaves wither, and the fairies hide. It will
be your duty to emerge when the spring comes and rebuild everything. Your leaf
will grow as you build your town, and you will play spring for your little
fairies.''

Philip and Mary listened to George's prediction. The adults always talked about
the seasons. But the kids could not yet decide whether they really existed or
not.

They accepted another cup and listened to a story about how George had bought
the leaf of Dewington from the ants for the price of seven aphids. They said
goodbye then, and the king taught them the safe path through the cracked
surface of the Stain.

``Thank you for your help,'' Mary said on their way back. ``Thanks to us,
Babouche is not stuck inside the drop!''

``I am glad to have helped.''

``If you want to help some more, let us fetch a little water for my fungus.''

\secbreak

The main square was alive with a colorful crowd of fairies. Most came to fetch
water from the huge drop of dew. But many had stopped for a chat, vendors
offered their wares, and old fairies bathed in the light of the rising sun
where the big lens of the drop focused it into a spot.

Mary formed a capillary from her hands as well as any adult. She touched it to
the surface of the drop, and Philip hurried to pick the outflow of dew
dumplings into his bag. One droplet tried to roll away, but Babouche sipped it
up.

Mary was growing a tiny fungus in her room. It was a present from her aunt, and
she wholeheartedly believed it would turn sweet as honey with just a little
more tending. Only someone who never tasted it could have believed that, Philip
thought. He had tried a little tendril the last time they came over for lunch.
It was terribly tough and bitter. Since then he avoided the oddly shaped germ.
He excused himself on account of his important work as an explorer and said
farewell to her.

\secbreak

Philip came to the stem of the leaf. He looked back one last time at the drop
shining among the fairy buildings and knew this was his moment. He was about to
become an explorer. He jumped on the stalk and slid into the unknown on his
backside.

He remembered her grandmother's stories about the ground. The old
fairy had told him that even the ants were afraid to climb down that far
because there were monsters there that could swallow a whole ant. Fortunately,
these creatures were too large to walk upon the leaves, so the fairies had
nothing to fear. But what if a little fairy left the leaves?

The leaf grew from a young twig. The stalk had big green bumps at its base to
hold it firm. These bumps cushioned Philip's arrival. Although he had no choice
in the matter, he decided he was better off with three small thumps than a
single big one. Fearing his grandmother's monsters, he slid into a fairy-sized
cranny and peeked out from there. He saw monsters neither to the left nor to
the right.

The twig was about the width of a whole block of buildings and stretched to the
horizon in both directions. Its surface was much rougher than the leaf. A fairy
could always find a niche here that would serve as emergency shelter. This put
Philip at ease as he climbed out of the narrow crack. He circled the stem and
thought about how he would climb back up at the end of the expedition. The stem
was covered in a strong wax. Pants slipped well on it, but small fairy fingers
would pierce it and find a good grip.

Philip saw which way the stalk of the leaf leaned and walked off in that
direction. He looked up curiously. He had spent his entire life on its other
side, but had never before seen the bottom of the leaf. He could see the veins,
and the brown circle of the Stain was clear as well. Philip unwrapped his
travel diary and made a drawing of what he saw. He had hardly seen a prettier
map of Dewington!

After a bit of a walk he glimpsed a pair of antennae bobbing above the
curvature of the branch. He had nothing to fear from insects on Dewington. But
Philip did not know what to expect in the wilderness outside of the leaf. He
watched the approaching antennae from a crevice. The antennae undulated on the
great black head of an ant, and a body followed behind. A bit later there was
another ant head and another ant body. Three ants were approaching Philip.

The ants were regular guests in Dewington, although the fairies did not always
understand them. Shiny chitin covered their huge bodies, stronger perhaps than
even the royal sword. The leaf trembled beneath their feet. Their mandibles
could have carried all the houses of the main street at once. Yet they never
caused trouble. Each day an ant would come, walk around the leaf without a
word, taste the dewdrop, and return to where he came from. Philip's parents
said the ants were looking for garbage, and therefore everything had to be kept
clean. Mary had heard they were looking for bad kids. And yet, Philip was not
taken when he misbehaved.

Philip felt that he had not been bad that day. In fact he had been good. He
stood in front of the ants and awaited them with his hands on his hips. The
ants seemed big from afar, but they just grew and grew as they approached.
Philip could have hidden behind one leg.

The ants did not notice Philip, or they were in a great hurry. The heads,
thoraces, and abdomens the size of houses passed high over him. He spun around
in surprise below.

``Wait! I am headed the same way! Let's go together!'' he shouted and waved,
but the ants did not wait. Philip ran after them and pounced on one of the huge
legs. He held on tight as the foot rose into the air. ``Wait, hold on! I am
going to fall!'' he pleaded with the ant. But it was either deaf or did not
speak the language of the fairies. Philip was a skilled climber, of course, and
with some difficulty managed to reach the ant's knees. ``Why such a hurry? Is
the winter chasing after you?''

Philip walked over the ant's thigh, cut across the jungle of hair on the
thorax, and found his way to the ant's head. ``Here we go,'' he said as he
settled between two hairs perfect for holding onto. ``We can go on now.''

``I am Philip, the explorer. You must have been to many places, Mr.~Ant. Have
you been to the top of the tree? It must be quite the view. Maybe you can see
beyond the end of the longest branch. Although, I do not know how well you ants
see. Perhaps you do not even see me? You do not hear me. Not even when I
sing?''


\begin{verse}

Knock, knock, in the night. \\*
\vin ``Will you let them in?'' \\*
It's an alga, dry but green, \\*
\vin Of course I let them in. \\
Sip, sip, water up, \\*
\vin You need the moistening. \\*
Send a postcard when you can \\*
\vin And meet us in the spring!

Knock, knock, in the day. \\*
\vin ``Will you get the door?'' \\*
It's the alga, round and green, \\*
\vin Who rolls onto our floor. \\
Knead, knead, make a dough, \\*
\vin Bake an alga cake. \\*
Just be sure to save the spores \\*
\vin And plant them in your wake.

\end{verse}


At the end a low bass voice repeated the melody of the last line. The ant under
Philip was humming it.

``Oh, so you can hear me! What is your name? Where are we going? Why did you
not say anything?''

``Yes, I can hear you, Philip. My name is Tony. We are patrolling the
boundaries of the empire. If someone intruded, something broke, or we found
something of value, we would report it to the hive. We have important work to
do. We do not have time to talk and sing.''

``I see,'' Philip said. ``Can I ask you something? How big is the empire? Is
the whole tree your empire?''

``No,'' Tony replied. ``From where we are it extends six branches down and four
branches up. The tree has no end, but an empire needs a border. Otherwise it
could not be defended.''

``But each branch has an end, and the trunk also gets thinner the higher you
go. It ends at the top, and there is nothing but blue sky above it. Had it no
end, the infinite number of leaves would block out the sky. Sunlight would
never reach us and we would live in eternal darkness,'' Philip explained. ``Did
they not teach you in school?''

``Ants do not go to school. We know all as soon as we hatch. We do not need to
learn.''

``I wish I could be an ant! But what if someone invents something new? Will you
also know that? Or will only those that hatch later know it?''

``A good ant does not invent anything. We take the bad ants to the border of
the empire and never let them return.''

Philip did not like that. The best thing about exploration was learning new
things. But he kept quiet rather than argue with his travel companions.
Especially given that Tony was also his swift mount. He leaned back comfortably
instead and enjoyed the sunlight breaking through the canopy. What could the
outcast ant inventors be up to? Surely they are sad that they cannot go home.
But they can go explore. And if they have invented something clever, they must
fare even better than the ants of the empire. They may be building wonderful
cities in remote leaves. Maybe they roll whole leaves into palaces and live in
them at the very top of the tree.

\secbreak

Philip awoke to some nudging. The touch was similar to Babouche's, but hairless
and hard. As he opened his eyes, four long black antennae were swaying
overhead. He sat up and saw that he was still riding on Tony's head. But they
were facing another ant, and its antennae had woken up Philip.

``Where are we? Why did we stop? I fell asleep for a little while,'' the fairy
said.

``We have reached the end of the patrol route. This is the border of the
empire. We found everything to be in order and are going back to the hive,''
Tony said.

Philip was curious about the ant hive, but the ants did not invite him, and he
did not want to be pushy. Instead, he looked up to the top of the tree and took
a deep breath. The trunk seemed to stretch into infinity, and what the ants
said had worried Philip a little. But he was determined to climb to the top of
the tree and prove that they did not know everything.

Philip thanked the ants for carrying him that far and said goodbye. He also
learned that the empire of woodpeckers lay above the empire of ants. It was
extremely dangerous for insects, but Philip knew that birds were so
unbelievably large that they could not see fairies. He was more afraid of the
amoebae and other feral protists hiding in the rough of the bark.

The thick bark made the climb easier. Philip did not have to use his hands. It
was just like going up a staircase. Moisture remained here and there in the
dark corners of the bark; Philip carefully avoided these. Wild relatives of
Babouche lived in the water droplets. They would have surely snapped at the
fairy's ankles if he stepped on the water.

At the bottom of one deep shadow, however, something bright caught Philip's
eye. It was neither the glint of water nor of chitin. As Philip searched for
handholds and flat surfaces on the way to the small cave, the glint seemed to
float in the air. Its shape and size was like a fairy's forearm, but more
angular. Its surface gleamed in the darkness of the cave as it rotated. Philip
could not take his eyes off of it. Leaving behind his usual caution, he
approached the curious object.

The tiny space under the bark was almost completely dark. From time to time the
faces of the mysterious object reflected the brighter world behind Philip. He
reached for it and saw the distorted image of his hand on the crystal surface.
He took it and turned it around in his hand. How was it floating in the air?
How could it be so shiny? Why such a strange shape? It could be better examined
in the sun---

As Philip turned, he found himself facing a dark creature. A spider spawn, full
of legs and eyes, blocked his way out of the cave. The adult spiders were even
larger than ants, but the youngling facing Philip was shorter than a fairy.

``Thief,'' it hissed and approached Philip. ``You wanted to steal my treasure.
But I have caught you red handed. And I have to prosecute you to the fullest
extent of the law.''

The spider raised four legs menacingly and spread them wide. But Philip was not
so easy to scare. He jumped on the spider, and they began to wrestle. The eight
long legs gave no advantage to the spider in close combat. Philip's momentum
upended it and the legs waved in the air in vain. Its mandibles snapped, but
Philip's hands pinned its cephalothorax to the ground. With yarn spun from its
abdomen it tried to tie up the wildly attacking fairy but had no luck. As the
wrestlers got entangled in more and more threads the fight slowed down. Soon
seven of the spider's legs were bound to Philip's right ankle in a knot of
yarn. The end of the eighth leg was in Philip's grip. The spider looked up at
the fairy with eight disheartened eyes.

``I was just kidding,'' it said. ``You can have the looking stone.''

``Thank you,'' Philip said, and he caught the crystal swinging on a thread.
``I~am Philip, the explorer. What is your name?''

``Greg, the cunning,'' the spider spawn introduced itself proudly and shook
Philip's hand with his foot. ``I~am practicing laying traps. You are my biggest
quarry so far.''

Philip did not want to discourage the little spider. He steered the
conversation from the topic of the efficacy of the trap.

``What does the looking stone do? Come, let us go out to the sun and take a
better look,'' Philip said and walked toward the exit of the cave with the
spider still tied to his foot. Greg tried to contribute to the walking with his
one free leg.

``The looking stone is a very tiny grain of sand. The curved surfaces distort
what you see in a peculiar way. When you look through it, everything is yellow,
upside down, and seems much closer than it really is,'' the bound spider spawn
said. Philip raised the crystal to his eyes and looked around from the bark of
the trunk. When he turned even a little, everything moved swiftly in the stone.
But once he learned to hold it very carefully, the distant leaves seemed as if
he were standing on them. Just more yellow. And upside down.

Philip could not have enough of it. He turned the looking stone from one leaf
to the next. He watched distant fairies, ants, spiders and parameciums. Looking
up, he found fairies who really lived at the bottom of a leaf. They plastered
the houses to the leaf, had to build floors instead of roofs, and walked on
yarns stretched between their homes. Scanning along the branch, Philip saw the
toes of a giant bird and then the last leaves. And, beyond the branch, another
branch. Another tree. Everything looked smaller there, and Philip could not
tell how fairies lived their lives on other trees. But he could still spot
larger animals there with the looking stone.

``Look, Greg! Spiders live over there too! Do you know them?'' he asked his new
friend. But in place of the little spider there lay just a discarded pile of
silk thread now. The spider spawn had escaped while Philip admired the looking
stone. He could not find the spider in the cave nor in the surrounding area. He
carefully wrapped the looking stone and put it in his backpack along with the
thread from his ankle. He continued on his way up, hoping that Greg went the
same way, and they would meet again.

\secbreak

Further up, the easy climb through the fibers of the bark was interrupted by a
great hole. The bark had been removed in an area the size of a small leaf. A
deep fissure opened into the body of the tree. Of course an explorer's job is
to investigate mysterious places like this, so the little fairy walked around
the opening to find out more.

The chasm was deep, and Philip suspected the woodpecker. He knew the sound of
pecking. He learned in school that that was how woodpeckers found the tunnels
that larvae dug in the wood. The fairies were grateful to them for removing
the pests of the tree and also for not eating leaves. But it was a different
matter to hear about it and to stand alone in the great crater.

His suspicion was confirmed when he came upon the wide mouth of a tunnel. The
tunnel opened to the left and to the right from the hole drilled by the
woodpecker. Philip reached the opening on the right. It sloped upward while the
opposite tunnel seemed to descend. Did the woodpecker catch the larva? Or did
it scurry back into the tunnel and sneak away from the bird?

``Hello,'' Philip called into the cool gloom of the tunnel, but only the echo
replied. He set out to explore the passage.

The slight initial slope gradually turned steeper. In the vertical spans Philip
had to climb skillfully because the tunnel wall was much smoother than the
bark. But he could also move faster because he did not have to search for a
path or avoid protists hiding in droplets.

As he got further from the hole made by the woodpecker, it turned darker and
quieter. A steady breeze filled the shaft, which thankfully pulled Philip up
and made his progress easier. The little fairy knew it meant an exit was
waiting at the end of the tunnel.

The darkness was not yet complete when the tunnel started getting brighter
again. The weak light had a yellow tint, as if it had entered the tunnel
through the looking stone. The light was growing stronger until Philip reached
its source. The light of the sun filtered through the wall of the tunnel. Thin
plates of wood lent it their yellowish color. At the center of the ring of
light was a hole the size of a fairy. The draft left the tunnel through this
hole.

A strange scene awaited little Philip in the yellow glow. An ant stood opposite
the radiant gate. It stood not on six legs, like normal ants did, but on two,
like a fairy did. The other two pairs of legs were clasped in front of its
body, as if it were praying in the alcove of the tunnel. Peculiar designs
covered the walls of the tiny shrine. The ant must have had carved the lines
into the polished wood.

Philip sat down at the entrance of the shrine and watched the silent ant in
awe. He could not figure out what the lines meant and was eagerly hoping the
ant would explain when it woke up. He got hungry from the long climb and
unpacked two lovely balls of pollen to pass the time usefully. His hands also
found the looking stone. Close up, however, the ant looked like an endless
golden statue. Flipping the stone around pushed everything into the distance.
The unwrapped pollen seemed unreachable. But Philip's hand also stretched like
a spider's thread and fetched a remote morsel to his mouth.

\secbreak

Ants do not blink. Their eyes are covered with thin chitin. They do not need
moisture. Philip did not know if the ant was watching or asleep. He had barely
finished the first pollen when movement caught his eye. The ant spread its
middle pair of legs, then the top pair, then froze into this new pose.

``Would you like some pollen?'' Philip asked and offered the second ball to the
ant. ``Do you live here? Were you banished from the empire? What did you
invent? These carvings are beautiful. Were they the reason?''

The ant did not respond but accepted the tiny snack with a slow movement. It
brought the pollen leisurely to its mouth and ate it.

``Do you know any other exiles? What things have you invented? Will you teach
me too?'' Philip saw a secret ant city built at the top of the tree in his
mind's eye. This mythical place would have been the prize of his expedition.
All he wanted to hear was the ant acknowledge its existence.

``I found the truth. The light. This is the shrine of radiance. The sun is the
source of all life,'' the ant replied. ``The empire of the ants would be lost
without the rays of the sun as well. And yet they cast me away. They did not
want to build the temple of the sun. I have to do the work on my own.''

``I am happy to help!'' Philip suggested.

``The sun is waiting for us at the top of the tree. But I cannot go any
further. Please, take this song to the top of the tree.''


\begin{verse}
\sunsong
\end{verse}


Philip recorded the song in his diary and said goodbye to the ant.

\secbreak

Philip had to squint as he crossed the gate of the dim shrine and walked out
onto the sunny bark. He was high up in the tree, much closer to the sun than
Dewington. The pollen had filled him with satisfaction. So had the old ant
speaking about the top of the tree. He rose toward his goal with light strides.

The merry walk ended abruptly when Philip slipped and fell on his back. As he
got on his feet his fingers felt strange slimy forms. Everything around him was
teeming with yellow bacteria. They were squirming across the tree bark, spreading
their slippery mucus over it. They surrounded Philip and nudged his shoes with
their noses. Kicking the fingerlings out of his way he retreated to the dry
part of the bark. Looking up, he saw that the germs covered most of the way up.
They used the slippery mucus to extract nutrients from the bark. Philip did
not feel a rash on his skin, so the mucus had not been dangerous. On flat
ground he would have carefully cut across the field of bacteria. But it was too
dangerous to continue the climb like this.

Philip looked for another route. Due to the curvature of the trunk he could not
see how far the bacterium colony extended. But he saw a thick branch level with
the ant's shrine. After climbing onto this branch he was better able to assess
the situation. The bacteria had not infested the area by accident. Above them
was the nesting hole of the woodpeckers, and the slimy microbes lived off of
their waste. He could not see how much of the other side of the trunk the
unnavigable field covered. But Philip would rather not risk the trip. The
yellow bacteria were harmless, but he knew that more dangerous varieties could
also be living in the area.

Deep in thought he walked on along the branch. As he got further from the trunk
he got a better picture of the layout of branches. If he could somehow get from
one branch to another, he could keep his distance from the woodpeckers and
continue his journey to the top.

Philip stumbled again and fell forward this time. The sound of a string plucked
filled the air. It still hummed when the fairy got back on his feet and started
looking for the invisible obstacle that had tripped him. He found it by the
sound and the vibration. A clear yarn the thickness of a fairy's finger was
fastened to the branch. The yarn rose into the air at an angle. No matter how
Philip squinted he was unable to make out the other end of the fine strand. He
hoped that it might be a way to a higher tree branch.

In his mind he extended the line of the string and tried to guess which branch
the other end might be tied to. Halfway to the next branch, however, he spotted
something on the imaginary line. A huge black spider seemed to float in the
air. Its legs worked the fine yarn, and before Philip could tell whether the
spider was coming or going it was already standing above him.

``Well, well, look what my web has caught,'' the spider droned as it towered
above Philip. ``If it is not one crafty fairy!''

``I am sorry, I did not see your web. I am Philip,'' he started his
introduction.

``The explorer. Yes, I know. Playmate of Gregory. I am his mom. Should I ask
him if he wants to come down and play?''

``Thank you. But the truth is I would rather go up. Where does this thread
lead? Can I get to the next branch this way?''

``Of course! You are very welcome to use our web, free of charge. I can also
offer you an inexpensive map if you need one,'' the spider said slyly.

Philip knew the story of John. The fairy knight went to fight an evil spider
who exploited the inhabitants of the leaf. After a long journey he found the
web of the spider and started to climb. The spider saw his fearsome sword and
fled to the other side of the web. John chased after it and deftly climbed the
yarns of the web. But the spider always avoided him and the web became ever
stickier under John. Only the spider knew which thread was sticky and which was
safe.

In the story John cut off the web with his sword and chased the spider out of
the tree. But Philip had no sword nor did he want to cut off the web.

``What would you ask in return for the map?'' he asked.

``Knowledge for knowledge,'' the spider mom replied. ``Teach me how to knit a
nice jacket like yours. I have a lot of yarn and would love to make a winter
coat for Gregory. But I do not know the first thing about knitting.''

So it came to pass that Philip taught the spider to knit.

\secbreak

Once the spider had learned enough, she started knitting a simple square cloth.
She wove an intricate pattern from three alternating fibers. Line by line her
work revealed a map of the web to Philip. Seven long threads anchored the web
to various branches. Radial fibers stretched between them, and a dense spiral
of sticky threads covered them. A network of walkways allowed the spiders to
get around without getting stuck.

The map of the secret walkways was very useful. But Philip's attention was
drawn to something else. The spider had embroidered detailed pictograms at the
seven anchor points. Philip recognized only some of them. A woodpecker was
peeking out of a hole where he had met the spider. An aphid was grazing at
another anchor at the same height. At the other end of the thread a fairy was
smiling. To the left a squirrel was chewing on something at the anchor of the
web.

Philip's eyes gleamed as the spider knitted the map. He had not even dreamed of
so many things to explore. A whole day would not be enough to uncover all these
secrets. He had to decide which one of the threads to leave the web by first.
The top of the tree was still the destination of the expedition, so he chose a
yarn leading high up, to the smiling fairy.

Philip was no spider and one branch was far from the next. Greg accompanied him
for a span, but the climb still felt endless. Gradually, however, he began to
suspect where the thread led. It was not fastened to the bark of a branch like
the anchor at the woodpeckers was, but to the bottom of a leaf. And as he came
closer, he saw it was not fastened at a single point either. On approach the
thread unwound into three fibers and ran to three different sites on the leaf.
Philip chose the one closest to the tip of the leaf.

When he got close enough to see the fairy buildings, a busy city spread out
ahead of him. As seen through the looking stone, the inhabitants built their
homes on the bottom of the leaf. Garish floorings adorned the bottoms of the
houses. Terraces with colorful fences and the hanging bridges between them
served as streets for the fairies of the city. Thin threads connected one porch
to another across distances great and small for sending messages and getting
around. Philip also noted an interesting method of freight transport. From time
to time big packages swung through the air from one end of the leaf to the
other on threads of silk. Fearless fairies stood upon them and guided the
shipments to their destinations by leaning against the thread.

The spider thread on which Philip climbed led inside a large building. The
atrium around the thread connected three floors. On each floor planks reached
out to the thread to make getting on and off easier. As Philip stepped on the
plank of the first floor, exhausted from the climb, he found himself right
inside a busy warehouse. An army of strong fairies filled the floor. They
carried large bales over their heads and directed the loading process through
loud jabber.

``Where are the rest of the algae?'' someone called down from the second floor.
``Hey, little fairy! Go ask which storeroom they put the algae in! We've got to
ship them to grand central.''

There was no such organized trade in Dewington, but Philip was used to
conveying messages between adults. He ran among the stevedores and asked where
they took the algae. They took them to storeroom three.

``They took them to storeroom three!'' Philip shouted upstairs.

``Why not to number two?!'' the fairy upstairs grumbled. ``All the same in the
end. Let the cranksmen in the loft know they can get started. We cannot wait
any longer.''

Philip did not understand much of this, but he grabbed the spider silk from the
plank and climbed up to the top floor. The end of the thread was not fastened
to the surface of the leaf. It wound around a big reel instead, thickly
covering the axle. A large wheel was attached to the axle and could be pulled
by many fairies from both sides of the floor.

Philip landed on the plank of the loft and conveyed the message to the idle
workers. They sprang into action at once and eagerly started turning the big
wheel. The spider silk stretched and twisted around the reel. Up close, it
seemed as if they were pulling the web closer to themselves. But Philip
realized that the web was not moving. As the thread shortened, the leaf itself
swung into motion. With each turn of the wheel the entire leaf, the entire city
bent a little.

Philip was impressed by the legion of workers and the stream of goods their
hands unloaded. He helped out for a while in the silk docks. He got to know the
local fairies and the history of the leaf in the short breaks between loading
and unloading. They collaborated with Greg's family and had built a trading
network along the silk threads. The spider family supplied them with plenty of
silk and assisted in setting up the system of reels. In exchange, the fairies
offered special delicacies, spices, news, and colored crystals to the spiders.

At the end of the shift the workers invited Philip to lunch. After a hearty
meal they produced their greatest pride, the spider brandy. Philip had not
heard of it in Dewington. Seeing his interest, the locals started on a detailed
explanation. Those with eight legs and those with two enjoyed this distillate
equally. Golden dewdrops condensed at dawn on the bottom of the leaf and of the
fairy buildings. Teams of fairies collected the tiny droplets into large
barrels using safety harnesses and agility worthy of spiders. The heady
beads were then stowed in casks and tapped whenever the fairies gathered and
danced together.

Their best kept secret was the origin of the golden droplets. The spirit rose
to the leaf dissolved in warm air. The touch of the cool surface forced it into
a liquid form. The air was warm from the light of the new sun and lifted off of
a leaf on the branch below. Upon this leaf the fairies had put a herd of gentle
aphids to pasture. The aphids produced honeydew, and the fairies fermented a
part of it with the help of a fungus. The heat of the rising sun evaporated the
spirits from the fermented droplets, and these spirits ended up in the barrels
of the city.

Philip did not try it, but was told that it tasted like sunlight. After a sip
the fairies would feel like they were basking in the summer sun. Their city was
named after this beverage. Solperla, he learned from the locals, meant ``pearls
of the sun'' in a foreign language.

The spirit was round and yellow and could be ignited by a spark. It would give
off light and heat then and could burn a careless hand. A round black spot,
not unlike the dry Stain of Dewington, marked the invention of fire in
Solperla. The fairies had mastered the use of fire since then. At night a tiny
flame always flickered below the city. The light helped travelers find their
way home and drew many unwary insects into the spiders' web. The spiders
mistrusted the dangerous beacon at first, but the rich harvest had soon
convinced them.

Philip said thanks for his lunch and for the interesting stories, and headed
off to explore Solperla. He welcomed the change of pace after the flurry of the
docks. He walked through the bridges of the city and looked down on the leaves
below. He knew one of the leaves had to be Dewington. But even with the looking
stone he found it impossible to find where he had set off on his journey in the
morning.

\secbreak

Philip soon found himself playing tag with a flock of Solperlan children. The
game took more skill than in Dewington. Even the smallest fairies knew all the
bridges and stray threads. They walked across silk fibers, swung on threads
hanging from terraces, and every time they appeared to dive into the deep, they
caught on to a ledge or overhanging fabric at the last moment.

Philip could barely navigate the maze of suspension bridges, so he often got to
be ``it''. At one time he was chasing an older playmate. The boy gained
distance through a series of unexpected shortcuts, but Philip did not give up
the pursuit. They were heading for the stem of the leaf, and he knew that the
labyrinth that was Solperla would come to an end there. The network of swinging
bridges, terraces, and spider silk woven between them was running out ahead of
the boys. Nowhere to run, Philip tagged him with a good lunge.

The crowd of local children had been right on their heels. They scattered
screaming as the new ``it'' turned around and started after them. Philip was
left alone at the edge of Solperla. The wild pace of the chase still pounded in
his ears. But his eyes turned to the stem of the leaf, and his heart settled
down. If he wanted to get to the top of the tree and get back to Dewington
before dinner, he could not spend any more time playing.

At this altitude the trunk was growing thinner and the horizontal branches were
getting shorter. The bark was also younger and smoother. The tiny hands of a
fairy could still find a hold, of course. The smoother and younger surface just
meant that Philip did not have to worry about germs lurking in dark cavities.

With each passing branch and twig the foliage above him grew thinner, until at
last the trunk ran out. Two leaf stalks grew at the end, and he started to
climb the one that led to the higher leaf.

Philip saw nothing of note from below, but he knew something exciting must wait
for him on the other side. It was impossible that he would not find something
special on the very highest leaf of the entire tree. Perhaps the magical city
of outcast ants would be revealed? Or the sun itself lived up there in a
glowing palace? In his court a bustle of fairies would walk around in strange
dresses, important papers in their hands, and they would have no time for
Philip?

When he came level with the leaf, he counted to three. Then, ready for
anything, he jumped on top.

\secbreak

The top leaf was more slanted than the rest, and Philip almost tumbled right
back from it. He regained his balance, but found nothing exciting other than
that. The immaculate green leaf was completely barren. Walking through the
central vein, he could have as well been walking down the main street of
Dewington in the early spring, before the fairies had arrived.

It felt good to stretch out on the soft warm surface after the long climb. It
felt good to cartwheel and roll down the main street. And it felt good to run
around in circles. Half a circle down the slope, half a circle up the slope. It
felt good to climb to the tip of the leaf too, where his father and mother
would have awaited him, had this leaf been Dewington. And at last it felt good
to look at the whole wide world from the highest point of the tree.

He had seen distant leaves from the window of his own room. He knew the most
remote ones grew from branches of a different tree. He knew that birds flew
from one branch to another and could fly from tree to tree as well. But all
that did not prepare him for the sight of the forest. Philip's tree stood in
the side of a large hill with great view of the rolling landscape. An
infinitude of trees covered the hills. There were more trees in the forest than
leaves on a tree. The tree was finite, but the forest seemed boundless.

And every tree was different. One was taller, the other shorter. Some had
berries, others had cones. Even their leaves were not all the same. Philip
looked in awe and tried to imagine what life might be like on the oddly shaped
leaves. Some were round, while some had three or five tips. Many leaves were
larger than Dewington. But many were less green than those on Philip's tree.
The autumn seemed to have arrived already on other leaves. The rolling hills
were shining yellow, red, and brown, matching the hues of the clouds in the
west. The sun was setting behind them. Its faint disc reminded Philip of the
song of the old ant.


\begin{verse}
\sunsong
\end{verse}


The song teased the sun out from behind the clouds, and it flooded the hills
with golden light. A lark flew up from a nearby tree and called its playmates
with a sharp squeak. Some did join it in flight, and they danced together in
the sky. Others joined them in song. But all of a sudden they all vanished.

Philip lifted his hand to his eyes and looked between his fingers toward the
distant hills. A second sun was rising from behind the horizon. Its flickering
light seemed to set fire to the colorful slopes. The dark clouds became
shadows. The song of birds dwindled. Every creature in the forest was watching
the awesome sight.

The blinding light rose higher in the sky, and soon a deep rumble filled the
landscape. Philip watched it lying down on the leaf. As the rumbling sound grew
louder and louder, he was frightened to see that a wave was sweeping through
the forest. The trees bowed in its way. Their yellowed leaves were cast high.
Flocks of birds took flight in panic. The wave rolled out from the new sun
through the landscape and relentlessly headed toward Philip.

Philip did not exactly know what was going on, but his instincts put an end to
placid observation. Without thinking he jumped up and ran. Away from the tip of
the highest leaf, away from the path of the wave tearing everything up. He ran
down the barren main street of the leaf, slid down the smooth stalk and held on
to the unyielding bark of the trunk. And then the wave hit.

The whole tree shook from a dreadful thunder. The air filled with whirling
grains of sand. Entire leaves tumbled among them, and Philip could even see
twigs torn off as a whole. A bird was escaping the bustle, but unpredictably
writhing branches got in its way. Philip held on to small cracks on the bark
with all his might and hoped to see his family again.

Sand grains the size of a fairy tore at the branch around him. A brown leaf
covered him and blotted out the glare for a moment, but then the wind carried
it on and tried to take Philip along. Tiny droplets of mist peppered around
him, and the landscape swung into motion as the trunk bent. Something grabbed
Philip, and they tumbled into the void together.

\secbreak

Fortunately, Philip had clung to the bark through the worst of the blast, so
when he tumbled they fell straight down. They whooshed past a branch, then the
next one generously extended a leaf in their way. Head over heels they landed
on the surface in a heap. The world revolved around them until the trembling of
the leaf slowly subsided.

Philip sat up, and an old blue fairy sat up against him.

``Oh, my, where may I be,'' said the blue-skinned lady. She straightened out
her fine clothes and put a strand of hair back into the tangle on her head.

``Thank you for helping me land here, little fairy,'' she thanked Philip.
``That's strange weather, is it not? We'd better go look for your parents and
make you a nice cup of tea. You have paled entirely, I see the blue has drained
right out of your face.''

Philip felt his face in surprise. Indeed, it was not blue, but it was not
usually blue.

``I am Philip, the explorer,'' he blurted out for lack of something better.

``Oh, such a very nice name. I am Raisy. Which leaf did you leave for your
expedition, Philip? I think we have seen it all now and it is time for a
rest.''

Philip agreed with her. As soon as the tree settled down, they started their
journey toward Dewington.

Raisy was as small as Philip, despite being older than even the adults. And she
was very blue. The wind flew her off of a distant tree, and she said everyone
was blue there. She knew a lot about the forest, and even knew that the new sun
that rose so violently was called a rocket. Its grumble had faded by then, and
when they spotted it high above the canopy, its brilliant flame had shrunk to a
pinpoint.

``How are you going to go home, Raisy?'' Philip asked. ``Maybe you could jump
from a very long branch to a neighboring tree, but the journey would take
days.''

``Weeks,'' Raisy corrected him. ``The autumn is knocking on our doors. We
cannot go on such long journeys now. There is nothing to eat on the withered
leaves, and they will all turn brown before I would get home. I will have to
move in with your family, Philip, unless---''

``Unless we build a rocket!'' Philip suggested.

Raisy came from an evergreen tree, where the leaves were narrow needles, and
the fairies dug their homes into them. The homes were connected by long winding
staircases, but their children came and went through the windows too. The
sticky resin of their tree did not freeze in the winter, and their excellent
engineers had found a way to heat the apartments. Raisy too was an engineer,
and while she had never heard of fairies building a rocket, she did not
consider it to be impossible.

``The most important issue is fuel. If something burns fast, it also evaporates
fast, so it will not be easy to find it in nature,'' contemplated Raisy.

``I know exactly where we can get fuel!'' Philip said and slid down the stem of
the leaf.

\secbreak

Philip waited for Raisy at the base of the stalk, but it appeared that the
elder fairy had chosen a more cautious way of descent. He watched through the
looking stone as she started off carefully. But the leaves were different from
the needles of the blue fairies, and Raisy did not fare well on the waxy
surface. She tumbled a bit, grabbed the stem again, climbed a little, rolled
down, then held on and continued climbing. It did not take long before her
erratic descent ended with Raisy hanging on her long blue scarf from the stem.
Philip started to climb back up to help her but had not made it half of the way
when he heard Raisy scream. His companion's scarf had slipped off the stem and
was trailing behind the falling fairy. Philip had decide in a heartbeat. He
decided to jump from the stem, grab Raisy mid-flight, and plummet together for
a second time.

Her long scarf slowed down their descent, but also made it unpredictable. The
rocket's rumble was a barely audible murmur by now, but the air had not fully
settled yet and took the two fairies for a spin. They looped around the leaf
stem, sped by the branch, made two loops below it, rose a little, then fell
again spinning.

``Great fun, is it not, Philip?'' Raisy screamed over the sound of the air
rushing past them.

The world looped around them a few more times then came to an abrupt halt. The
blue scarf froze in the air, as if caught by invisible hands. The fairies had
ended up on a sticky line of spider silk.

``Watch out, Philip, it's a spider!'' cried the little blue elf. ``We must get
out! We have to break free! If the spider gets here we are done for!''

Raisy struggled desperately, but to no avail. But her panicked efforts
attracted the attention of the spider. The big black body was walking to them
across an invisible thread.

``It is good to see you again, Philip,'' said the spider. ``I~was very worried
about you. What a gust! Fairies were falling into my web like raindrops, I
could barely keep freeing them. Fortunately, I caught a nice big bug too,'' she
giggled.

The terror on Raisy's face turned into incomprehension, then disbelief, and
finally a wide smile.

``This is Raisy,'' Philip introduced her. ``The wind blew her here from an
evergreen tree.''

Raisy found her voice and greeted the spider warmly. It turned out that
fairy-eating spiders lived on her tree, and it was extremely dangerous to fall
into their webs.

``Of course, friendship is much better,'' added Raisy, a little embarrassed.
She was not keen to give the spider new diet ideas.

``I agree wholeheartedly,'' said Greg's mother, ``and fairies are not tasty
anyway.''

After freeing both of them from the sticky silk she showed them Greg's
half-finished coat. They had a nice chat about the subtleties of knitting, and
the spider even offered Raisy some brandy.

``A lot of barrels fell into my web from Solperla. They left them behind to
thank me for to rescuing the fairies from the web. Here's to the great luck
that we all survived this whirlwind!''

``Good stuff,'' said Raisy with a click of her tongue. ``This will make an
excellent fuel for the rocket.''

The spider said she would never consume so many barrels, and offered a deal to
the fairies. They could have all of the fuel in exchange for tying a silk thread
to the rocket. The spider could then walk to the faraway tree. And if commerce
began between the two fairy peoples, she could levy a tax.

The two fairies were enthusiastic. They accepted the offer and received eleven
barrels of spider brandy. As they arrived at the bottom of the web, Raisy asked
for the looking stone and the travel journal and started to measure and
calculate. How far was the evergreen tree? What was the best arc for the
flight? How much did a fairy weigh, and how many barrels of brandy would they
need?

Philip followed along at first, but then his attention drifted. Something was
moving on the surface of the branch in the distance. He asked for the looking
stone so he could get a better look, but Raisy would not give it up. He tried
to gain a better view by hanging from the web. He stuck his ear to the bark to
see if he could hear footsteps. Something small and furry was riding toward
them.

A few moments later he was running circles around Raisy with Babouche in tow.
The paramecium had tracked him down somehow, and Philip was very glad for the
company. Chasing one another was much more exciting than following the complex
calculations. Mary soon appeared on the horizon as well. Babouche ran from
Philip to Mary and from Mary to Philip until the fairy boy and fairy girl
reached each other.

``There you are, Philip,'' Mary rejoiced. ``Everyone in Dewington is worried
for you. Good thing you were not at the top of the tree when the wind picked up
everything! We would have to haul you back from the far end of the forest.''

Philip was happy to see Mary too. He told her that he was actually right at the
top when the blast from the rocket reached the tree. And he also told her what
a rocket was and introduced Raisy, who flashed a smile at Mary but did not
stop working on her calculations.

``And this is our rocket fuel,'' he pointed at the eleven barrels of brandy at
the base of the spider web.

\secbreak

``Are you crazy? If a rocket that far from us created such a big wind, then
yours will tear the tree from the ground,'' Mary said.

``Oh, Philip, you are scaring Mary,'' Raisy interjected. ``That distant rocket
was loaded with more barrels of fuel than there are leaves in the forest. Our
rocket will only have a hundred barrels.''

``One hundred barrels?'' Philip asked, surprised. ``But we only got eleven
barrels from the spider.''

``I know, Philip. Unfortunately, it appears we will need to find eighty-nine
more barrels.''

They climbed up to Solperla. Many bridges hung broken, and the colorful fences
were torn by the wind. One of the homes had almost separated from the bottom of
the leaf, and they watched a team of fairies rescue the residents and try to
pull the collapsed house back in place with ropes. One dock worker got stranded
on the swinging freight transport line and looked up at the city anxiously.
The strong building of the docks was thankfully unharmed, but hundreds of boxes
and casks had rolled out from the storerooms inside, and workers were busy
restoring order.

Raisy, Mary, and Philip helped sort the contents of a warehouse and learned of
the damages the city had suffered. The fairies were accustomed to unexpected
gusts, and most were able to hold on to something. Those who fell were brought
back by the spider. But unfortunately, the barrels of brandy all broke free,
rolled out of the stores and fell into the depths. There was one unfinished
barrel in the foreman's room, and he gave it to Philip in return for their
help.

``If this goes on, we will not have one hundred barrels by spring,'' Mary
lamented. ``And I was starting to look forward to traveling in a rocket.''

``If Solperla has lost its stock of brandy,'' Philip thought aloud, ``the
surest way is if we distill it ourselves from fermented honeydew. There is a
leaf under Solperla, where the aphids are put to pasture. Maybe they can give
us honeydew.''

The small party descended to the leaf below the city to assess the reserves
there.

``I am telling you I will sprout four more limbs, if we spend the whole day
climbing up and down on the spider's web,'' said Raisy as they arrived on the
leaf. Philip and Mary were ahead of her, pushing their way through between the
legs of a crowd of fairies. Raisy followed them.

The fairies stood in a ring around a raging aphid. The insect was smaller than
an ant, but still towered over the fairies. They all held thin ropes of spider
silk and tried to keep the unruly beast in the center of the circle. The aphid
neighed and tried to cast off the ropes, but a brave fairy climbed onto its
back, tied its antennae together, and tried to subdue it. A young fairy quickly
ushered Mary and Philip out of the circle.

``Take care, my dears,'' she said while leading them from the circle of fairies
wrestling the beast. ``Someone could trample you. What is going on here, you
ask? Aphids are usually very gentle animals, but they go wild when separated
from the flock. Unfortunately, the flock broke out of the pen when the wind
bent the tree. This poor calf fell behind, and my sister lassoed it by the
trunk. A dozen strong fairies helped her to bring it back. Not much longer
until it calms down now.''

``What will become of us, Judy?'' an older boy asked her. ``Without the flock
we will not have anything to eat. What will we do when the winter comes?''

``All year we were preparing for the winter. The pantry is full,'' Judy
reassured him. ``But we will have to tighten our belts in the spring if this
calf is going to be grazing on the leaf all by itself.''

``I am Philip, the explorer,'' said Philip timidly. ``Maybe we can help you
find the stray flock.''

``Where are you from, Philip?'' asked Judy.

``I am from Dewington.''

``So you live in the ant kingdom. Then you should know that the ants will
capture the herd before sunset, and no matter how much we ask, they will not
give it back. They will not answer, and you will not even know whether they see
you.''

Philip knew the ants saw him, and Tony might even answer him. But he knew Judy
was right, the ants would never return the flock of aphids. Then he remembered
the old ant who worshipped the sun, and hoped that maybe he can offer them
advice.

Once more they set off on the spider's web. Mary broke the silence of the
climb.

``What if we cannot find more fuel? Are we going to dilute these twelve
barrels? Or ask the fairies of the evergreen tree to send us some more? Or just
play something else?''

``But we are only eighty-eight barrels short,'' said Philip. ``I~am sure the
old ant will help. The fuel is almost ready for the launch.''

``Roger that, Captain Philip! And what else do we need for the rocket?'' She
turned to Raisy now.

``We need a house with strong walls, in which we are going to fly. We need to
attach wings to it, so we can steer, and we need a nozzle,'' Raisy listed.

``A nuzzle?'' Mary asked. ``For what?''

``When the fuel is ignited, its flame goes every which way. We need to set it
on fire in a space such that the flame can only go backward. Then it will push
us forward,'' the blue fairy explained. ``The nozzle is a special tube with one
narrow mouth, where the fuel trickles in, and one wide mouth, where the flame
comes out.''

``But in Solperla they said the flame burns everything,'' said Philip
thoughtfully. ``Will it not burn the nozzle?''

``I too heard a story from my grandmother,'' Mary said, ``in which the spider
catches John and breathes fire upon him.''

``Now that is really something,'' Raisy gasped. ``A fire-breathing spider?
Maybe it drank too much brandy? And what did it spin its web from so that it
would not burn?''

``That is the point,'' Mary said. ``The web burned, the leaves burned, the
whole tree burned, and even the earth blackened. In the end, in the midst of
the soot and blackness there was only John left standing against the spider.''

``How come they did not burn?'' Philip asked. ``Are you suggesting we ought to
make the nozzle from fairies or spiders?''

``They did not burn,'' Mary explained, ``so that the story could continue. But
the earth they stood on did not burn either. Let us make the nozzle out of
earth!''

No one could argue with this line of reasoning, so they agreed on the plan.
Philip's task would be filling the brandy reserves. Raisy would build the hull
of the rocket. Mary would create a nozzle from earth. For a while longer,
however, they travelled together, carrying the barrels from the spider web.
They marched on the branch to the trunk, each balancing four barrels above
their heads. Along the way Raisy told the children stories of the family she
had left on the coniferous tree.

\secbreak

``And my oldest grandson is called Robert,'' Raisy completed the list. ``He is
about as big as you are, Philip! I wonder what he might be up to at this
moment.''

While Raisy told them about her grandchildren, they arrived at the gate of the
ant's sanctuary and unloaded the heavy barrels.

``I cannot wait for you to meet them, Mary! You will get along splendidly.
Come, let us see what Philip's ant friend can tell us!'' she said, and they all
stepped through the ornate gate.

A wondrous golden glow still filled the sanctuary, and tiny specks of light
swirled in the scented air. Although there were three of them now, the silence
seemed even deeper than when Philip had visited alone. Raisy hugged the kids
and they watched the ant for a long time without saying a word. Surrounded by
the wood carvings, the ant touched its top limbs together in prayer. Nothing
moved but the shadows of the leaves swaying in the light that shined through
the gate. But the leaves were very far and their ever-changing shadows danced
in vain on the ant's armor. The guardian of the sanctuary was dead. Long stalks
of golden spears had pierced its forehead and shoulders. Yellow daggers emerged
from its legs as well.

A yellow shard fell down and broke the fairies' daze. The shard was as big as
Philip's arm, and dropped to the floor with a puff of golden dust. Raisy was
hesitant to let go of the fairy boy's hand, but a moment later Philip was
picking up the strange fragment. It was very light, but seemed strong too. Its
surface was covered with a thousand holes, as if someone had tried to imitate
the ant's carvings. As it moved through the air, it left a thick cloud of dust
in its wake. To avoid dying everything yellow, Philip wrapped it in the map of
the spider web.

\secbreak

Mary slid down first through the abandoned tunnel. When her shrieks died away
in the darkness, Raisy followed. Her voice faded too, then fairy laughter
chimed in the dark passage. This was the signal Philip was waiting for, and one
after another he launched the barrels of brandy down the slide. Sounds of
rolling and fairy laughter announced the arrival of each one, until the barrels
ran out. Philip descended too, then said goodbye to Mary and Raisy. The two of
them went sliding again, down into the other section of the woodworm's tunnel,
and Philip sent the barrels after them. The plan was to meet up in Dewington.
With his eighty-eight barrels and Mary's nozzle they would be ready to start
the construction of the rocket.

Philip unwrapped his looking stone and scanned the leaves below him carefully.
He was searching for the grazing land of the ants, but another leaf caught his
eye. The looking stone brought him back to the streets of Dewington. The wind
had ruffled the roofs of their homes, and there was no trace of its namesake,
the dewdrop in the main square. But all the buildings were standing, and after
Philip traced his way back to where he started in the morning, he arrived at
his home. He could not see through the roof, but he hoped his parents were at
home and not worrying about him. He did not want to make them wait
unnecessarily, so he resumed searching for the flock.

Fairies lived on the surrounding leaves too, and he saw an ant making the
rounds on one of them. He found two more on the branch, all of them on patrol.
He follow them to another branch. Scanning this one he found a bud covered in
suspicious spots. It was in shadow, so Philip could not see it well, but he
suspected the lost flock could be found there.

Philip began his brisk descent thinking about how he would convince the ants to
return the fairies' aphids. Judy was right, they would never listen to reason.
But if he could lasso the ants, he might be able to tether them like the
fairies tethered the young aphid. Or he could imitate the sound of the rocket
to startle the herd, and when they flee from the bud to a leaf, he would chop
it right off with the royal sword. Timed right, a gust might lift them into the
air. He could loop spider silk around a branch to swing the floating leaf
around the tree and into the spider's web. From here, the aphids would all
tumble straight down and into the pen below.

The plan was missing some details. For example, the royal sword was hanging on
the wall of George's living room. But overall it was fairly solid. Philip was
striding confidently toward the bud already. From this distance he could see
that the bud was covered with aphids and guarded by ants. An ant coming from
the direction of the bud walked over Philip unexpectedly. He was ready to
fight, but the ant trotted past on its huge legs without notice. Philip's knees
shook as the large mass moved by, but he did not turn back.

What if ants really could not see him? Perhaps he could lead away the biggest
aphid. All the rest would follow, and the ants would just stare blankly.

That was not what happened once he arrived. No matter how much he pushed or
pulled, the leader of the herd did not move a bit. Philip tried to lift the
foot of the monstrous aphid, but it did as much good as wrestling with an ant.
He climbed on its back and slapped the rear instead. The aphid did not notice
this either. He sat on its head and waved in front of the eyes. He poked its
nose with his feet and shouted in its ear, but the aphid only cared about the
sap. How did he manage to communicate with Tony? Perhaps this behemoth would
also enjoy a fairy song.

\secbreak

While Philip was struggling with the biggest aphid of the tree, Mary and Raisy
had arrived in Dewington. Both of them balanced four barrels on their heads
and rolled two more ahead of them with their feet. It was no wonder that their
arrival on the leaf drew much attention.

``What is it with all the barrels?'' an alga farmer thought at the edge of the
village and took a break from roof repairs to watch their arrival.

``Look, a blue lady!'' a fairy boy pointed out to his siblings as he came to a
halt. It was a mistake to stop, for his brother and sister stumbled over him
from behind, and they ended up wrestling in a heap.

``Do I smell brandy?'' a shoemaker said as he opened his window. The blast from
the rocket had mixed up his shoes, and a good number of them had still not
found their match.

``How did they climb the stalk with so many barrels?'' a young fairy pondered
in the main square. Nevertheless, he politely helped them unload, and by the
time they finished they were surrounded by alga farmers, fairy children, and
also shoemakers.

Mary told them of the plan for building a rocket, and those with nothing more
important to do offered their help. Raisy organized a team to collect damaged
roofs and other debris to build the rocket hull from. Mary hurried to Philip's
parents to reassure them.

Philip's father and mother were very grateful for the news of their son and
offered her fructose crystals. She could still smell the berry from which the
crystal was extracted. It tasted very good after the heavy lifting. She
accepted one more for the road, but in return she had to promise to return
their wayward son before dark.

She ran through the main street with renewed energy and rushed past Raisy's
team, who were just trying to prevent the collapse of their construction with a
plant fiber. When they failed, the rumble made Mary stop for a moment.

``Go on, Mary,'' Raisy said. ``By the time you get back, we will be ready. This
is only a sketch; the final structure will be much stronger.''

Mary wanted to lift their spirits and gave them the delicious fruit sugar. She
listened to their discussion of the rocket hull structure for a while, then set
out on the way to the ground.

There were few stories about the ground. Her grandma had told her it was
because the fairies had moved into the trees long ago. What did they do before
that? From her great-grandmother she heard they lived in a beautiful city. Its
towers were as big as trees. And they had giant machinery that did everything
for them. The machines were the shape of a fairy, but their arms were thick as
tree branches. The machines served them for a long time, and the fairies grew
lazy and forgot how to work. They even forgot how to repair the machines when
they broke. One day the eyes of the machines broke and they could no longer see
the tiny fairies. They were no longer able to feed them, they could not build
houses for them, and could not protect them from insects. The ants were waiting
for this and took over the magnificent city. The fairies fled every which way.
Mary's ancestors had fled into the forest. They did not know where to find food
or what should and should not be eaten. The first year brought nothing but
hardship, but they did not give up and learned once again to live on their own.

The trunk grew thicker and the crevices of the bark deeper as Mary descended.
Grains of sand got in her way. Some of them were as tall as a grown up fairy.
Mary examined the base of each grain. Mud splashed on by the rain glued the
grains of sand to the bark, and Mary wanted mud. Crouching under a block she
took out her quartz blade and scraped off a handful of clay from underneath.
Shifting the tiny flakes in her hand, she tried to pick out the cleanest ones.
She touched them to her tongue for taste, and discarded one of them with a
grimace. The other two she dropped in her pocket and resumed her trip.

\secbreak

Meanwhile, the lead aphid had made just three steps. One step forward, one to
the right, and one step back. Philip could not make it move by either strength
or song. He tried locking eyes with it through the looking stone. Perhaps it
would listen to an enlarged fairy. But it did not work. He was digging around
the bottom of his backpack now, hoping to find something that could affect his
steed.

Half a dozen brightly colored quartz crystals. They did not write on the fine
chitin, the aphid did not want to eat them, and Philip juggled them in vain.
The aphid did not even move an antenna.

A large jumble of spider silk. It took Philip a lot of effort to tie this
around the legs and body of the aphid, but then he realized how flexible spider
silk was. It did not hinder the beast at all. It made for an excellent swing,
however.

Two wrinkled fruit fly wings. He unfolded these and attached them to the
aphid's head with spider silk, as if they were droopy ears. Had the aphid moved
its head, its fake ears would have flapped adorably.

A tassel woven from plant fibers. The fibers were dyed to different colors, and
Philip's mother showed him how to weave them. He also showed it to the plant
lice, and wove three of its bristles into a tassel, but to no avail.

There was the backpack itself, woven from fine plant membranes. For lack of a
better idea, Philip stuck his head in it and ran around clapping and shouting
on the back of the aphid until he fell down. Fortunately, he got stuck in the
spider silk stretching between the legs.

Finally, wrapped in the map, there was the golden yellow shard. Philip did not
know exactly what it was, but he was afraid of it. He honestly expected that
the old ant would help him. But all he got from it was this dusty fragment. He
had heard from adults that everything had a soul. Perhaps the souls of ants
were yellow. The soul of the old ant might have grown too large and broken out
of its body.

Sitting on the back of the aphid, he carefully unwrapped the shard and shifted
it from one hand to the other. Its touch dyed his palms a bright yellow.

``Philip, wrap it back up,'' an ant said behind his back. Philip recognized the
voice of Tony, who had carried him from Dewington to the border of the ant
empire. But it was also possible that all ants had the same voice.

``The aphid flock of the fairies ran away,'' Philip explained, ``and I have
come to herd them back. I hope you do not mind.''

``I carried the sick ant up high, beyond the border of the ant empire,'' the
ant said. ``Everything that the golden fungus has touched has to go. You cannot
stay here. You are bad. Your hands are so yellow. Do not touch anything!''

``What if I touched something?'' Philip retorted.

``Everything will be bad! Do not touch anything!''

The ant tried to catch Philip, but it was clearly wary of the yellow shard. It
went around the lice and tried from the other side, but recoiled when Philip
turned to face it.

``This aphid is yours, just wrap up the fungus.''

Philip wrapped up the piece of soul fungus and climbed down from the aphid,
leaving a line of yellow palm-prints behind. The ants all kept a fair distance
away. Everywhere he went they backed away from him. Once he was far enough from
the lead aphid, the ants quickly picked it up and carried it off the bud.

Philip held his golden fingers in front, clawing at the ants to scare them.
He enjoyed not being ignored at last. On the contrary, they were all watching
him now. He started running between the ants and managed to scare one of them
so much that it fell off the branch. He drew a yellow stripe on the green
surface and laughed at the ants who did not dare to cross it.

The lice were not scared though, no matter how much he tried. They did not
care even if he left yellow stamps on them. Of course, the ants were afraid of
these marked lice and carried them away with great care. Philip felt bad about
scaring the ants but could not stop playing this game. He was running around
and stamping the aphids until the whole herd had been marked with yellow spots,
and the ants obediently carried them all off.

``I am sorry,'' he said to Tony. But the ants had backed away from the bud, and
Philip was left alone.

\secbreak

Mary reached the base of the tree. The trunk was at its widest here and the
roots burrowed into the ground. She managed to find some fine clay on the way
down, but she needed at least thrice as much. Before descending to the surface
she looked around from atop a thick root. It was a scary thought to leave the
tree. Detritus covered everything around her. Dry, dead leaves. Fungi gnawing
on their surfaces. The leaves at the bottom were in the worst shape. They might
have fallen off last year. If someone dared to approach them, they might have
found ruins of fairy cities from the old year.

Mary heard a noise. A sand lizard ran through the detritus, pushing whole
leaves to the side along the way. A large animal like that could break
Dewington right off the branch. The lizard ran toward the setting sun. She saw
the dark silhouette of an even larger animal against the sun. It was a deer
tearing off a twig from the next tree. As she watched it chewing patiently,
Mary could only hope that no fairies lived on those leaves.

She gathered her courage and descended further on the root. The bark was
breaking up, and she walked in its long grooves. The grooves were deep as
canyons. Remains of the old bark watched over them. The highest parts of the
bark were from the first years of the tree. Worn and dried by countless winters
and summers. They had long lost their defenses against fungi, and a lacework of
colorful lichens cast shadows on Mary.

``Little miss fairy! Fairy girl! Where to, where to?'' a voice called down from
the lichen. Had he not called, Mary would not have noticed the fat gray mite.
His obese body rested on a couch woven from the mycelium. His eight feet
dangled around the edges of the recess. One of them lazily dug into the dense
weave of fungal filaments and pulled out a lump of green alga from the bottom
layer.

``I will answer you if you can answer a riddle for me,'' said Mary. The mites
were famous for never worrying about anything. They were not much bigger than a
fairy but were as strong as an ant. So they had no one to fear and could take
anything they liked. They lived their comfortable lives peacefully and were
always up for a game. ``What are clouds made of?''

``Oh, little fairy, to ask that of a mite,'' the mite above laughed and
swallowed the lump of alga. ``My cousin was a famous traveler. The wind took
him everywhere, even above the clouds,'' he continued with his mouth full.
``The cloud is nothing more than a lot of water droplets that the wind holds in
the air. They may be about the size of your head, little girl. If they grow big
enough to fit you, the wind can hold them no longer and they fall down. That is
the rain!''

``I will be a traveler too when I grow up,'' said Mary, ``and I will go up
above the clouds. I will see then whether you were right or not. I will come
back if you were wrong!''

``I will wait here, sweetie. But now tell me, where are you going? The fairies
live on the leaves above. Have you perhaps dropped something?''

``I am searching for clay. We will build a rocket nozzle from it.''

``You want to search among the roots?'' the mite asked in surprise. ``Did you
remember to bring your lantern? It is pitch dark down there, little fairy. You
know what? If you can answer my riddle, I will give you a lantern. I found it
between the gills of a milk-cap. A firefly must have lost it. It can be yours
if you answer right. What does it mean when a firefly is blinking in yellow?''

``It is signaling a turn,'' Mary said right away. She got the lantern from the
mite too. It was a hairball that smelled like a mushroom and glowed faintly in
the twilight.

Mary thanked the mite for the useful gift and continued on her way. The root
curved to the right, curved to the left, then curved down. Mary cautiously
followed it down under the big black lumps of earth. She saw nothing until her
eyes became accustomed to the darkness. The air was musty with the smell of
mold and bacteria working in the soil. Much moisture was trapped in the rich
structure of the ground, and Mary was wary of falling into a muddy pool. It
would be the end of her pretty dress and of the clay collected so far.

As the root passed under the ground, fine hairs stemmed from it. They ran out
into the mysterious world of the soil and absorbed all the moisture and
nutrients that the tree needed. Mary descended further. The topsoil had a lot
of decaying plant matter and tiny creatures were having a feast all over it. It
would have been difficult and gross to push them aside to reach the mineral
clots.

Deeper still a long discarded pod of chitin blocked Mary's way. An insect
larva wore it years ago, chewing on the root, until it grew large and cast off
its skin. The root had grown thicker since then and pressed into the abandoned
husk. Mary put her lantern, the glowing knot of hair, on her head as a hat and
pulled out her sharp quartz knife. She pierced the barrier and cut a round hole
into it with a quick motion. As she was rolling up the patch of skin, she saw a
red light deep below. She fumbled in surprise and got tangled in the thin
chitin. She struggled to break free and by the time she cast off the larva
skin she was tumbling down into the hole she had cut.

To stop her fall, Mary stabbed the knife into the root surface rushing by. She
slowed, then stopped and finally stood on a root hair. A clear drop of water
seeped out from the scratch the blade had etched and fell on her head. She
shivered. Where did the red light come from? It was nowhere to be seen now.

Mary balanced herself on the narrow root hair. The white path wound among
large, cool nuggets. A patchwork of black hyphae covered their sides and
filaments stretched in the space between them too. She held her sharp knife
ahead and opened a path into the darkness.

She had just gotten used to shredding the black strands like a windmill when
they ran out. Her arm swung at the empty air for a moment, then she stopped and
looked around. The strands stayed on the right because they found nothing to
digest on the left. The nugget to the left shimmered bare and gray. Clay! The
quartz knife went to work again and Mary's pockets started filling out.

\secbreak

Meanwhile, Philip arrived to the ranch under Solperla. There was a great deal
of excitement there. They had found a big aphid, to the head of which someone
had tied a pair of wings. And as they drove it toward the farm they saw another
one. When they threw a lasso on that aphid, an ant had just carried up one
more. The ant put it down at the border of their empire and refused to let it
climb back. It stood in the aphid's way, until the fairies led it off to their
ranch. This went on all night. The flock the ants returned was twice as large
as the flock the fairies had lost. And on the side of each aphid the yellow
prints of five tiny fingers gleamed.

Once they saw Philip's yellow hands and heard his story, all the fairies, big
and small, celebrated him. They set a long table right away, and everyone
brought something delicious from their pantries. With such a large flock they
no longer feared the winter. They served soured insect egg soup and fried alga
rolls for starters. The next dish was balls of cheese wrapped in thin slices of
smoked aphid ham. And while the fairies feasted, the dessert was prepared as
well. They cut a large pollen in half and filled it with fruit pulp. One half
was prepared for the children with fresh pulp; the other half for the adults
with partly fermented pulp.

The feast lasted all night, but Philip still had important things to do. He
asked Judy for help, and together they milked all the lice. They gave more than
enough honeydew to fill up all barrels.

``What a big flock you have brought us, Philip,'' Judy mused. ``We need to
build more barrels. We will need a larger pen too. It will be a lot of work. We
cannot let our young ones go to Solperla to try their luck now. The leaf may
turn out to be too small too. We will have to find two leaves in the spring. If
they lay many eggs, perhaps even three.''

When the barrels ran out, they collected the honeydew into baskets,
tablecloths, and bedsheets. Overall they got more of it than there was water in
the dewdrop of Dewington. The only remaining question was how they would
ferment and distill all of it.

``It is good sweet honey the new lice gave,'' said a fairy with a mustache,
``and the evenings are still warm. If we add the yeast now, it will get fizzy
by the morning.''

``But how will we distill it?'' asked another fairy, who could have been his
twin brother. ``You cannot kibble that much dew mead by hand.''

They argued on, and Philip learned of more tricks of the distillation process
than he ever could have imagined. For instance, after fermentation the dew mead
was usually kibbled into small beads so that it would evaporate quickly. They
were trying to figure out something more efficient, but everyone agreed on one
thing. If they wanted to let the honeydew ferment by the morning, they had to
add the yeast now. All the fairies who were on a short break from the
celebrations helped with this. And Philip headed home.

\secbreak

The clay weighed so heavily on Mary's pockets that it was pulling her skirt
down. The clever solution she found was to fashion suspenders from a few black
fungal filaments to hold it up. The nugget of clay she had been mining was
covered in scratches from her quartz knife. Her hard work had stripped off the
surface to the depth of three fingers at least. The clay was entirely pure.
Neither grains of sand nor plant-matter-digesting hyphae were to be found in
it. This was exactly what Mary needed for the nozzle, and she was very glad to
have found the nugget.

One pocket was full and the other was getting there too when, to Mary's
surprise, red light traced her knife's path. The dim radiance had to have come
from inside the nugget. To find its source, she twisted the knife to enlarge
the hole. No matter how much she peeked she only saw clay illuminated in red.
She continued to work until she had enlarged the hole to the size of her hand.

Sticking her face to the opening she could now see that the nugget was hollow.
She had cut a funnel-shaped hole into a span of clay wall. On Mary's side the
hole was as big as her hand, but on the other end it was as if she had poked a
hole with a finger \dots\ As if she had poked a hole with a finger through the living
room wallpaper of a fairy. She saw the back of a couch. The couch faced a
fireplace. Compressed blocks of cellulose smoldered inside and radiated a warm
glow upon the room. Neatly framed pictures lined the mantelpiece. Mounted on
the wall above them the scary head of a worm stretched its mustache toward
Mary. A red-striped wallpaper covered the walls. They bulged outward a bit, but
the room still had corners. In one corner a well-shaped fungus grew in a clay
pot. The yellow mushroom was practically glowing against the red wallpaper.

A fairy stood up from the couch and turned to face Mary. His outline was
gleaming from the light of the embers. He was tall and burly. The wide
shoulders carried a bald head, but on both sides a thick mustache broke its
dark circle. It was just like the mustache of the trophy above the fireplace.
A black cloth covered his right eye. The left one watched the hole closely. As
Mary recoiled, the stern face flashed green from the light of her glowing hat.
The fairy girl screamed and ran.

She did not run far before the hyphae she had cut earlier slapped into her
face. She lost her balance and almost fell off the narrow root hair. She clung
to the web of filaments and hurried on, grabbing them every few steps for
balance. Her heart beat so loudly she could not hear anything else. But she
imagined that the fairy with the big mustache had knocked down the wall and was
furiously galloping behind her. She dared not look back, not into the single
eye flashing in green. She stumbled toward the root, away from the soil, back
to the leaves, back to the well known world of Dewington.

She thought she had lost her way in the black jungle of hyphae, when she
finally saw the pale column of the root. But just as she started to run toward
it, a terrible shadow formed ahead of her. The one-eyed fairy was riding a long
slithering mass as thick as her waist. In his eye the cold reflection of her
green lantern danced. The blood froze in Mary's veins; she was paralyzed by
fear. The monster pulled a red light from under his jacket and held it up high.

He looked just as evil in the red light as Mary had imagined. Or at least
almost as evil. A little less evil, because the mouth under the thick mustache
formed a wide smile.

``Steady, Anaconda,'' he soothed his steed. ``You are scaring our guest.''

\secbreak

The sun was setting. In the long shadows that were filling the grooves
of the bark Philip saw serpentine shapes.

A short while earlier he knew no fear and swung from one thread of the spider
web to the other, whistling a song. But in the sanctuary, among the intricate
carvings, he fell silent. As he had hoped, he received help from the peculiar
ant. But he did not quite understand what happened. The ants said he did
something bad, but the fairies said he did good. He did not know what his
parents would have to say about it. As he turned the thoughts in his mind, he
felt relieved at times and guilt-ridden at others. But, as the shadows grew
longer, the pride waned and the guilt waxed. Did the ants have a full pantry?
Would they have something to eat in the spring?

The world of the bark was busy at dusk. The stalking forms of the night had
risen but the alert beings of the day had not retired yet either. Everyone was
hunting or being hunted at this hour. Protozoa were running from waterbears.
Mites were wrestling nematodes. Hyphae snatched the fallen. And the
ever-changing shadows of conscience chased Philip.

He ran from the long shadows and wet sounds as fast as his legs would carry
him. He wanted to be back in Dewington. He was tired, but if he slowed down the
sounds came closer. Sticky hyphae caught his foot in a ditch. While he pulled
it free, the shadows caught up. And they passed him. An army of diversely
shaped protozoa and other small creatures slipped and crawled by him. They
ignored the fairy as if they were hurrying to Dewington as well, so they would
be home by dinner too. Philip freed his leg and looked after the hairy, sticky,
tentacled crowd in bewilderment.

Unexpectedly, a shadow fell upon him from behind. So that was why the protozoa
fled! A big yellow wall rose above the cliffs of the bark and raced toward
Philip. It slowed down, stopped, and then took off again. Philip stopped
staring at it and broke into a sprint. The slime mold must have crawled into a
dark, dank corner during the day, and once shadows covered the bark its time
had come. The huge, throbbing cell pushed itself forward with each pulse and
digested everything in its path. That was why the tiny inhabitants of the bark
were running. A dozen fairies could not have stopped it.

Philip tried to keep to the last sunny spots and hop across shadows. But when
approaching a big outcropping of bark he had no choice but to step into the
dark. It was dangerous to run blindly on the uneven ground, but he could not
slow down. The slime mold picked up speed in the cool shade and was on Philip's
heels with each pulse. Philip ran close to the cliff, which offered some
shelter from the slime mold. But when the cliff ended he had to run through an
open field. He had almost reached the end of the shadow when his foot got
caught in another ditch and he stumbled. He tried to jump back on his feet,
but his shoe had become stuck. This was the slime mold's chance, and its yellow
wall rushed toward Philip.

Suddenly, a big ball of fur popped out of nowhere and jumped against the wall.
The ball bounced back like rubber, but the impact also stopped the wall for a
moment. A ring of waves spread on the surface of the slime mold. The waves
did not get far before the ball of fur slammed into the wall again.

``Run, Babouche!'' shouted Philip when he managed to untangle his foot.
Paramecium and fairy ran side by side from the yellow wall. Fortunately, the
bark became smoother and the shadows fewer as they approached Dewington. The
light of the setting sun forced the slime mold to stop there, and the two
friends merrily chased each other until they got home.

\secbreak

The dewdrop had long ago evaporated from the main square. The morning's crowd
of fairies had evaporated too. There were no vendors, and no one came to fill
their canteens. At dusk the streets emptied, the doors and windows shut. Philip
saw just one old fairy sitting on his porch, but he did not return the boy's
greeting. He must have fallen asleep and would only wake up when the early
chills of the autumn pinched him.

Yet the square was not empty. A big pile of cellulose and chitin littered it.
Plant fiber and spider silk wove through the heap. Twelve barrels of brandy lay
strewn about. And from one of the barrels a monstrous snoring was heard. The
barrel rocked back and forth lazily to the rhythm of the snore.

``Hello,'' said Philip timidly.

There was a pause in snoring, then the barrel resumed rocking. With Babouche by
his side he moved closer and peeked into the barrel.

``Raisy!'' he exclaimed.

``Leave me alone,'' snorted the barrel. Babouche was excited to hear the
familiar voice and jumped in by Raisy. It must not have been comfortable for
either of them because a lot of tossing and turning followed. In the end the
barrel started to roll and hit the pile of building materials. Raisy tumbled
out of it with Babouche on her head and lay spread out on the surface of the
leaf.

``Captain Philip,'' she perked up a moment later, ``we have evaluated two dozen
different rocket hull designs, sir. And,'' she added almost inaudibly,
``emptied a dozen barrels of brandy in the process.''

``This is great news, navigator Raisy,'' responded Philip. ``Have you found the
perfect design then?''

``We have \dots\ some more ideas. And tell me Philip, how did it go with the
ants? Did you get them to return the flock?''

``Yes. And they returned more than had been lost,'' he boasted.

``That is great news, Captain!'' she said. Then her eyes widened. ``What? Did
they really return the lice? What did you do, Philip?''

Philip told her what had happened on the pasture of the ants. To share the
story they climbed on top of the collapsed buildings and enjoyed the fresh air.
Babouche curled up on their feet and, faithful to his name, kept the tiny toes
of the fairies warm. Raisy listened carefully to how much the ants had feared
the golden powder and how they carried the flock up to the fairies. When Philip
went on to describe the feast, she got rather hungry. But only a small clump of
dried alga remained in her satchel.

``A slime mold so high in the tree?'' she gasped when he reached the end of the
story. ``You were lucky that Babouche found you. I also met a slime mold when I
was about your age, Philip. My brother Gil was even younger and a restless
little rascal. Together we ambushed a flea and rode it around the tree. There
was not much to hold on to on its smooth back, and Gil clowned about until it
threw us off with a jump. We landed in a shady cleft in the bark and laughed so
much we could not see from the tears. Suddenly Gil stopped laughing. When I
noticed that I was laughing alone I opened my eyes. The remains of a mite lay
beside us. Something had completely stripped the exoskeleton. Under our feet
the decayed cell membranes of protozoa covered the bark. We did not know how
these tiny creatures might have perished, but we were very scared. We tried to
tiptoe out of the crevice, when suddenly---''

The stars had disappeared from the sky during her story, and the darkness
deepened. When she came to the scariest part, lightning flashed. The lightning
illuminated a terrible serpentine form in the main square of Dewington.

``Ah!'' screamed Raisy and Philip. They staggered from the sight of the monster
and tumbled from the top of the experimental rocket hull. The thunder and
shouting woke up the old fairy sleeping on his porch too, and he stared into
the darkness. By the next lightning he too could see the half-snake, half-fairy
form. He stumbled over his bench, fled into the house, and slammed the door.
And then shuttered the window too.

``This is scarier than the slime mold,'' said Raisy from one of the barrels.
Philip snuck back to the top of the pile and stuck his head out to peek at the
terrible squirming form. The clouds had covered the moon and the stars.
Everything was black. But Philip had a good eye, and different things were
different shades of black.

The winding form had to have been that of a large nematode. Three dark figures
were riding on its back. One was a big, brawny fairy; another was a small,
fragile fairy. But the largest figure sat between the two. Shapeless arms,
tentacles, and claws protruded from it in every direction. It really was scarier
than the slime mold, but Philip boldly crept closer. He heard a whisper from
the black riders.

``Did you see Raisy tumble down?'' asked one.

``I did! And did you see Philip's face in the lightning's flash?'' asked the
other with a suppressed chuckle in her voice.

Philip recognized Mary's voice right away, but he did not want to miss a good
chance for a scare. In the dark anyone could surprise anyone. Mary and her two
mysterious companions on the back of the worm slithered toward the center of
the square while Philip took a small detour to get behind their backs. He put
his empty backpack on his head and grabbed the unsuspecting Mary.

A great deal of screaming, wrestling, and chasing was the result. By the time
they were both so exhausted that they could no longer be scared, Raisy had also
emerged from her barrel, and Mary introduced her companions.

``This is Anaconda,'' she pointed at the ever-winding worm. ``And this is
John.''

The bald fairy climbed off his steed and pulled out a glowing red lantern.
His left eye glowed in the light but the right eye was covered with black felt.
Under a big nose his long mustache covered a friendly smile.

``I am not the famous hero,'' he said, ``just his namesake. I apologize for
scaring you. It was Mary's idea.''

``I had to scare someone after being scared myself,'' Mary said. ``John lives
underground and is a fearsome hunter. When we met I was so afraid of him I
wanted to run away. He just wanted to talk, however. So many exciting things
had happened to him under the ground, but he is the only fairy there, and there
is no one to listen to his stories.''

``And your other companion?'' Raisy said.

Mary pulled out her green fur lantern and shone its light on the third rider.
The misshapen figure was neither a fairy nor any other creature. A big pile of
clay was molded into all sorts of appendages. They had used strips of bast
fiber to tie it to Anaconda's back. In the green light all its terrifying
features had disappeared. It even had a big smile sculpted on its face.

``This is Jimbo,'' beamed Mary. ``Until recently, he was the wall of John's
living room. But today he has decided to become a nozzle!''

Neither Philip nor Raisy understood much of that. Mary had to tell them
everything on their way home. Mary's parents invited John inside, and Philip's
parents sat Raisy down to their dinner table. Conversation filled the two small
houses on the tip of the leaf. The children told of their adventures, and the
polite travelers compared the dark world of the soil and the distant world of
the pine tree with the green leaf of Dewington. Slowly the light of the
lanterns faded, and everyone grew curious of the next day. They said goodbye to
each other and all went to bed.

\secbreak

Philip turned in his bed twice, then had a better idea. He picked up the
looking stone from his nightstand and snuck out to the back of the house. Mary
was sitting on the edge of the leaf already, her feet dangling above the abyss.
He lent her the looking stone and gazed thoughtfully at the black leaves
sprawling below.

``Mary, where would you like to fly the most?''

``To Raisy's pine! She promised to teach me how to bake a resin cake.''

``What about after that? If the rocket works, we can fly it anywhere.''

``Oh, that's right,'' agreed Mary. She started to think on it now. ``There is
the raspberry bush by the big rock. The wind always blows delicious smells from
there, right? Do you think it could take us there?''

``Of course it could! We would have to be careful with the landing. We would
not want to get caught on the sharp thorns. But the leaves are safe. And where
would you fly from there?''

``Back to Dewington. We could bring back a whole berry, and everyone could have
some of it.''

``All right. But after that we will fly up to the tip of the old tree that
stands on the hill top. From there we would see trees that we cannot see from
here. And we would go on until we reach the edge of the forest.''

``And maybe we would find another forest there. One which is full of raspberry
bushes. And blackberry bushes.''

That is how the two little fairies planned their trip until they grew sleepy.

\secbreak

Philip dreamed that he was little. As small as one of his eyelashes. He was in
his room, adventuring within the forest of the soft fibers in his carpet.
Suddenly it became dark, and as he looked up he saw that Mary was sitting on
the bed. They were in Mary's room and not Philip's.

Philip climbed to the top of the carpet and leapt from one fiber to the next.
From red to yellow, from yellow to white, from white to blue. Then he realized
that if he jumped on the colored threads in the right order, his shoes would
become brighter. Mary held the looking stone in reverse, so she could examine
the minuscule, radiant shoes. The only light was now coming from the shoes, and
the tiny Philip cast a giant's shadow on the wall of the room. The shadow was
so big that it could lift Mary in its hand.

Philip, the shadow giant, rode on the back of a giant ant. He wore a golden
yellow robe, and his steed had the same yellow color. But Mary saddled a lizard
and met Philip's army of golden ants on the plain below the leaf. The lizard
swept away dozens of ants with its tail, but then made peace with them. She
helped up the little ants and fed them raspberries from her saddlebags. The
ants in turn helped her build a huge machine shaped like a fairy. Its foot was
bigger than a whole leaf. Its shoes were built from a dozen large snail shells.
Its leg was a tree trunk. It was so tall that it could reach the lower branches
of the tree while standing on the ground. On top of its neck it wore Philip's
smiling head.

``But what is this giant machine good for?'' Mary asked.

``For playing tag!'' Philip exclaimed and lunged for the lizard. But the lizard
was nimble, and the giant's hands grasped only dry leaves. He held the leaves to
his eyes to see who dwelled on them, but did not see anything. The lizard ate
the leaves. It ran circles around Philip and cleaned up the forest floor in his
path. As it sucked in all the leaves it grew bigger and bigger. When it got big
enough, Philip sat on its back behind Mary, and they rode among the trees.

Soon the lizard became as big as a tree, and Philip jumped from its back onto
the top of their tree. He balanced on the uppermost leaf of the tree just as a
rocket rose from the valley. Its light shone throughout the forest, and all the
leaves fell off. Philip also began to fall, but the rocket quickly changed
direction and caught him mid-fall. The rocket was big as the giant lizard, and
covered in complex winches, ropes, and arms. Mary opened a hatch and let
Philip inside. They flew up into the sky and flew to the sun and the moon. It
was autumn on the sun and all the leaves were yellow. It was winter on the moon
and white snow covered the trees. Only one tree was green, Raisy's pine. The
kind lady waved at them from her round window cut into the side of one of the
needles. They skillfully landed on her doorstep. Since everything was upside
down on the moon, when they got out Philip began to fall toward the earth. Mary
did not understand why he was falling toward the earth and drank a cup of tea
with Raisy.

Philip was falling and falling and getting smaller. When he fell from the moon
his foot was as big as Dewington. But when he reached the branches of their
tree, he was only as big as an ant. By the time he fell on the roof, he had
shrunk to the size of a fairy. He fell right into his bed and sat up. His
mother offered him a cup of tea for breakfast.

\secbreak

Raisy was sitting on the terrace of a café in the main square. A sheet of
cellulose paper covered her small, round table like a tablecloth. The
calculations were already flowing off the edges of the table, but the pencil
was still busy in her hand. One formula led to another, and her model of the
behavior of different rocket hulls became more accurate. She calculated the
forces which would affect various parts of the hull during takeoff given
different designs. A hull of greater weight could be stronger but at one point
would start to have difficulty supporting its own weight. When Raisy was young,
she was taught how the different varieties of cellulose and chitin held under
pressure. During her work as an engineer she learned about using support beams,
columns, arches, and domes to strengthen a structure. Now she was making
calculations to know how each design would work in this case. After all the
unsuccessful experiments last night, she preferred to do the analysis on paper.

``Let me see where this derivative becomes zero,'' she muttered when Philip
joined her at the table. ``I~am almost done with the calculations, Philip. No
need to worry about the rocket hull. I will be ready by noon at the latest. I
am going to have one more coffee.''

Philip did not want to disturb her. It rained overnight, and the raindrops
washed away the half-built rocket and the empty barrels. Nothing but a huge
ball of water sat in the middle of the main square. It scattered the light of
the rising sun in every direction, but it was especially generous to a bench on
the edge of the square. It cast a shadow on the houses around the bench, but
the light that it took from the houses it gave to the bench. At sunrise only
the old fairies who woke early basked in the warmth of the focal point on this
bench. But over time a group of children had joined them. In the end their old
ears started aching from the squeals of the younglings, and they let the
children have the bench to themselves.

As the sun rose higher, the hot spot slowly moved closer to the center of the
square. It had mostly crept off of the bench, and the crowd of children was
trying to push the bench back under it. A blonde fairy boy took his chance for
a bit of sunbathing while the others struggled, causing a tug of war to erupt.

Once they had made peace, Philip helped out with moving the bench. Soon they
could all enjoy the comfort of condensed sunlight. They had worked up a sweat
in the effort, but the tiny droplets quickly evaporated from their brows in the
heat.

``That's it!'' Philip exclaimed. ``We can easily distill the mead here!''

``The what?'' asked the blonde boy. But Philip was already on his way to the
shepherd fairies.

\secbreak

At the outskirts of Dewington Philip stumbled into Mary and John. Mary was
sitting on Anaconda's back, and John, with the reins in his hand, was teaching
her how to ride. Philip saw the nematode in daylight for the first time. Its
skin was bright red and smooth like a drop of water. A girth of sticky hyphae
held the saddle on its muscular body, and John explained how to control
Anaconda by the bridle on its nose.

After Mary was done, Philip too learned how to ride the gentle worm. Meanwhile,
he told Mary and John of his plan to distill the rocket fuel in the focal point
of the big drop in the main square. Mary in turn explained how the nozzle would
be prepared.

``We will need two tamed nematodes. One of them slides around in a small circle
while the other one makes a larger circle around it in the opposite direction.
We place wet clay in between them, and their smooth skins will shape it into a
thin ring. As we add more clay, the pressure from the worms will squeeze the
moisture out of it, and a solidifying wall of clay will rise between them. We
carefully guide both worms to gradually widen or narrow the inside and the
outside. This way we can build the nozzle in the exact shape that Raisy has
engineered.''

John and Philip shared appreciative glances. Mary had indeed put a lot of
careful thought into the method of nozzle construction. Fairies could use the
same method later for building large tanks and kegs too. With more worms and
enough clay they could even use it for constructing buildings.

They all took off on Anaconda's back to capture nematodes for the children.
Even with the three fairies on its back, the bright red worm slid on the uneven
surface of the branch like the wind. The fairy kids enjoyed the rush, but all
too soon they reached their destination.

Moss grew on the northern side of the tree even at this height. In the tangled
forest of moss stems moisture was trapped, and the dark water was teeming with
life. Tardigrades frolicked, amoebae exercised, and a mosquito larva filled one
of the larger droplets entirely by itself. And on the wet clods wild nematodes
slipped from shadow to shadow.

``They move in swarms, you see,'' John pointed at the worms. The dark, dank
world under the moss was his world. Each animal reminded him of a past
adventure. Philip and Mary listened in awe. ``This yellow-striped swarm is too
wild. Can you see the teeth? It would rather tear my arm off than let me ride
it. The white ones in the drop over there are easy to handle and very good
swimmers, but they cannot deal with dry land. But up there, high up in the
moss \dots\ Do you see the red flashes?''

Green light filtered down into the forest of dark stalks. The moss leaves at
the very top bathed in the sun and covered the blue sky entirely. As Mary and
Philip strained their eyes, they saw something red appear and then disappear
immediately among the moss leaves.

``Those are relatives of Anaconda. Agile, strong worms. They love alga. Come
on, let us collect some alga. If we trap a worm and then hand feed it, it will
turn into a tame steed right away.''

They fished out green, fist-sized balls from a bright drop of water. Philip
tasted one, and it was good and juicy; just a little tart. John also taught
them which hyphae made a good harness. This fungus was a danger to the wild
worms, because they could easily get stuck in its adhesive threads. The fairies
carefully coiled up a long hypha and started climbing toward the moss leaves.

The red worms avoided them at first, but could not resist when they saw the
beautiful green balls of alga. They circled the fairies and snatched the algae
from their hands with lightning fast movements. John tied the hypha into a loop
and held it in front of the alga in his other hand. When the worm took the
alga he pulled on the rope, but the worm had avoided the noose or was just too
fast. John loosened the noose and tried again. The second time he got lucky,
and the sticky thread fastened on the midriff of the worm. John paid out some
rope until the worm ended up orbiting them on a leash.

``And now we carefully draw it in,'' John said. ``It could easily drag off all
three of us, but the moss stems are even stronger. I tie the thread around a
stem like this,'' he demonstrated. ``And note this knot. We can draw on it, but
when the worm pulls from the other direction it will hold.''

He used the wily knot to pull the worm gradually closer. The other members of
the swarm recognized the trap and bolted. Left on its own, the scared worm
soon tired of struggling and gave up. Trembling, it awaited its fate. Philip
held an alga to its snout. The worm sniffed it uncertainly then accepted it.
Philip fed it some more and scratched its back like John advised him to. The
nematode calmed down and let them turn the loop around its back into a simple
saddle and tie a bridle on its head.

Philip named his new friend Rocket. Anaconda gave it a friendly sniff, and
after a bit of hesitation Rocket allowed Philip to lead it on a leash. It was
slightly slimmer than Anaconda, but it still had the strength of several
fairies. Soon it allowed Philip to sit on its back, and until Rocket got used
to the reins they zigzagged between the moss stems with abandon.

Once they learned to understand each other, they set out to catch a worm for
Mary. On his nimble steed Philip helped steer a swarm of red worms toward John
and Mary.

``Good luck,'' he waved to Mary as she baited the worms with an alga. ``We will
meet in Dewington. I will bring down the mead and by the time the nozzle is
built we will have the fuel as well.''

``Good luck, Philip,'' they waved to Philip from the ring of red worms as he
rode away on Rocket.

\secbreak

Philip's knees were a little weak when he dismounted from Rocket on the
shepherds' leaf. The speed of the worm was terrifying in the rifts of the bark,
but they arrived safely. The fairies of the leaf admired his shiny red steed
and greeted Philip cheerfully.

``Captain Philip,'' said Alexander, one of the mustachioed masters of honeydew
fermentation. He picked up the habit of addressing Philip as \textit{captain}
from Raisy, but a playful esteem rang in his voice. ``I~took a look at the
barrels just now. They got good and smelly. The honeydew fermented finely. The
only problem is that we could not knead this much mead in the whole day.''

``And we will not have to,'' Philip said while tethering Rocket to the stables.
``I~think we can distill it in Dewington. There is a big drop of dew on the
main square of our leaf which collects the sunlight. If we put a barrel in its
focal point, it will warm up right away, and the spirit will rise up.''

``What do you make of that, Alexander?'' asked Alexander his partner, with whom
he shared not only the task of making brandy but the name as well. ``Could we
distill spirit like that?''

``It could work if we kept stirring the juice. If we hit the right temperature,
the spirit will part from the water.''

Once they exhausted this topic, the discussion turned to the question of how
one hundred barrels full of mead could be transported. Half a dozen lice could
have carried them, but running a caravan through the empire of the ants was a
great risk. Philip did not know how many barrels Rocket would be able to carry,
but certainly much less than an aphid.

``Nice nematode,'' Greg spoke from behind Philip's back. Philip turned in
surprise to see his spider friend hanging upside down on a fine thread of silk.
Greg focused his eight eyes on Rocket.

``Good morning, Greg,'' Philip greeted him. ``I~hope last night's rain did not
soak you.''

``We took shelter under the branch. The web always fills up with food before
the rain, so we do not regret the inconvenience. Mom is cooking a delicious
lunch right now. Want to play a game until she is done?''

``Did you see how the fairies in Solperla swing their cargo from one end of the
leaf to the other on threads of silk? Let's play that! We will be the fairies,
and these barrels will be the cargo. We have to deliver them to Dewington
below.''

``To Dewington?'' Greg asked slowly swaying.

They hatched a plan together. Dewington was not exactly below the pasture, but
was visible from its edge. They piled up the mead-filled barrels, bowls, table
linens, and blankets at one end of the leaf, and Greg wrapped them into a neat
package. They attached the package to the other end of the leaf with a long
thread of Greg's silk and let the loose thread fall off around the edge of the
leaf, so that it would run below it.

``Are you ready, Greg?'' asked Philip. They had pushed the cargo to the edge of
the leaf together and only needed one more nudge to let it go.

``Come on, let's go,'' Greg replied. With another shove the center of mass of
the package was over the edge. The bundle of barrels and sacks slowly tumbled.
As it rolled over, Philip and Greg climbed on it, and when the spider's thread
went taut, they were sitting on the top.

``We are swinging in the right direction, Greg!'' Philip shouted as they flew.
``But our swing does not reach all the way to Dewington.''

``We are just getting started,'' Greg replied. ``Check out how a spider
swings!''

Greg added more and more silk to their thread. He always added the extra length
as they were reaching the highest point of the swing. He skillfully took
advantage of air movements too, and tightened and loosened the line in a way
that kept accelerating their cargo.

``Here we go!'' Philip cheered. ``Look, Dewington is right below us!''

``Prepare for mooring!'' said Greg. The next time they swung above the target,
he threw himself fearlessly off the shipment. Philip reached for him in a
moment of panic, but then understood what the little spider was doing. Greg
had started a new thread off the swinging cargo and anchored the other end to
the leaf when he landed in Dewington.

``Hold on!'' he shouted from below, but it was too late. The new thread went
taut, restrained the swing of the package over the village square, and Philip
tumbled off of it. The fairy would have plunged straight into the sparkling
dewdrop if not for the webbing that covered their cargo. He managed to hold on
to the knot of spider silk and was dangling from the bottom of the package.

The package of hundreds of barrels of mead was bigger than the dewdrop. All
eyes in the main square were on Philip.

``Philip, do not squirm,'' Greg said, and began to climb up. ``If the mooring
breaks, you will be swinging there all day. And I get a swimming lesson in your
dewdrop!''

Philip obediently stopped moving and hung forlornly from the package. Greg
arrived and attached a new thread to a barrel. Not sparing a glance for
Philip, he descended with the new thread while carefully avoiding the dewdrop
through the use of air currents. He tied the tether to the top of a house and
climbed back on it. He extended three more threads the same way and moored the
shipment perfectly. When Philip felt that not even a butterfly would be strong
enough to break the package free, he climbed over to a tether and descended to
the leaf. Spiders loved hanging from a silk thread all day long, but it was not
for Philip.

``I cannot believe my eyes,'' Raisy welcomed them. ``That's enough brandy to
fill a dozen rockets, Philip. And such graceful delivery!''

``This is not brandy, but mead,'' Philip corrected her. ``We still have to
distill the spirit, and collect it somewhere. Over there \dots\ is that Mary?''

Mary looked as if Jimbo, the clay monster, had come to life. The first nozzle
that they built was too soft, and when it started to collapse Mary tried to
hold it together. Unfortunately, they had to start over, but nothing more than
a little bit of time and the clay that had stuck to Mary was lost.

``Yes, it is me. Over there is Raspy,'' she pointed at the clay-covered worm
circling the unfinished nozzle, ``and on the inside is his partner, Cranny. We
are molding the shape very carefully to avoid defects. But we will be ready
soon.''

``May I peek inside the nozzle?'' asked Greg as he landed.

``No,'' replied Mary. ``You did not help Philip to climb down.''

``But he was hanging so comfortably. I hang like that all day.''

``But he is not a spider!''

Greg and Mary continued the debate while Raspy and Cranny shaped the clay.

\secbreak

``I have finished the calculations, Philip,'' Raisy said. They found a place to
sit down among the heaps of notes covering the terrace of the café. Raisy swept
all but one page off the table. She smoothed this one neatly and turned it to
Philip. ``The results for the various constructions. Unfortunately, we get a
number that is less than zero for almost every structure. These would break up
in the air. But this one works.''

Columns of numbers were accompanied by tiny illustrations depicting the
structures.

``A sphere!'' said Philip as he found the winning variation. ``We need to build
a sphere then?''

``And not just any sphere. Only cold-forged chitin is strong enough. If we just
glued or riveted plates together, the chitin would fall apart,'' she said
pointing to the next row of the table.

``I know such a sphere.''

``You do? Who built it?''

``Balthazar, the caterpillar,'' said Philip.

They both rose up and went to have an audience with the king. The evening rain
had soaked the dead surface of the Stain, so it was soggy now. But at least
there was no risk of sudden cracks. A few black bacteria were grazing around
Balthazar's severed head. They scattered when Philip and Raisy reached the door
and knocked.

``Oh, nice to see you, Philip,'' George said when he opened the door on the
caterpillar's head. ``I~was not expecting guests today. The Stain is so muddy.
But come on in. I am George, former king of Dewington.''

``I am Raisy, a friend of Philip's. The gust from the rocket launch yesterday
has flown me here from a distant pine tree. The children are helping me return
home. We are building a rocket of our own and flying to my home.''

George's mouth dropped open.

``I have traveled a lot when I was young. I have even been to a branch of the
neighboring tree. But the pines are on the other side of the valley! Unless the
wind that carried you here is kind enough to take you back, I am afraid there
is no way to the pines.''

Raisy just smiled and pulled out a roll of paper. George sat them at his table,
made a pot of sweet tea, and listened to the grand plan. Although he was not
taught at school, he had a sharp wit and asked many serious questions. Raisy
was happy to have someone check her calculations and explained everything
enthusiastically. George smiled ever more broadly, but was most pleased to
learn that Balthazar's head would be the ideal rocket hull.

``Are you saying I do not even have to set foot outside of my home, and we
will fly all the way to the pine trees?''

``We will need to make some transformations,'' Raisy said. ``We have to fill
the lower level with brandy.''

``Yes, absolutely. That will not be a problem. Where do you have the fuel
now?''

``It is hanging above the main square in barrels,'' Philip chimed in. ``But we
still have to distill it.''

``Let us get to it quickly. I am sure Raisy is sorely missed at home.''

George and Philip headed out to organize the distillation. Raisy stayed in the
head of the caterpillar and started the necessary remodeling.

\secbreak

Dewington residents helped Raisy arrange the heavy plates and beams on the day
before. They also watched with interest as Mary and the two worms struggled
with the clay since the morning. But they had not taken their work entirely
seriously, and most did not even know what their goal was. Now more and more
fairies stopped in the main square and, looking up at the huge bundle of mead,
they asked one another.

``What is in those barrels?''

``I do not know, but it was delivered from up above. From the pasture.''

``It must be honeydew then. But we would not drink so much honeydew in a whole
winter~\dots''

``It is not for drinking! They will burn it all in a flying house.''

``Yes, I heard they want to fly to the Moon, because it is eternal winter
there.''

``Why do they want to get to the Moon if there is eternal winter there?''

A growing crowd of fairies exchanged such questions and hypotheses on the main
square. They nudged each other and fell silent when they saw George arrive. He
looked majestic in his purple armor and big white beard. He smiled and waved at
Mary, then spoke to the fairies.

``Balthazar, the caterpillar, drank our water, chewed on our houses, and
poisoned our leaf. Of the palace grounds only a brown stain remains. Today,
however, we can be grateful to him because he left behind a real treasure in
return. His desiccated head.''

At this point George was surrounded by many a raised eyebrow. Holding up a
hand, he carried on with the explanation.

``Fall is upon us, and today or tomorrow Dewington too will begin to lose its
green color. All the trees turn yellow and red, and all the fairies seek
refuge. But there is a distant tree where these days are just like any other.
The pines do not care about the winter. The fairies insulate their homes with
resin and continue their work until spring. And, thanks to Balthazar, we too
will have a place to stay during the coming winter. We are building a rocket
out of his head and flying across the valley. We are connecting the two trees
with a long line of spider silk, allowing us all to get across. The
blue-skinned brothers and sisters living in the pine will welcome our caravans
laden with mead and spices, and will teach us to build our homes in one of the
countless needles.''

The fairies moved closer and their hearts filled with hope. No one was looking
forward to the winter. The dark and the cold were the enemy of all life.
Waiting for the spring in their tiny shelters was a desperate struggle.

``And what will raise the spirits of our distant relatives the most?'' George
continued. ``The brandy! As our luck would have it, Raisy here was carried all
the way from the pine by yesterday's storm. She says aphids do not live on the
needles. Thus, the blue fairies rarely taste anything sweet and know not the
hearty warmth of spirits. It is a miracle they can survive the winter. So let
us cook as much brandy as we can! Let us load up Balthazar's head with the
golden liquid! And let us build the bridge between the two trees! The brandy
will also serve as propellant for the rocket, but what is left at the end of
the trip will be our present to the residents of the pine.''

The drink was a symbol of rare celebrations on Dewington. They did not produce
brandy locally, and it was a long way from Solperla. The arrival and supply of
merchants from the city was unpredictable. Thus the mention of the drink filled
the fairies gathered in the square with enthusiasm. Now they just wanted to
know how they could be a part of the plan.

``Philip, the little explorer, has ingeniously delivered the mead, from which
the brandy is made, and contrived a process for its distillation. Please follow
his instructions. Philip, where do we start?''

Philip had never been so proud, but as much as he straightened up, he was still
only visible from the front row. He climbed onto the bench in the focal point
of the drop and started to coordinate the volunteers while standing in the
bright light.

As the first step, the children climbed up the threads to the barrels. The
adults gathered below them and caught the barrels as they were released. A
little girl also released from the tangle a drop of mead bundled in a sheet,
but it opened when a volunteer caught it and covered him in the fermented
juice. They took more care after that and passed the bundles down in a chain.

As the cargo dwindled Greg pulled on the threads to keep them taut. Once all
the mead had landed, he tied the threads into one knot in the place where the
bundle had been. The line coming down from the pasture was now anchored in
several points on Dewington, similar to what Philip had seen in Solperla.

``There is a thick branch a little to the east above Dewington,'' Philip
pointed. ``The leaves are warming up in the morning light, but that branch is
still in shadow. If the spirits evaporate from the barrels under it, the brandy
will condense on the cold bark. Come and help Greg tighten the lines so that
Dewington is pulled under the branch!''

Using the method learned in Solperla, they set out to move Dewington by pulling
on the web. All the sweat of the strongest fairies was in vain though. The
web would would not budge a bit. John even let Anaconda help them, but still
they were not strong enough.

``The edge of the leaf is under the branch. Would it not be easier to carry the
mead there?'' said the accidentally doused volunteer. He was the baker of
Dewington, but his usual smell of delicious bread had now been replaced by the
stink of mash, and everyone kept a distance from him.

``No point in carrying only the mead, as we could not distill it without the
drop,'' Philip explained. ``Which is easier? To transport the drop to the edge
of the leaf, or to figure out how to contract these threads?''

They discussed the matter for a while, but since they had no clear plan for
either approach, they could not decide which one was easier. They had to hurry,
however, because once the sun warmed the branch, they would have nowhere to
condense the fuel. They formed two teams. Philip set out to implement the
principle of winches that he had learned on Solperla. Peter, the baker, set out
to roll the dewdrop like a bun.

Greg insisted that all lines were equally tense in a good web, but Philip still
examined them one by one. He even climbed to the roof of a house so he could
pull on the tether, and here he finally felt a bit of give.

``It must have loosened when the shipment was unloaded,'' Greg said. No one
blamed him. They were happy to have found a line which they could hope to
handle.

An older member of their team fetched a hair that must have been shed by a
caterpillar. It was a smooth, strong rod, about twice as thick as a fairy's
finger. They hoped to be able to loop the thread around it, so they could then
start winding it, thereby slowly pulling the leaf toward the branch. Trying to
twist the silk around the hair, the team wrestled with the thread until Philip
fell off the roof.

Anaconda sniffed at the sprawled fairy boy. Philip grinned mischievously,
grabbed his reins and led him up to the roof. At this point the owner of the
house, a pretty young lady, came out to her porch and watched with her hands on
her hips.

``I heard you playing on my roof, but I did not expect to see you trying to
loop a spider's line around a caterpillar's hair with help from a nematode.''

But that was exactly what was going on, and after a bit more struggle they
succeeded at it. They had all worked up a sweat, but after the first twist they
did not have to worry anymore that the hair would pop out of the loop. Two
fairies could grab the long pole on each end now and push with all their
strength to wind the silk thread. Every turn brought Dewington a little closer
to the dark arch of the tree branch.

``It is working!'' Philip rejoiced. ``I~will let Peter know they do not need to
roll the drop.''

The clever baker had not been as lucky as Philip. His team was excitedly
running around the dewdrop, trying to figure out how to rescue Peter from it.
The poor fairy was floating upside down inside the drop, trying to paddle
toward his friends.

``Philip, what are you doing?'' boomed John's deep voice at them. ``That fairy
is about to drown! Where is Anaconda?''

``On the roof,'' Philip pointed.

John cast a surprised look, then called the nematode with a long whistle. He
held up his hand with a short whistle, then pointed to Peter while whistling a
third time. Anaconda immediately understood the task. His pointy nose crossed
the surface of the drop as if it had never been there. He was fast on land but
even faster in the water. He reached Peter in the blink of an eye, coiled
around his entire body and took off to leave the drop. His nose easily pierced
the surface again, but he got stuck when he tried to pull out Peter.
Fortunately, the fairies were there at once and, taking hold of his bridle,
heaved them both through.

Peter spat out a lot of water but was not hurt. He had endless gratitude for
Anaconda. Once he recovered fully, he jumped on his feet and fetched some alga
scones from the bakery to reward his rescuer.

\secbreak

Two fairies lifted a barrel, carried it to the light collected in the focal
point of the drop and, as recommended by Alexander, shook it thoroughly. Due to
the shaking the mead warmed evenly and started fizzing once it was warm enough.
The brown liquid bubbled up and let out an intoxicating smell. Then the foam
collapsed and the barrel stopped fizzing. They carried it off and the next pair
of fairies with the next barrel took their place in the sunlight.

Philip and Greg were watching the procedure from the bottom of the branch above
Dewington. Greg's claws were sharp enough to hold on to the surface of the
branch, while Philip sat very careful on the edge of a plate of bark, dangling
his feet. The shade was quite cold, and Greg regretted that his winter coat was
not done yet.

Soon after the fairies began the distillation, the stinging smell reached the
branch. Upon contact with the cold surface the spirits condensed and formed
tiny globules. Philip was watching eagerly and still did not notice them right
away. The dew formed droplets too small for the eye to see. The droplets
covered everything, and whenever they touched each other they fused into a
larger droplet.

``Look, Greg! You are covered in golden beads!''

``So is your hair, Philip! There is a little bead at the end of each strand!''

They both swept up the beads and merged them into a radiant orb that just fit
in Philip's palm. As he rolled the ball around in his hand, its size grew ever
so slightly.  At the same time new droplets appeared in his hair. The brandy
also collected on the tips of the bark and formed balls of a similar size to
the one that Philip held in his hand. When he touched his ball to one growing
on an outcropping, they merged, and a larger drop now sat trembling on the
ledge.

Philip and Greg both enjoyed the game. They realized that merging the droplets
was not just fun, but also necessary. The small droplets evaporated, the large
droplets stayed around.  Greg made a small web and dragged it across the bark
with two feet while he crawled around with the other six, sweeping up a large
golden sphere. Philip visited the internal cavities of the bark and ousted the
pearls that had collected there.  They merged the drops into one, which kept
growing with every barrel that the fairies distilled below.  When it grew
larger than Philip, Greg fastened it with a couple loose threads to be safe.
They did not restrict its growth, but made sure that it would only drop down
when they were ready.

The boys kept hunting after the beads until the barrels ran out below. ``That
was all,'' shouted George from Dewington. ``That was the last barrel. Raisy
says the drop looks great!''

Indeed, it was rather large. The unusual smell drew out the inhabitants of the
tree. Creatures of all shapes and sizes were looking at the golden drop. Its
rotund form in the shade was about the same size as the sparkling dewdrop in
the main square of Dewington below. A ring of fairies surrounded that
lower drop, gazing up at the one above. The unusual sight made the fairies
in the small streets halt too, and filled them with delight. Someone had
climbed on the top of their house to watch the drop hanging from the branch
above.

The bark of the tree also filled with gleaming eyes fixed on the fruit of the
morning's hard work. Spiders, mites, tardigrades, caterpillars, larvae, and
even adult insects watched curiously. The midges danced excitedly around the
brandy, and the hum of their wings filled the air.

``Peter, you can start unwinding,'' Philip called down. ``Move Dewington to the
second stop!''

The team on the roof of the house, responsible for winding the spider silk,
grabbed the rod of hair again. They turned it in the opposite direction this
time, but it was still not an easy task. If it slipped out of their hands, the
knot would unfold and the leaf break free.

``And here we are,'' said Philip shortly. The drop of fuel hung precisely above
Balthazar's head.

\secbreak

``Here they are,'' said Mary from the caterpillar's eye. George had built two
well-lit reading rooms beneath the enormous eyes. The comfortable couches were
now glued to the floor with mucilage. The fairies collected the strong glue
from a particularly slimy variety of alga. Raisy and George had fixed all the
furniture and ornaments in preparation for the flight. Mary had even renamed
the rooms. The reading room became the cockpit. The kitchen on the ground floor
became the machine room. Raisy was busy installing the nozzle there.

``Just a moment,'' Raisy said. ``Let me check the sealant once more, and then
we can load it up.''

They had glued the nozzle in place with mucilage as well. Raisy rapped along
the bond and was pleased to find that it had set well. The opening of the
nozzle was stuffed with wool from parameciums, and a long wick ran outside from
it. Raisy inspected this as well, then climbed up the stairs.

``Come Mary, we better watch this from the outside.''

They scrambled down the hastily constructed stairs from the top floor to the
brown swamp of the Stain, then retreated from the huge skull and started
jumping and waving.

This was the signal Philip and Greg were waiting for above. To make sure the
precious drop would fall straight they waited until the air stood completely
still for a moment, then released the silk tethers from the two sides at once.
The drop started to stretch and then separated from the bark. A moment later it
hit the repurposed head of the caterpillar. Thanks to the perfect aim, almost
all of the fuel fell into place. It flooded the upper floor through the opening
cut into the roof, then streamed down the staircase into the machine room.

Mary and Raisy had not entirely avoided the splashing liquid. They jumped
around squealing and shaking the fragrant drops from their hair. Raisy paused,
licked her lips and nodded with satisfaction.

Philip and Greg descended between them on a thread, and the time came to say
goodbye.

``Lunch is almost done,'' Greg said. ``My mother will ground me if I do not get
home in time. Take this thread. Attach it to the rocket, and we will make sure
that no matter how far you go it does not snap. Thank you for playing with me
and have a nice journey!''

``Have a nice journey, Philip,'' his mother said. ``It is very nice of you to
help Raisy. Enjoy your time on the pine tree, but come home soon!''

``Have a nice journey, Mary,'' her father said. ``You did an amazing job! Learn
well from Raisy. Take Babouche with you, he will watch out for you.''

``Have a nice journey, Raisy,'' John said. ``I~hope to see you soon and share
stories of our latest adventures!''

``Come on, get in!'' George said. ``The fuel has soaked the cotton. We have to
get going.''

Before boarding they waved to the fairies gathered around the rocket. Then
the onlookers withdrew from the Stain to avoid getting scorched by the fire of
the engine. The crowd of fairies lit the wick at the edge of the Stain and
waited for the take-off hand in hand.

``Ten, nine, eight, seven, six, five, four, three, two, one~\dots''

The fire reached the rocket, and a deep rumbling was heard from Balthazar's
head. The sound slowly grew louder, and a blinding light flooded Dewington.
Everyone on the leaf took cover, while the travelers inside the head tightened
their safety harnesses. With a sudden jerk the rocket took off.

The fire dried the Stain and even charred its center, but did not significantly
damage the leaf. The rocket rose ever more rapidly among the branches, and a
moment later only a bright pinpoint of light was visible from the leaf. The
spider thread they had glued to it had skyrocketed.

``We are flying!'' they cheered in Balthazar's head.

``The trajectory is good too. We only need to turn three degrees to the left,''
Raisy said as she excitedly compared measurements and calculations. George
pulled on a handle that adjusted the inclination of an external flap and
changed their orientation a little.

From the cockpit Philip and Mary watched the scenery unfolding below them,
their hearts overflowing with the joy of discovery.

\begin{center}
The End
\end{center}

% Make sure the last page is not facing the inner cover.
\cleartoverso
\cleartorecto

\end{document}
